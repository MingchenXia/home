\documentclass[12pt,a4paper,notitlepage]{article}

\usepackage{amsmath,amscd,amssymb,amsthm,mathrsfs,amsfonts,layout,indentfirst,graphicx,caption,mathabx, stmaryrd,appendix,calc,imakeidx,upgreek} % mathabx for \widecheck
\usepackage{ulem} %wave underline
\usepackage{palatino}  %template
\usepackage{slashed} % Dirac operator
\usepackage{mathrsfs} % Enable using \mathscr
%\usepackage{eufrak}  another template/font
\usepackage{extarrows} % long equal sign, \xlongequal{blablabla}
\usepackage{enumitem} % enumerate label change e.g. [label=(\alph*)]  shows (a) (b) 

\usepackage{fancyhdr} % date in footer


\usepackage[nottoc]{tocbibind}   % Add  reference to ToC


\makeindex


% The following set up the line spaces between items in \thebibliography
\usepackage{lipsum}
\let\OLDthebibliography\thebibliography
\renewcommand\thebibliography[1]{
	\OLDthebibliography{#1}
	\setlength{\parskip}{0pt}
	\setlength{\itemsep}{2pt} 
}


\allowdisplaybreaks  %allow aligns to break between pages
\usepackage{latexsym}
\usepackage{chngcntr}
\usepackage[colorlinks,linkcolor=blue,anchorcolor=blue, linktocpage,
%pagebackref
]{hyperref}
\hypersetup{ urlcolor=cyan,
	citecolor=[rgb]{0,0.5,0}}


%\setcounter{tocdepth}{1}	 hide subsections in the content

\usepackage{fullpage}
\counterwithin{figure}{section}

\pagestyle{plain}

\captionsetup[figure]
{
	labelsep=none	
}


\theoremstyle{definition}
\newtheorem{df}{Definition}[section]
\newtheorem{eg}[df]{Example}
\newtheorem{exe}[df]{Exercise}
\newtheorem{rem}[df]{Remark}
\newtheorem{ass}[df]{Assumption}
\newtheorem{cv}[df]{Convention}
\newtheorem{nota}[df]{Notation}
\newtheorem{st}{Step}
\theoremstyle{plain}
\newtheorem{thm}[df]{Theorem}
\newtheorem{thd}[df]{Theorem-Definition}
\newtheorem{pp}[df]{Proposition}
\newtheorem{co}[df]{Corollary}
\newtheorem{lm}[df]{Lemma}
\newtheorem{cond}{Condition}
\renewcommand{\thecond}{\Alph{cond}} % "letter-numbered" theorems



%\substack   multiple lines under sum
%\underset{b}{a}   b is under a


% Remind: \overline{L_0}



\newcommand{\fk}{\mathfrak}
\newcommand{\mc}{\mathcal}
\newcommand{\wtd}{\widetilde}
\newcommand{\wht}{\widehat}
\newcommand{\wch}{\widecheck}
\newcommand{\ovl}{\overline}
\newcommand{\udl}{\underline}
\newcommand{\tr}{\mathrm{t}} %transpose
\newcommand{\Tr}{\mathrm{Tr}}
\newcommand{\End}{\mathrm{End}} %endomorphism
\newcommand{\id}{\mathbf{1}}
\newcommand{\Hom}{\mathrm{Hom}}
\newcommand{\Conf}{\mathrm{Conf}}
\newcommand{\Res}{\mathrm{Res}}
\newcommand{\KZ}{\mathrm{KZ}}
\newcommand{\ev}{\mathrm{ev}}
\newcommand{\coev}{\mathrm{coev}}
\newcommand{\opp}{\mathrm{opp}}
\newcommand{\Rep}{\mathrm{Rep}}
\newcommand{\diag}{\mathrm{diag}}
\newcommand{\Dom}{\scr D}
\newcommand{\loc}{\mathrm{loc}}
\newcommand{\con}{\mathrm{c}}
\newcommand{\uni}{\mathrm{u}}
\newcommand{\ssp}{\mathrm{ss}}
\newcommand{\di}{\slashed d}
\newcommand{\Diffp}{\mathrm{Diff}^+}
\newcommand{\Diff}{\mathrm{Diff}}
\newcommand{\PSU}{\mathrm{PSU}}
\newcommand{\Vir}{\mathrm{Vir}}
\newcommand{\Witt}{\mathscr W}
\newcommand{\Span}{\mathrm{Span}}
\newcommand{\pri}{\mathrm{p}}
\newcommand{\ER}{E^1(V)_{\mathbb R}}
\newcommand{\bk}[1]{\langle {#1}\rangle}
\newcommand{\GA}{\mathscr G_{\mathcal A}}
\newcommand{\vs}{\varsigma}
\newcommand{\Vect}{\mathrm{Vec}}
\newcommand{\Vectc}{\mathrm{Vec}^{\mathbb C}}
\newcommand{\scr}{\mathscr}
\newcommand{\sjs}{\subset\joinrel\subset}
\newcommand{\Jtd}{\widetilde{\mathcal J}}
\newcommand{\gk}{\mathfrak g}
\newcommand{\hk}{\mathfrak h}
\newcommand{\xk}{\mathfrak x}
\newcommand{\yk}{\mathfrak y}
\newcommand{\zk}{\mathfrak z}
\newcommand{\hr}{\mathfrak h_{\mathbb R}}
\newcommand{\Ad}{\mathrm{Ad}}
\newcommand{\DHR}{\mathrm{DHR}_{I_0}}
\newcommand{\Repi}{\mathrm{Rep}_{\wtd I_0}}
\newcommand{\im}{\mathbf{i}}
\newcommand{\Co}{\complement}
%\newcommand{\Cu}{\mathcal C^{\mathrm u}}
\newcommand{\RepV}{\mathrm{Rep}^\uni(V)}
\newcommand{\RepA}{\mathrm{Rep}^\uni(A)}
\newcommand{\RepAU}{\mathrm{Rep}^\uni(A_U)}
\newcommand{\RepU}{\mathrm{Rep}^\uni(U)}
\newcommand{\BIM}{\mathrm{BIM}^\uni}
\newcommand{\BIMA}{\mathrm{BIM}^\uni(A)}
\newcommand{\shom}{\underline{\Hom}}
\newcommand{\divi}{\mathrm{div}}
\newcommand{\sgm}{\varsigma}
\newcommand{\SX}{S_{\fk X}}
\newcommand{\DX}{D_{\fk X}}
\newcommand{\mbb}{\mathbb}
\newcommand{\mbf}{\mathbf}
\newcommand{\blt}{\bullet}
\newcommand{\coker}{\mathrm{coker}}
\newcommand{\Vbb}{\mathbb V}
\newcommand{\Wbb}{\mathbb W}
\newcommand{\Mbb}{\mathbb M}
\newcommand{\Gbb}{\mathbb G}
\newcommand{\Cbb}{\mathbb C}
\newcommand{\Nbb}{\mathbb N}
\newcommand{\Zbb}{\mathbb Z}
\newcommand{\Pbb}{\mathbb P}
\newcommand{\Rbb}{\mathbb R}
\newcommand{\Ebb}{\mathbb E}
\newcommand{\cbf}{\mathbf c}
\newcommand{\wt}{\mathrm{wt}}
\newcommand{\Lie}{\mathrm{Lie}}
\newcommand{\btl}{\blacktriangleleft}
\newcommand{\btr}{\blacktriangleright}
\newcommand{\svir}{\mathcal V\!\mathit{ir}}
\newcommand{\Ker}{\mathrm{Ker}}
\newcommand{\Coker}{\mathrm{Coker}}
\newcommand{\Sbf}{\mathbf{S}}
\newcommand{\low}{\mathrm{low}}
\newcommand{\Sp}{\mathrm{Sp}}
\newcommand{\Rng}{\mathrm{Rng}}
\newcommand{\vN}{\mathrm{vN}}







\numberwithin{equation}{section}




\title{Spectral Theory for Strongly Commuting Normal Closed Operators}
\author{{\sc Bin Gui}
	%\\
	%{\small Department of Mathematics, Rutgers university}\\
	%{\small bin.gui@rutgers.edu}
}
\date{}
\begin{document}\sloppy % avoid stretch into margins
	\pagenumbering{arabic}
	%\pagenumbering{gobble}
	%\newpage
	%\setcounter{page}{1}
	%\setcounter{section}{-1}
	%\setcounter{equation}{6}
	
	
	
	\maketitle
	
	
%%%%%%%%%%%%%%%%%%%%%%%%%%%%%%%%%%%%%%%%%%%%%%%5
\newcommand\blfootnote[1]{%
	\begingroup
	\renewcommand\thefootnote{}\footnote{#1}%
	\addtocounter{footnote}{-1}%
	\endgroup
}
% Footnote without marker

%%%%%%%%%%%%%%%%%%%%%%%%%%%%%%%%%%%%%%%%%%%%%

\tableofcontents

\vspace{-0.5cm}
\blfootnote{Last major revision:  2021.9}

%%%%%%%%%%%%%%%%%%%%%%%%%%%%%%%%5
%\makeatletter
%\newcommand*{\toccontents}{\@starttoc{toc}}
%\makeatother
%\toccontents

% title and table of contents same page

%%%%%%%%%%%%%%%%%%%%%%%%%%%%%
	
	
	
	

	
	

	
	
	
	
	%%%%%%%%%%%%%%%%%%%%%%%%%%%%%%%%%%%%%%%%%%%%%%%%%%%%%%%%%
	
	%\newpage
	%$~$
	%\renewcommand\contentsname{} % the empty name
	
	%\begingroup
	%\let\clearpage\relax
	%\vspace{-2cm} % the removed space. Set as appropriate
	
	
	% Remove header of table of contents
	
	%%%%%%%%%%%%%%%%%%%%%%%%%%%%%%%%%%%%%%%%%%%%%%%%%%%%%%%

	
\newpage


\section*{Preface}


The goal of this monograph is to give a detailed and self-contained account of the spectral theory for strongly commuting normal closed operators on a Hilbert space $\mc H$, and their (bounded and unbounded) Borel functional calculus. We assume the readers are familiar with general topology (as in \cite{Mun}), measure theory and basic Hilbert space theory (as in \cite{Rud-R}), and basic properties of bounded linear operators between Hilbert spaces (recalled in Section \ref{lb63}). No previous knowledge on the general theory of Banach spaces or locally convex spaces is assumed.

Our approach in this monograph has the following features:

The crucial step in the proof of spectral theorem for adjointly commuting normal bounded operators is to establish an inequality for polynomial functional calculus as in Prop. \ref{lb1}. Unlike many approaches, ours relies neither on Gelfand-Naimark theorem nor on Gelfand's formula $\sup\{|\lambda|:\lambda\in\Sp(T)\}=\lim_{n\rightarrow\infty}\lVert T^n\lVert^{1/n}$ for a bounded operator $T$. Instead, we prove it by establishing the algebraic properties of holomorphic functional calculus; see Thm. \ref{lb2}. We in turn give a new proof of Gelfand-Naimark theorem; see exercise \ref{lb64}.

Before introducing the theory of unbounded closed operators, we first establish the spectral theory for unbounded positive operators, i.e., those unbounded $T$ on $\mc H$ satisfying that $\bk{T\xi|\xi}\geq 0$ for each $\xi\in\mc H$ and that $1+T$ is surjective. This is easy, since we have spectral theorem for the bounded positive operator $(1+T)^{-1}$.

Our treatment of the general theory of closed and preclosed operators rely on polar decomposition (Thm. \ref{lb22}), which factors a closed operator as the product of a partial isometry and a positive operator. Thus, the spectral theory for strongly commuting normal closed operators follows from that for adjoint commuting partial isometries and bounded positive operators, which are established in Section \ref{lb65}. Our preference for the method of polar decomposition is due to the fact that it is also an important tool in the study of non-normal closed operators, or more generally, non-abelian von Neumann algebras.

We give in this monograph an introduction to the strong commutativity of unbounded closed operators, not assuming they are normal. (Indeed, normality can be understood using strong commutativity; see Def. \ref{lb52}). von Neumann algebras appear naturally in the study of strong commutativity. See Section \ref{lb66}. On the other hand, we introduce von Neumann algebras mainly to understand strong commutativity. Unlike \cite{Kad}, von Neumann algebras are not widely used in our proofs of spectral theorem and Borel functional calculus. The readers who are not interested in von Neumann algebras can skip the sections on strong commutativity, and read the proof of spectral theorem (Thm. \ref{lb35})  by assuming there  is only one normal operator $T$.

We present spectral theorem in the ``multiplication form": that is, strongly commuting normal closed operators $T_1,\dots,T_N$ are simultaneously unitarily equivalent to the multiplication of the coordinate functions $z_1,\dots,z_N$ on $\bigoplus_{n\in\fk N} L^2(\Cbb^N,\mu_n)$, where $\{\mu_n\}_{n\in\fk N}$ is a collection (indexed by a non-necessarily finite or countable set $\fk N$) of positive finite (necessarily Radon) Borel measure on $\Cbb^N$. This is in the same spirit as \cite{RS}, but slightly more general. Spectral theorem in the ``resolution of the identity" form follows easily from the multiplication form.

As one can see in Sections \ref{lb67} and \ref{lb68}, relations like $A_2T\subset TA_1$ (where $T$ is an unbounded (pre)closed operator from $\mc H_1$ to $\mc H_2$ (with dense domain), $A_1,A_2$ are bounded linear operators on $\mc H_1,\mc H_2$ respectively) have important analytic consequences.

Our presentation of spectral theory is influenced by \cite{Kad,RS,Rud-F}. These texts focus mainly on single normal operators rather than several strongly commuting ones (especially when treating unbounded operators). Besides the present monograph, we also recommend \cite{Sch} for a text on spectral theory which treats several unbounded operators.







\section{Preliminaries}\label{lb63}

We set $\Nbb=\{0,1,2,3,\dots\}$, $\Zbb_+=\{1,2,3,\dots\}$. 
%For each $r\geq 0$, we define discs ${\mc D_r}=\{z\in\Cbb:|z|<r\}$ and $\ovl{\mc D_r}=\{z\in\Cbb:|z|\leq r\}$.\index{D@$\mc D_r,\ovl{\mc D_r}$}


\subsection*{Nets}

A \textbf{directed set} $\fk A$ is a set equipped with a binary relation $\leq$ which is reflexive ($\alpha\leq \alpha$ for all $\alpha\in\fk A$), transitive (for each $\alpha,\beta,\gamma\in\fk A$, if $\alpha\leq \beta$ and $\beta\leq\gamma$ then $\alpha\leq\gamma$), and satisfies that for any $\alpha,\beta\in\fk A$, there is $\gamma\in\fk A$ such that $\alpha,\beta\leq\gamma$.

A \textbf{net} of elements in a set $X$ is a function from $\fk A$ to $X$, written as $(x_\alpha)_{\alpha\in\fk A}$ or simply $x_\blt$. Assume $X$ is a topological space. Then for each $x\in X$, we write
\begin{align}
\lim_\alpha x_\alpha=x,	
\end{align}
or simply $\lim x_\blt=x$, if for each neighborhood $U$ of $x$ there is $\alpha\in\fk A$ such that $x_\beta\in U$ for all $\beta\geq \alpha$. A map $f:X\rightarrow Y$ (where $X$ and $Y$ are topological spaces) is continuous at $x\in X$ if and only if for each net $x_\blt$ converging to $x$, $f(x_\blt)$ converges to $f(x)$. For the ''if" part, one suffices to choose the directed set $\fk A$ to be the set of neighborhoods containing $x$ with $\leq$ being $\supset$.

If $x_\blt$ is a net in a \textit{Hausdorff space} $X$, then any two limits of $x_\blt$ are equal.

If $\Omega$ is a subset of a topological space $X$, then the closure of $\Omega$ is the set of all $x\in X$ such that there is net $x_\blt\in X$ converging to $x$. When $X$ is first countable, nets can be replaced by sequences.

We refer the readers to \cite[Chapter 3]{Mun} for more  about nets.

\subsection*{Hilbers spaces and bounded operators}

The sesquilinear norm on a Hilbert space will be denoted by $\bk{\cdot|\cdot}$, where the left bracket is linear and the right one antilinear. 

Given Hilbert spaces $\mc H_1,\mc H_2$, we let $\Hom(\mc H_1,\mc H_2)$ \index{Hom@$\Hom(\mc H_1,\mc H_2)$} denote the space of bounded linear maps from $\mc H_1$ to $\mc H_2$. $\Hom(\mc H,\mc H)$ is denoted by $\End(\mc H)$.\index{End@$\End(\mc H)$} 

$\Hom(\mc H_1,\mc H_2)$ is a Banach space equipped with the operator norm $\lVert T\lVert=\sup_{0\neq\xi\in\mc H} \frac{\lVert T\xi\lVert}{\lVert\xi\lVert}=\sup_{0\neq\xi,\eta\in\mc H}\frac {\bk{T\xi|\eta}}{\lVert\xi\lVert \lVert\eta\lVert}$. If $S\in\Hom(\mc H_2,\mc H_3)$, then $\lVert ST\lVert\leq\lVert S\lVert \lVert T\lVert$.

The adjoint $T^*$ of $T$ is in $\Hom(\mc H_2,\mc H_1)$ and defined by $\bk{T\xi|\eta}=\bk{\xi|T^*\eta}$ for each $\xi,\eta\in\mc H$. (I.e., $T^*\eta$ is the unique vector corresponding to the bounded linear functional $\xi\mapsto\bk{T\xi|\eta}$, whose existence is guaranteed by Riesz representation theorem.) It is clear that $\lVert T^*\lVert=\lVert T\lVert$. We also have the $C^*$-property
\begin{align}\label{eq3}
	\lVert T^*T\lVert=\lVert T\lVert^2.
\end{align}
Indeed, $\leq$ follows from the above general inequality for $\lVert ST\lVert$. And $\lVert T\lVert^2=\sup_{\xi\neq 0} \lVert T\xi\lVert^2/\lVert\xi\lVert^2=\sup_{\xi\neq 0} \bk{TT^*\xi|\xi}/\lVert\xi\lVert^2\leq \lVert T^*T\lVert$.




The kernal and the range of $T\in\Hom(\mc H_1,\mc H_2)$ are denoted respectively by $\Ker(T)$ and $\Rng(T)$. We have
\begin{align}
\Rng(T)^\perp=\Ker (T^*).\label{eq19}	
\end{align}
Indeed, $\xi\perp\Rng(T)$ iff $0=\bk{\xi|T\eta}=\bk{T^*\xi|\eta}$ for each $\eta$, iff $T^*\xi=0$.

$T\in\End(\mc H)$ is called \textbf{normal} resp. \textbf{self-adjoint} resp. \textbf{positive}  if $T^*T=TT^*$ resp. $T=T^*$ (equivalently, $\bk{T\xi|\xi}\in\Rbb$ for each $\xi\in\mc H$) resp. $\bk{T\xi|\xi}\geq 0$ for each $\xi\in\mc H$.

The \textbf{strong (resp. weak) operator topology} of $\Hom(\mc H_1,\mc H_2)$ is the one generated by $\{T:\lVert (T-T_0)\xi_1\lVert<\epsilon,\dots,\lVert (T-T_0)\xi_N\lVert<\epsilon\}$ (resp. $\{T:|\bk{(T-T_0)\xi_1|\eta_1}|<\epsilon,\dots,|\bk{(T-T_0)\xi_N|\eta_N}|<\epsilon\}$) for some $N\in\Nbb$, $T_0\in\Hom(\mc H_1,\mc H_2)$,  $\epsilon>0$, $\xi_1,\dots,\xi_N\in\mc H_1,\eta_1,\dots,\eta_N\in\mc H_2$. A net $T_\blt$ in $\Hom(T_1,T_2)$ is said to \textbf{converge strongly (resp. weakly)} to $T$, if and only if they converge in the strong (resp. weak) topology, if and only if $\lim T_\blt\xi=T\xi$ (resp. $\lim\bk{T_\blt\xi|\eta}=\bk{T\xi|\eta}$) for each $\xi\in\mc H_1,\eta\in\mc H_2$. If $\lVert T_\blt\lVert\leq M$ uniformly for some $M>0$, then it suffices to verify the limit for $\xi,\eta$ in dense subspaces of $\mc H_1,\mc H_2$ respectively.

If $(\mc H_n)_{n\in\fk N}$ is a collection of Hilbert spaces indexed by a (non-necessarily countable) set $\fk N$, then  $\bigoplus_{n\in\fk N}\mc H_n$ denotes elements of the form $(\xi_n)_{n\in\fk N}$ where each $\xi_n\in\mc H_n$, and $\sum_{n\in\fk N}\lVert\xi_n\lVert^2<+\infty$. This is a Hilbert space, called the \textbf{direct sum} of $(\mc H_n)_{n\in\fk N}$. The vector space structure is defined componentwisely. The inner product between $(\xi_n)_{n\in\fk N}$ and $(\eta_n)_{n\in\fk N}$ is $\sum_{n\in\fk N}\bk{\xi_n|\eta_n}$.

The \textbf{Hahn-Banach separation theorem} for $\mc H$ says that if $C$ is a convex subset of $\mc H$ (i.e. $\xi,\eta\in C$ implies $a\xi+b\eta\in C$ for each $a,b\geq 0,a+b=1$), and if there is a net of vectors $\{\xi_\alpha\}_{\alpha\in\fk A}$ in $C$ converging \textbf{weakly} to $\xi$ (i.e. $\bk{\xi_\alpha|\psi}\rightarrow\bk{\xi|\psi}$ for each $\psi\in\mc H$), then $\xi$ is in the (strong) closure $\ovl C$ of $C$ (which is also convex).\footnote{Here is one way to see this without appeal to the general Hahn-Banach theorem for locally convex spaces. Assume for simplicity that $\xi=0$. Note that $\ovl C$ has a vector $\psi$ such that $\lVert\psi\lVert=\inf_{\eta\in \ovl C}\lVert \eta\lVert$ (\cite[Thm. 4.10]{Rud-R}). Consider $\mc H_\Rbb$ as the real Hilbert space $\mc H$ with inner product $\bk{\cdot|\cdot}_\Rbb:=\mathrm{Re}\bk{\cdot|\cdot}$.  Then for each $\alpha$, we have $\bk{\xi_\alpha|\psi}_\Rbb\geq\lVert \psi\lVert^2$; otherwise, by looking at the (at most two dimensional) real subspace spanned by $\xi_\alpha$ and $\psi$, we see that there must be a vector on the line segment between $\xi_\alpha$ and $\psi$ (and hence inside $\ovl C$) whose length is $<\lVert\psi\lVert$, impossible. Since $\lim\bk{\xi_\blt|\psi}_\Rbb=0$, we must have $\lVert \psi\lVert^2=0$ and hence $0=\psi\in\ovl C$.}

\subsection*{Projections and partial isometries}


A \textbf{projection} $E$ on a Hilbert space $\mc H$ is a bounded linear map which fixes vectors on a closed subspace $\mc H_0$ of $\mc H$, and maps all vectors in $\mc H_0^\perp$ to $0$. Then $\Rng(E)=\mc H_0$. We say $E$ is \textbf{projection} of $\mc H$ onto $\mc H_0$. Then $1-E$ is the projection of $\mc H$ onto $\mc H_0^\perp$. 

Equivalently, a projection $E$ is a bounded linear operator satisfying $E^*=E$ and $E^2=E$. It fixes vectors in $\Rng(E)$ and acts trivially on $\Rng(E)^\perp$.

 



The range of a projection $E$ is necessarily closed, and we have $\Ker(E)=\Rng(E)^\perp$. 

A net $E_\blt$ of projections on $\mc H$ is called increasing if $\alpha\leq\beta$ implies $\Rng(E_\alpha)\subset\Rng(E_\beta)$, or equivalently, $E_\alpha=E_\alpha E_\beta$.

\begin{pp}
An increasing net of projections $(E_\alpha)_{\alpha\in\fk A}$ on $\mc H$ always converge strongly. If $\mc H_0=\ovl{\bigcup \Rng(E_\blt)}$, then $E_\blt$ converges strongly to the projection $E$ onto $\mc H_0$.
\end{pp}

\begin{proof}
$E_\blt\xi=0=E\xi$ when $\xi\perp\mc H_0$. Now assume $\xi\in\mc H_0$. For each $n\in\Zbb_+$, suppose we have found $\alpha_1,\cdots,\alpha_{n-1}\in\fk A$  such that $\lVert \xi-E_{\alpha_j}\xi\lVert \leq 1/j$ for each $1\leq j\leq n-1$. Since $\Rng(E_\blt)$ is increasing and has union dense in $\mc H_0$, we can find $\alpha_n\in\fk A$ and a vector $\eta_n\in\Rng(E_{\alpha_n})$ such that $\lVert\xi-\eta_n\lVert<1/n$. Since the smallest distance between $\xi$ and a vector in $\Rng(E_{\alpha_n})$ is $\lVert\xi-E_{\alpha_n}\xi\lVert$, we conclude $\lVert\xi-E_{\alpha_n}\xi\lVert<1/n$.

Now that the sequence $\alpha_n$ is constructed, for each $n\in\Zbb_+$, we have $\lVert\xi-E_\beta\xi\lVert<1/n$ for every $\beta\geq\alpha_n$, again by the fact that $E_\blt$ is increasing. So $E_\blt\xi\rightarrow \xi=E\xi$. This finishes the proof.
\end{proof}





A \textbf{unitary} operator/map $U$ from $\mc H_1$ to $\mc H_2$ is by definition a bounded linear map which is bijective and preserves inner products ($\bk {U\xi|U\eta}=\bk{\xi|\eta}$ for each $\xi,\eta\in\mc H$), equivalently, preserves the norms ($\lVert U\xi\lVert=\lVert \xi\lVert$ for each $\xi\in\mc H$). Equivalently, $U\in\Hom(\mc H_1,\mc H_2)$ satisfies $U^*U=\id_{\mc H_1},UU^*=\id_{\mc H_2}$. $U^*$ is a unitary map from $\mc H_2$ to $\mc H_1$. A unitary $U:\mc H_1\rightarrow\mc H_2$ is an equivalence of the two Hilbert spaces. 


A \textbf{partial isometry} $U$ from a Hilbert space $\mc H_1$ to another $\mc H_2$ is by definition a bounded linear operator, which restricts to a unitary map
\begin{align*}
U:\sgm(U)\xrightarrow{\simeq}\tau(U)	
\end{align*}
from a closed subspace $\sgm(U)$ of $\mc H_1$ to a (necessarily closed) subspace $\tau(U)$ of $\mc H_2$, and which is zero when acting on $\sgm(U)^\perp$. We say $\varsigma(U)$ is the \textbf{source space} of $U$, and $\tau(U)$ the \textbf{target space} of $U$. \index{zz@$\varsigma(U),\tau(U)$} Note that $\Rng(U)=\tau(U)$. $U^*$ is a partial isometry with source space $\varsigma(U^*)=\tau(U)$ and target space $\tau(U^*)=\varsigma(U)$, $U^*$ restricts to a unitary map
\begin{align*}
U^*:\tau(U)\xrightarrow{\simeq}\sgm(U)	
\end{align*}
which is the inverse of the above restriction of $U$.

$U^*U$ is the projection of $\mc H_1$ onto $\varsigma(U)$, and $UU^*$ is the projection of $\mc H_2$ onto $\tau(U)$. (As a consequence, we have $U=UU^*U$ and $U^*=U^*UU^*$.)

Equivalently, a partial isometry $U:\mc H_1\rightarrow\mc H_2$ is defined to be a bounded operator such that both $U^*U$ and $UU^*$ are projections. Then $U^*U$ is the projection of $\mc H_1$ onto $\varsigma(U)$, and $UU^*$ is the projection of $\mc H_2$ onto $\tau(U)$. 

We leave it to the readers to check the equivalence of definitions.

A partial isometry $U:\mc H_1\rightarrow\mc H_2$ whose source space is $\mc H_1$ is called an \textbf{isometry}.

\subsection*{Borel measures}



A positive measure $\mu$ on a locally compact Hausdorff space $X$ is called \textbf{Radon measure} when: (local finiteness)$\mu$ is finite on compact subsets; (outer regularity) for each Borel set $E\subset X$, $\mu(E)=\inf\{\mu(U):E\subset U,U\text{ is open}\}$; (inner regularity on open sets) if $U\subset X$ is open,   then $\mu(U)=\sup\{\mu(K):K\subset U,K\text{ is compact}\}$.  

A Radon measure is automatically inner regular on any $\sigma$-finite Borel subset, cf.  \cite[Prop. 7.5]{Fol}.

The \textbf{Riesz-Markov representation theorem} \cite[Thm.2.14]{Rud-R} says that any positive linear functional on $C_c(X)$ (the algebra of continuous functions on $X$ with compact supports) can be written as $f\mapsto\int_Xfd\mu$ for a unique Radon measure $\mu$ on $X$.

All measures are positive unless otherwise stated (that it is a complex measure). A complex Radon measure is by definition a finite $\Cbb$-linear combination of \textit{finite} Radon measures. 

In this note, we only consider  locally compact Hausdorff spaces $X$ which are also \textit{second countable}. (The only exception is the proof of Gelfand-Naimark Theorem, which is not used elsewhere in the note.) Then local finiteness implies inner and outer regularity \cite[Thm. 2.18]{Rud-R}. Thus, \textit{locally finite (in particular, finite) positive Borel measures on $X$ are automatically Radon measures}.

Suppose $\phi:X\rightarrow Y$ is a measurable map between two measure spaces $X,Y$. If $\mu$ is a measure on $X$, its \textbf{pushforward} $\phi_*\mu$ is defined by
\begin{align*}
\phi_*\mu(\Omega)=\mu(\phi^{-1}(\Omega))	
\end{align*}
for each measurable $\Omega\subset Y$. Then for each  measurable $f:Y\rightarrow[0,+\infty]$ we have
\begin{align}
\int_Y f~d\phi_*\mu=\int_X f\circ\phi ~d\mu.	\label{eq16}
\end{align}
Indeed, this is obvious when $f$ is a characteristic function $\chi_\Omega$, and hence true when $f$ is a step function. So it is true in general by monotone convergence theorem.

The \textbf{support} of a Borel  measure $\mu$ on a topological space is the (necessarily closed) subset of all points $x$ such that any neighborhood at $x$ has non-zero measure.



\section{Spectral theory for adjointly commuting normal bounded operators}\label{lb65}



Fix a Hilbert space $\mc H$. Choose \textbf{adjointly commuting normal operators} $T_1,\dots,T_N\in\End(\mc H)$. This meas that  $T_iT_j=T_jT_i$ and $T_i^*T_j=T_jT_i^*$ for each $i,j$.   Set $\Re(T)=(T+T^*)/2$ and $\Im (T)=(T-T^*)/{2\im}$ for each $T\in\End(\mc H)$. Then $\Re (T_1),\dots,\Re (T_N),\Im (T_1),\dots,\Im (T_N)$ are (adjointly) commuting self-adjoint operators. Moreover, a polynomial of $T_\blt,T^*_\blt$ is equivalently a polynomial of $\Re (T_\blt),\Im (T_\blt)$.

Let $\mc P_N$ \index{PN@$\mc P_N$} denote the set of polynomials with complex coefficients and mutually commuting and independent formal variables $z_1,\dots,z_N,\ovl z_1,\dots,\ovl z_N$. Thus, a generic element is a finite sum of elments of the form $az_1^{m_1}\cdots z_N^{m_N}\cdot \ovl z_1^{n_1}\cdots \ovl z_N^{n_N}$ where $a\in\Cbb$ and $m_1,\dots,m_N,n_1,\dots,n_N\in \Nbb$. Multiplications of $\mc P_N$ are defined as multiplications of polynomials. Moreover, there is an involution $*$ (i.e., an anti-linear\footnote{An antilinear map $T$ between two $\Cbb$-vector spaces $V$ and $W$ is a map satisfying $T(au+bv)=\ovl a Tu+\ovl b TV$ for each $a,b\in\Cbb,u,v\in V$.} satisfying $f^{**}=f$) such that
\begin{align*}
(az_1^{m_1}\cdots z_N^{m_N}\cdot \ovl z_1^{n_1}\cdots \ovl z_N^{n_N})^*=\ovl a\cdot \ovl z_1^{m_1}\cdots \ovl z_N^{m_N}\cdot z_1^{n_1}\cdots z_N^{n_N}.	
\end{align*}
In this way, $\mc P_N$ becomes a $*$-algebra.

By changing variables $z_j=x_j+\im y_j,\ovl z_j=x_j-\im y_j$, elements $f$ of $\mc P_N$ are equivalently polynomials $\uwave f$  of $x_\blt,y_\blt$, related by
\begin{gather*}
\uwave f(x_1,y_1,\dots,x_N,y_N)=f(x_1+\im y_1,x_1-\im y_1,\dots,x_N+\im y_N,x_N-\im y_N),	\\
f(z_1,\ovl z_1,\dots,z_N,\ovl z_N)=\uwave f(\Re(z_1),\Im(z_1),\dots,\Re(z_N),\Im(z_N))	
\end{gather*}
where $\Re(z_j)=(z_j+\ovl z_j)/2,\Im(z_j)=(z_j-\ovl z_j)/(2\im)$. We have $x_j^*=x_j,y_j^*=y_j$. %We will not distinguish between $f$ and $\uwave f$ in the following.


Note that $\End(\mc H)$ is also a $*$-algebra whose involution $*$ is given by the adjoint of operators. We have a unique unital \textbf{$*$-homomorphism} $\pi:\mc P_N\rightarrow\End(\mc H)$ defined by
\begin{align}
\pi(z_j)=T_j \label{eq1}
\end{align}
for each $j$. Equivalently,
\begin{align*}
\pi (x_j)=\Re(T_j),\qquad \pi(y_j)=\Im(T_j).	
\end{align*}
By (unital) $*$-homomorphism, we mean
\begin{gather}
\pi(1)=\id,\qquad 	\pi(fg)=\pi(f)\pi(g),\qquad \pi(f^*)=\pi(f)^*\label{eq15}
\end{gather}
for each $f,g\in\mc P_N$. 


We know that in Linear algebra, polynomial rings play an important role in the study of spectral theory/Jordan decomposition. In the infinite-dimensional case, polynomial rings (or more precisely, polynomial $*$-algebras) are not sufficient. We need to consider $C(X)$ \index{CX@$C(X),C_c(X)$}, the $*$-algebra of continous functions on a compact Hausdorff space $X$. It's $*$-structure is given by $f^*(x)=\ovl{f(x)}$ for each $f\in C(X),x\in X$. $\ovl{\cdot}$ stands for the complex conjugate. Moreover, $C(X)$ is equipped with the norm topology $L^\infty(X)$.


The first question in spectral theory is: what is an appropriate $X$? To answer this question, we need the following crucial result. For each $r>0$, set \index{Br@$B_r=[-r,r]^2$}
\begin{align}
B_r=[-r,r]^2	
\end{align}
regarded as a subset of $\Cbb$.

\begin{pp}\label{lb1}
Let $r_j$ be $\lVert T_j\lVert$, the operator norm of $T_j$. Then for each $f\in\mc P_N$, 
\begin{align}\label{eq25}
\lVert \pi(f)\lVert\leq \sup_{x_j,y_j\in[-r_j,r_j]}| \uwave f(x_1,y_1,\dots,x_N,y_N)|.	
\end{align}
\end{pp}


Let
\begin{align*}
X=B_{r_1}\times\cdots \times B_{r_N}.
\end{align*}

\begin{thm}\label{lb17}
There is a unique  continuous (unital) $*$-homomorphism $\pi:C(X)\rightarrow\End(\mc H)$ satisfying that for each $j$, $\pi(z_j)=T_j,\pi(\ovl z_j)=T_j^*$. (Equivalently, $\pi(x_j)=\Re(T_j),\pi(y_j)=\Im (T_j)$.)
\end{thm}


\begin{proof}
For each  $f\in\mc P_N$, $\uwave f$ can be regarded as a continuous function of the variables $x_1,y_1,\dots,x_N,y_N$. So $\uwave f\in C(X)$. Moreover, $\uwave f$ as a polynomial is determined by $\uwave f$ as a function (since all the coefficients of the polynomial can be calculated by the values of the multi partial derivatives of $\uwave f$). Thus, $\mc P_N$ is identified with a unital $*$-subalgebra of $C(X)$ by identifying $f\in\mc P_N$ with $\uwave f\in C(X)$.

By Stone-Weierstrass theorem, $\mc P_N$ is dense in $C(X)$. Moreover, by Prop. \ref{lb1}, the $\pi$ defined on $\mc P_N$ is continuous with respect to the norm of $C(X)$. Therefore, $\pi$ can be extended uniquely to a continuous unital $*$-homomorphism from $C(X)$ to $\End(\mc H)$.
\end{proof}


To prepare for the proof of Prop. \ref{lb1}, let us consider a slightly different $*$-algebra $\mc Q_N=\Cbb[t_1,\dots,t_N]$ of polynomials of $t_1,\dots,t_N$, and the involution $*$ is defined by $t_j^*=t_j$ for each $j$. 

\begin{lm}\label{lb69}
Suppose that for every $N\in\Zbb_+$ and every self-adjoint bounded operators $H_1,\dots,H_N$ on $\mc H$, the unique unital $*$-homomorphism $\varphi:\mc Q_N\rightarrow\End(\mc H)$ defined by $\varphi(H_j)=t_j$ for every $1\leq j\leq N$ satisfies (by setting $r_j=\lVert H_j\lVert$)
\begin{align}
\lVert \varphi(f)\lVert \leq\sup_{t_j\in[-r_j,r_j]}|f(t_1,\dots,t_N)|\label{eq24}
\end{align}
for every $f\in\mc Q_N$. Then Prop. \ref{lb1} is true.
\end{lm}

\begin{proof}
Let us assume the condition in this theorem. Choose any $f\in\mc P_N$. Let $T_1,\dots,T_N$ be adjointly commuting normal bounded operators, and let $H_j=\Re(T_j)$, $K_j=\Im(T_j)$. Let $\varphi:\mc Q_{2N}\rightarrow\End(\mc H)$ be the unique unital $*$-homomorphism sending each $H_j$ to $t_{2j-1}$ and $K_j$ to $t_{2j}$. Consider the polynomial $\uwave f(x_1,y_1,\dots,x_N,y_N)$ as an element of $\mc Q_{2N}$ by identifying $x_j=t_{2j-1},y_j=t_{2j}$. Then $\varphi(\uwave f)=\pi(f)$ for all $f\in\mc P_N$ since this is true when $f=x_1,y_1,\dots,x_N,y_N$.

Let $r_j=\lVert T_j\lVert$. Then $\lVert H_j\lVert,\lVert K_j\lVert\leq r_j$. By \eqref{eq24}, we have
\begin{align*}
\lVert\pi(f)\lVert=\lVert \varphi(\uwave f)\lVert \leq\sup_{x_j,y_j\in[-r_j,r_j]}|\uwave f(x_1,y_1\dots,x_N,y_N)|.	
\end{align*}
\end{proof}



The crucial step of proving Prop. \ref{lb1} is to extend the $\varphi$ in Lemma \ref{lb69} to a unital $*$-homomorphism from a larger class $\mc A$ of functions, where $\mc A$ contains  $\mc Q_N$ and the positive ``square root" (defined in a proper sense) of any $r^2-f^*f$ where $f\in\mc Q_N$ and $r>\lVert f\lVert_{L^\infty(X)}$. As we see below, $\mc A$ is the set of analytic functions defined near $X$.

\subsection*{Holomorphic functional calculus}

Let $T\in\End(\mc H)$. Define the \textbf{spectrum}
\begin{align}
	\Sp(T)=\{\lambda\in\Cbb:\lambda-T\text{ is not invertible}\}.	\label{eq23}
\end{align}
Note the following easy fact:

\begin{pp}
Let $r=\lVert T\lVert$. Then $\Sp(T)\subset \{\lambda\in\Cbb:|\lambda|\leq r\}$. Moreover, if $T$ is self-adjoint, then $\Sp(T)\subset[-r,r]$.
\end{pp}


\begin{proof}
If $|\lambda|>r=\lVert T\lVert$, then $\lambda-T$ has inverse
\begin{align}
\sum_{n=0}^{+\infty}\lambda^{-n-1}T_j^n.\label{eq20}
\end{align}

Now assume $T=T^*$. We shall show that $\lambda-T$ is invertible when $\Im(\lambda)\neq 0$. By scalar multiplication and replacing $T$ by $T+a$ (where $a\in\Rbb$), it suffices to show that $S:=\im-T$ is invertible. By $T=T^*$, we have $\bk{S\xi|S\eta}=\bk{\xi|\eta}+\bk{T\xi|T\eta}$ and hence $\lVert S\xi\lVert ^2=\lVert \xi\lVert^2+\lVert T\xi\lVert^2\geq \lVert \xi\lVert^2$. So $S$ is injective (and similarly $S^*=-\im-T$ is injective), and its inverse $S^{-1}:\Rng(S)\rightarrow\mc H$ is continuous. To finish the proof, we shall show that $\Rng(S)=\mc H$. 

If $S\xi_n$ is a Cauchy sequence, then so is $\xi_n$, which converges to some $\xi\in\mc H$. So $S\xi_n\rightarrow S\xi$. This shows that $\Rng(S)$ is complete, equivalently, a closed subspace of $\mc H$.  Thus, it remains to show that $\Rng(S)$ is dense in $\mc H$. This follows because $\Rng(S)^\perp=\Ker(S^*)$ (by \eqref{eq19}) and $S^*$ is injective.
\end{proof}


\begin{lm}\label{lb4}
Let $O$ be an open subset of $\Cbb$ disjoint from $\Sp(T)$. Then the map $z\in O\mapsto (z-T)^{-1}$ is holomorphic (in the sense of Sec. \ref{lb55}).
\end{lm}



\begin{proof}
Note first of all that if $T$ is invertible and $a\in\Cbb$ satisfies $|a|\lVert T^{-1}\lVert<\frac 12 $, then $\sum_{n\in\Nbb}(-a)^nT^{-n-1}$ converges since $\lVert T^{-n-1}\lVert\leq \lVert T^{-1}\lVert^{n+1}$, and the limit is clearly the inverse of $a+T$. It is easy to check, using the infinite sum expression for $(a+T)^{-1}$, that
\begin{align*}
\lVert(a+T)^{-1}\lVert\leq 2\lVert T^{-1}\lVert.	
\end{align*}
	
Now,   assume $z\in O$ so that $z-T$ is invertible. Using the above inequality, we see that for any $h$ such that $z+h\in O$, the operator $(z+h-T)^{-1}-(z-T)^{-1}$, which clearly equals $-h(z-T)^{-1}(z+h-T)^{-1}$, has norm bounded by
\begin{align*}
|h|\lVert (z-T)^{-1}\lVert	\lVert (h+z-T)^{-1}\lVert\leq 2|h|\lVert (z-T)^{-1}\lVert^2
\end{align*}
whenever $|h|\lVert (z-T)^{-1}\lVert<1/2$. This shows that the map $z\in O\mapsto (z-T)^{-1}$ is continuous.
	
Finally, we compute the derivative: as $h\rightarrow 0$,
\begin{align*}
\frac{(z+h-T)^{-1}-(z-T)^{-1}}{h}	=-(z-T)^{-1}(z+h-T)^{-1}
\end{align*}
converges to $-(z-T)^{-2}$ by the continuity proved in the previous paragraph.
\end{proof}




We now let $H_1,\dots,H_N$ be self-adjoint operators on $\mc H$ with operator norms $r_1,\dots,r_N$. Let
\begin{align*}
Y=[-r_1,r_1]\times\cdots\times[-r_N,r_N]	
\end{align*}
Let $\mc A$ be the set of complex analytic functions $f(t_1,\dots,t_N)$ analytic on a connected neighborhood of $(t_1,\dots,t_N)\in Y$. (Thus, here we understand $t_\blt,t_\blt$ as complex variables. By saying that $f$ is analytic, we mean that it is continous as a multi-variable functions, and that it is holomorphic on each variable.) $\mc A$ is a $*$-algebra, whose involution $*$ is defined by.
\begin{align*}
 f^*(t_1,\dots,t_N)=\ovl{ f(\ovl t_1,\dots,\ovl t_N)}.	
\end{align*}
(Note that $f^*$ is holomorphic, cf. Prop. \ref{lb57}.) In this way, $\mc Q_N$ can be identified naturally  as a unital $*$-subalgebra of $\mc A$. Note that any $f\in\mc A$ is determined by its values on $Y$. This is due to the fact that any single-valued holomorphic function on a connected open subset of $\Cbb$ is determined by its values on a line segment inside this open subset; our case of multi-valued functions follows from induction on the number of variables. Thus, we equip $\mc A$ with the topology of $L^\infty(Y)$-norm.


For each $j$, we choose an anticlockwise piecewise-smooth simple closed curves $R_j\subset\Cbb$ such that $R_j$ surrounds $[-r_j,r_j]\supset \Sp(H_j)$,  and that $ f$ is holomorphic when $(t_1,\dots,t_N)$ is inside and also on a neighborhood of $R_1\times  \cdots\times R_N$. Define
\begin{align}
\varphi( f)=\oint_{t_j\in R_j}(2\pi\im)^{-N} f(t_1,\dots,t_N)\cdot(t_1-H_1)^{-1}\cdots(t_N-H_N)^{-1}\cdot dt_1\cdots dt_N.	\label{eq2}
\end{align}
Note that the integrand is holomorphic in the sense of Sec. \ref{lb55}, thanks to Lemma \ref{lb4}. The integral is defined as in Sec. \ref{lb55}. By complex analysis,  \eqref{eq2} is independent of the choice of $R_1,R_1',\dots,R_N,R_N'$. Also, note that $(t_i-H_i)^{-1}$ commutes with $(t_j-H_j)^{-1}$.



\begin{thm}\label{lb2}
$\varphi:\mc A\rightarrow\End(\mc H)$ is a continuous $*$-homomorphism and satisfies $\lVert\varphi( f)\lVert\leq \lVert  f\lVert_{L^\infty(X)}$ for each $ f\in\mc A$.  Moreover, if $f\in\mc Q_N$, then the $\varphi( f)$ defined by \eqref{eq2} agrees with the $\varphi(f)$ defined in Lemma \ref{lb69}. 
\end{thm}

\begin{proof}
For each connected open set $O$ containing $X$, we let $\mc A_O$ be the set  of  analytic functions on $O$. Then $\mc A$ is the union of all $\mc A_O$.

Step 1. Assume $ f=t_1^{n_1}\cdots t_N^{n_N}$, and assume each $R_j$ is a circle with radius larger than $\lVert T_j\lVert$ so that we can substitute each $(t_j-H_j)^{-1}=\sum_{n\geq 0}t_j^{-n-1}H_j^n$ into \eqref{eq2}. One then checks easily that the $\varphi( f)$ equals the $\varphi(f)$ defined by \eqref{eq1}. 

Step 2. We show that $\varphi$ restricts to a unital $*$-homomorphism from each $\mc A_O$ to $\End(\mc H)$. This will imply that $\varphi:\mc A\rightarrow\End(\mc H)$ is a unital $*$-homomorphism. That $\varphi(1)=1$ follows from step 1. That $\varphi(f^*)=\varphi(f)^*$ follows by applying successively Prop. \ref{lb57} to each single variable integral of the multiple integral in \eqref{eq2}. (Note that according to the notations in Prop. \ref{lb57}, if $C$ is anticlockwise then $\ovl C$ is clockwise.)

We now show $\varphi(f)\varphi(g)=\varphi(fg)$ for every $f,g\in \mc A_O$. Assume for simplicity that $N=2$. The general case of $N$ variables follows from the same method. For each $i=1,2$, we choose a smooth simple closed anticlockwise $\Gamma_i,R_i$ containing $[-r_i,r_i]$,  assume $R_i$ is small enough so that it is inside the interior of $\Gamma_i$, and assume $f$ is analytic near and inside $\Gamma_1\times \Gamma_2$. Set differential $\di=(2\im\pi)^{-1}d$. Then
\begin{align*}
&\varphi(f)\varphi(g)\\
=&\int_{x\in R_1,y\in R_2}\int_{z\in\Gamma_1,w\in \Gamma_2}f(x,y)g(z,w)(x-H_1)^{-1}(z-H_1)^{-1}(y-H_2)^{-1}(w-H_2)^{-1}\\
&\cdot\di x\di y \di z \di w
\end{align*}
Note that when $z\neq x$,
\begin{align*}
(x-H_1)^{-1}(z-H_1)^{-1}=(z-x)^{-1}(x-H_1)^{-1}-(z-x)^{-1}(z-H_1)^{-1}.
\end{align*}
Since $x\in R_1,z\in\Gamma_1$, and since $R_1$ is inside $\Gamma_1$, by complex analysis, we see that 
\begin{align*}
\int_{x\in R_1}f(x,y)g(z,w)\cdot (z-x)^{-1}(z-H_1)^{-1}\cdot (y-H_2)^{-1}(w-H_2)^{-1}\di x=0.
\end{align*}
So
\begin{align*}
&\varphi(f)\varphi(g)\\
=&\int_{x\in R_1,y\in R_2}\int_{z\in\Gamma_1,w\in \Gamma_2}(z-x)^{-1}f(x,y)g(z,w)(x-H_1)^{-1}(y-H_2)^{-1}(w-H_2)^{-1}\\
&\cdot \di x\di y \di z \di w.
\end{align*}
In this integrand, only $(z-x)^{-1}g(z,w)$ depends on $z$, and
\begin{align*}
\int_{z\in\Gamma_1}	(z-x)^{-1}g(z,w)\cdot \di z= g(x,w) 
\end{align*}
by Cauchy's formula. Thus
\begin{align*}
&\varphi(f)\varphi(g)\\
=&\int_{x\in R_1,y\in R_2}\int_{w\in \Gamma_2}f(x,y)g(x,w)(x-H_1)^{-1}(y-H_2)^{-1}(w-H_2)^{-1}\cdot \di x\di y \di w\\
=&\int_{x\in R_1}(x-H_1)^{-1}\\
&\cdot\Big(\int_{y\in R_2}\int_{w\in\Gamma_2}f(x,y)g(x,w)(y-H_2)^{-1}(w-H_2)^{-1} \di y\di w\Big)\di x.
\end{align*}
A similar factorization for $(y-H_2)^{-1}(w-H_2)^{-1}$ shows
\begin{align*}
&\int_{y\in R_2}\int_{w\in\Gamma_2}f(x,y)g(x,w)(y-H_2)^{-1}(w-H_2)^{-1} \di y\di w\\
=&\int_{y\in R_2}f(x,y)g(x,y)(y-H_2)^{-1}\di y.
\end{align*}
This proves $\varphi(f)\varphi(g)=\varphi(fg)$. 


Step 3. It remains to show that for each $ f\in\mc A$ we have $\lVert\varphi( f)\lVert\leq \lVert  f\lVert_{L^\infty(X)}$. Choose any $\delta>0$, and let $r=\delta+\lVert  f\lVert_{L^\infty(X)}$. Choose $O$ such that $ f\in\mc A_O$. Note that $r^2- f^* f$ takes values in $[\delta^2,+\infty)$ when restricted to $X$. Thus, we may choose $O$ small enough such that $r^2- f^* f$ takes values in $\Cbb\setminus(-\infty,0]$ when defined on $O$. Since we can define a holomorphic square root function $\sqrt z$ on $\Cbb\setminus(-\infty,0]$ which is positive on $(0,+\infty)$, we can define $ g=\sqrt{r^2- f^* f}$  in $\mc A_O$ which satisfies $ g^2=r^2- f^* f$ and takes positive real values on $X$. So $ g$ equals $ g^*$ on $X$ and hence on $O$. It follows that $ g^* g=r^2- f^* f$. Thus, for any $\xi\in\mc H$, we have
\begin{align*}
\lVert \varphi( f)\xi\lVert^2=\bk{\varphi( f^* f)\xi|\xi}=r^2\lVert\xi\lVert^2-\bk{\varphi( g^* g)\xi|\xi}	=r^2\lVert\xi\lVert^2-\lVert \varphi( g)\xi\lVert^2\leq r^2\lVert\xi\lVert^2.
\end{align*}
This proves $\lVert\varphi( f)\lVert^2\leq r^2=(\delta+\lVert f\lVert_{L^\infty(X)})^2$ for each $\delta>0$, hence finishes the proof.
\end{proof}

Prop.  \ref{lb1} follows immediately from the above Theorem and Lemma \ref{lb69}.


\subsection*{Spectral theorem}

We come back to the setting of adjointly commuting normal bounded operators $T_1,\dots,T_2$ on $\mc H$ with norms $r_1,\dots,r_N$. Recall that $X$ is defined by \eqref{eq25}.

The Riesz-Markov representation theorem for $C(X)$ can be presented in terms of cyclic representations. First of all, we say a vector $\xi\in\mc H$ is \textbf{cyclic} for $C(X)$, if $\pi(C(X))\xi$ spans a dense subspace of $\mc H$. If $\mu$ is a Borel measure on $X$, then $C(X)$ acts on $L^2(X,\mu)$ by multiplication.

\begin{pp}[Riesz-Markov representation theorem]\label{lb7}
Suppose that $\xi\in\mc H$ is a cyclic vector for $C(X)$. Then the representation $\pi$ of $C(X)$ on $\mc H$ is unitarily equivalent to the one of $C(X)$ on $L^2(X,\mu)$ for some Borel measure $\mu$ satisfying $\mu(X)<+\infty$. More precisely, there is a unitary map $U:\mc H\rightarrow L^2(X,\mu)$ such that $U\pi(f)U^*$ is the multiplication of $f\in C(X)$ on $L^2(X,\mu)$.

Moreover, we can choose $U$ such that $U\xi$ equals the constant function $1$.
\end{pp}


\begin{proof}
Since $\pi$ is a $*$-homomorphism, the linear functional $f\mapsto \bk{\pi(f)\xi|\xi}$ is positive since, when $f\geq 0$, we have $\bk{\pi(f)\xi|\xi}=\lVert \pi(\sqrt f)\xi\lVert^2\geq 0$. By Riesz-Markov representation theorem, we can find a finite Borel measure $\mu$ such that $\bk{\pi(f)\xi|\xi}=\int_Xfd\mu$ for each $f\in C(X)$. Thus
\begin{align*}
\bk{\pi(f)\xi|\pi(g)\xi}=\int_Xg^*fd\mu=\bk{f|g}_{L^2(X,\mu)},
\end{align*}
which shows that the linear map $\pi(f)\xi\in\pi(C(X))\xi\mapsto f\in L^2(X,\mu)$ is well-defined and extends to a unitary map $U:\mc H\rightarrow L^2(X,\mu)$ (note that the cyclic condition is used here). One checks easily that $U$ satisfies the desired property.
\end{proof}


\begin{thm}[Spectral theorem]\label{lb8}
Let $X=B_{r_1}\times\cdots\times B_{r_N}$ where each $r_j=\lVert T_j\lVert$. Then there exist a set $(\mu_n)_{n\in\fk N}$ of finite (positive) Borel measures, and also a unitary map
\begin{align*}
U:\mc H\rightarrow\bigoplus_{n\in\fk N} L^2(X,\mu_n)
\end{align*}
satisfying that for each $1\leq j\leq N$ and each $(f_n)_{n\in\fk N}\in \bigoplus_{n\in\fk N} L^2(X,\mu_n)$,
\begin{align}
UT_jU^*\cdot (f_n)_{n\in\fk N}=(z_jf_n)_{n\in\fk N}.\label{eq10}
\end{align}
\end{thm}
Here we let $z_j$ be function indicating the $j$-th component $B_j$ of $X$, i.e., the one sending $(z_1,\dots,z_N)\in B_{r_1}\times\cdots\times B_{r_N}$ to $z_j$. Thus, the spectral theorem says that the action of adjointly commuting $T_1,\dots,T_N$ on $\mc H_j$ is unitarily equivalent to the multiplication of $z_1,\dots,z_N$ on a direct sum of Borel $L^2$-spaces over $X$.

\begin{proof}
By Hausdorff maximal principle, $\mc H$ is an (orthogonal) direct sum of $C(X)$-invariant cyclic subspaces, i.e., $\mc H=\oplus_n\mc H_n$ where each subspace $\mc H_n$ is invariant under the action of $C(X)$, and the action of $C(X)$ on $\mc H_i$ possesses a cyclic vector. (Consider the partially ordered set, each element of which is a set of mutually orthogonal non-zero cyclic $C(X)$ invariant closed subspaces.)  By Proposition \ref{lb7}, each subrepresentation $\mc H_n$ is unitarily equivalent  to the multiplication of $C(X)$ on $L^2(X,\mu_n)$, such that $T_j$ is equivalent to the multiplication of $z_j$. The theorem thus follows immediately.
\end{proof}

%In the case that $\mc H$ is not separable, the above spectral theorem still holds, except that the direct sum is possibly uncountable. To obtain such a direct sum, one uses Hausdorff maximal principle to find a maximal set of mutually orthogonal closed cyclic subspaces of $\mc H$.

\subsection*{Bounded Borel functional calculus}


\begin{lm}\label{lb18}
Let $\mu$ be a finite positive Borel measure on $\Cbb^N$. Then for each bounded Borel function $f$ on $\Cbb^N$, there exists a net $g_\blt=(g_\alpha)_{\alpha\in\fk A}$ in $C_c(\Cbb^N)$, such that $\lVert g_\alpha\lVert_{L^\infty(\Cbb^N)}\leq \lVert f\lVert_{L^\infty(\Cbb^N)}$ for each $\alpha$, and that $\lim \int_{\Cbb^N}|f-g_\blt|d\mu=0$ for each finite (positive) Borel measure $\mu$ on $\Cbb^N$.
\end{lm}

\begin{proof}
Let $\fk A$ be the directed set of all $(K,\epsilon)$ where $K$ is a finite set of finite Borel measures on $\Cbb^N$, and $\epsilon>0$.  $(K_1,\epsilon_1)\leq (K_2,\epsilon_2)$ means $K_1\subset K_2$ and $\epsilon_1\geq\epsilon_2$. By Lusin's Theorem \cite[Thm. 2.24]{Rud-R}, for each $\alpha=(K,\epsilon)\in\fk A$, we can find $g_\alpha\in C(\Cbb^N)$ whose $L^\infty$-norm is bounded by $\lVert f\lVert_\infty$, such that the subset $\{x\in \Cbb^N:f(x)\neq g_\alpha(x)\}$ has $(\sum_{\mu\in K}\mu)$-measure less than $\epsilon/\lVert f\lVert_\infty$.  Then $\lVert f-g_\alpha\lVert_{L^1(\Cbb^N,\mu)}<\epsilon$ for each $\mu\in K$. So $g_\blt$ is a desired net.
\end{proof}


%\begin{df}
%Let $T_1,\dots,T_N$ have spectral decomposition as in Theorem \ref{lb8}. We say that a positive Borel measure $\nu$ on $X$ \textbf{dominates the spectral decomposition} of $T_1,\dots,T_N$, if for every Borel subset $\Omega$ of $X$ satisfying $\nu(\Omega)=0$, we have $\mu_n(\Omega)=0$ for each $\Omega$. A finite $\nu$ exists, e.g.,
%\begin{align}
%\lambda=\sum_{n=1}^\infty 2^{-n}\mu_n(X)^{-1}\mu_n.	
%\end{align}
%\end{df}

Recall $X=B_{r_1}\times\cdots B_{r_N}$.

\begin{df}\label{lb58}
For each bounded Borel function $f$ on $X$, we define a bounded operator $$\pi(f)\equiv f(T_1,\dots,T_N)$$ on $\mc H$ such that $Uf(T_1,\dots,T_N)U^*$ is the multiplication of $f$ on $\bigoplus_{n\in\fk N} L^2(X,\mu_n)$. 
\end{df}

Let $B(X)$ be the set of bounded Borel functions on $X$, which is a unital $*$-algebra, whose $*$-structure is defined by $f^*(x_1,y_1,\dots,x_N,y_N)=\ovl{f(x_1,y_1,\dots,x_N,y_N)}$.

\begin{thm}\label{lb28}
$f(T_1,\dots,T_N)$ is independent of the spectral decomposition in Theorem \ref{lb8}, and agrees with the $\pi(f)$ defined in Thm. \ref{lb17} when $f\in C(X)$. $f\in B(X)\mapsto f(T_1,\dots,T_N)\in\End(\mc H)$ is a unital $*$-homomorphism.
	
Moreover, if $f_\blt$ is a net of bounded Borel functions on $X$ whose $L^\infty(X)$-norms are uniformly bounded (over the directed set), such that $\lim \int_X|f-f_\blt|d\mu=0$ for each finite (positive) Borel measure on $X$, then $f_\blt(T_1,\dots,T_N)$ converges strongly to $f(T_1,\dots,T_N)$.
\end{thm}

\begin{proof}
We first define $f(T_1,\dots,T_N)$ using a fixed spectral decomposition as in Theorem \ref{lb8}. The $f\mapsto f(T_1,\dots,T_N)$ is clearly a unital $*$-homomorphism. For each $\xi\in\mc H$, write $U\xi=(g_n)_{n\in\fk N}\in\bigoplus_n L^2(X,\mu_n)$. Then
\begin{align*}
&\lVert (f(T_1,\dots,T_N)-f_\blt(T_1,\dots,T_N))\xi\lVert^2\\
=&	\lVert U(f(T_1,\dots,T_N)-f_\blt(T_1,\dots,T_N))U^*(g_n)_{n\in\fk N}\lVert^2\\
=&\sum_{n\in\fk N}\int_X |f-f_\blt|^2\cdot |g_n|^2d\mu_n\leq 2\lVert f\lVert_\infty \cdot \sum_{n\in\fk N}\int_X |f-f_\blt|\cdot |g_n|^2d\mu_n
\end{align*}	
which converges to $0$ by choosing $\mu=\sum_n |g_n|^2\mu_n$.

It is clear that $f(T_1,\dots,T_N)$ agrees with the $\pi(f)$ defined in \eqref{eq15} when $f$ is a polynomial. Since our map $f\in C(X)\mapsto f(T_1,\dots,T_N)$  defined here by spectral decomposition is clearly a continuous unital $*$-homomorphism,  it agrees with the one defined in Thm. \ref{lb17} due to the uniqueness statement of that theorem.  In the general case that $f$ is bounded Borel,  Lemma \ref{lb18} and the result in the last paragraph show that $f(T_1,\dots,T_N)$ is uniquely determined and independent of the choice of spectral decomposition.
\end{proof}


\subsection*{Another form of spectral theorem}





Spectral theorems are often presented in a form that is independent of the $L^2$ spaces $L^2(X,\mu_n)$. To begin with, we set
\begin{align}
E(\Omega)=\chi_\Omega(T_1,\dots,T_n)
\end{align}
where $\chi_\Omega$ \index{zz@$\chi_\Omega$} is the characteristic function of $\Omega$. Then $E(\Omega)$ is clearly a projection. $E$ is a projection-valued Borel measure, in the sense that $E$ associates to each $\xi,\eta\in\mc H$ the measure $\bk{E\xi|\eta}$ defined by $\bk{E\xi|\eta}(\Omega)=\bk{\chi_\Omega(T_1,\dots,T_N)\xi|\eta}$. It is easy to check that this is a complex Borel measure, which is (finite and) positive when $\xi=\eta$. Also, $E$ is determined by its evaluation$\bk{E\xi|\xi}$ for each $\xi\in\mc H$. We say $E$ is the \textbf{resolution of the identity} for $T_1,\dots,T_N$.

If $f$ is bounded Borel on $X$, we define $\int_XfdE$ to be the bounded linear operator on $\mc H$ satisfying
\begin{align*}
\int_X f\bk{dE\xi|\eta}.	
\end{align*}

\begin{thm}\label{lb49}
For each bounded Borel function $f$ on $X$, and for any $\xi,\eta\in\mc H$, we have
\begin{align}\label{eq4}
\bk{f(T_1,\dots,T_N)\xi|\eta}=\int_Xf\bk{dE\xi|\eta}.	
\end{align}
\end{thm}

Thus, we may write 
\begin{align}
f(T_1,\dots,T_N)=\int_X fdE.
\end{align}

\begin{proof}
By linearity, it suffices to assume $\eta=\xi$ so that $\bk{E\xi|\xi}$ is a positive Borel measure. Then, from the definition of $E$, it is clear that \eqref{eq4} holds when $f$ is a characteristic function.  Thus \eqref{eq4} holds when $f$ is a step function, hence (by monotone convergence theorem) when $f$ is a positive bounded Borel function, and hence when $f$ is a bounded complex Borel function. 
\end{proof}


\begin{df}
Let $T_1,\dots,T_N$ be adjointly commuting normal operators. Let $\Sp(T_1,\dots,T_N)$ be the set of all points of $\Cbb^N$ at which there is a neighborhood $W$ satisfying $E(W)\equiv \chi_W(T_1,\dots,T_N)\neq0$. \index{Sp@$\Sp(T_1,\dots,T_N)$} This is a closed subset of $\Cbb^N$, called the \textbf{joint spectrum} of $T_1,\dots,T_N$. In the setting of Theorem \ref{lb8}, one checks easily that $\Sp(T_1,\dots,T_N)$ is the closure of the union of the supports of all $\mu_n$ ($n\in\fk N$). In the case of a single normal operator $T$, the $\Sp(T)$ defined here agrees with the one defined by \eqref{eq23}.
\end{df}



\begin{exe}
In the case of a single normal operator $T$, use the relation between $\Sp(T)$ and the supports of $\mu_n$ to deduce  that  $T$ is self-adjoint (resp. positive (i.e. $\bk{T\xi|\xi}>0$ for every $\xi\in\mc H$), unitary), if and only if $\Sp(T)$ is a subset of $\Rbb$ (resp. $[0,+\infty)$, the unit circle).
\end{exe}

\begin{exe}
Use the relation between joint spectrum and the supports of $\mu_n$ and also the definition of $\Sp(T_j)$ in \eqref{eq23} to show that  $\Sp(T_1\times\cdots\times T_N)\subset\Sp(T_1)\times\cdots\times\Sp(T_N)$.
\end{exe}

\begin{rem}\label{lb12}
The above two exercises show that, once we have a bounded Borel function $f$ on $\Sp(T_1\times\cdots\times T_N)$, we can define $f(T_1,\dots,T_N)$ by extending $f$ to a Borel function on $\Cbb^N$ and define it as in Def. \ref{lb58}. Again, this definition is independent of spectral decompositions.
\end{rem}


For any Borel set $X$, we let $B(X)$ be the unital $*$-algebra of bounded Borel functions on $X$. The following proposition improves Prop. \ref{lb1}.
\begin{pp}\label{lb46}
Assume $\pi:B(X)\rightarrow\End(\mc H)$ is a unital $*$-homomorphism, i.e., preserves the identities, multiplications and linear combinations, and $*$-structures. Then $\lVert \pi(f)\Vert\leq \lVert f\lVert_{L^\infty(X)}$ for each $f\in B(X)$. 
\end{pp}

Thus,  the linear map $f\in B(\Sp(T_1,\dots,T_N))\rightarrow f(T_1,\dots,T_N)$ is bounded (with norm at most $1$) with respect to the $L^\infty$-norm and the operator norm.

\begin{proof}
The idea here is similar to (but slightly simpler than) Step. 3 of the proof of Thm. \ref{lb2}.	Let $r=\lVert f\lVert_\infty$. Then $r^2-f^*f$ is a positive function, for which we take the positive square root $g=\sqrt{r^2-f^*f}$. Then for each $\xi\in\mc H$, we compute
	\begin{align*}
		\bk{\pi(f)\xi|\pi(f)\xi}=\bk{\pi(f^*f)\xi|\xi} =\bk{\pi(r^2-g^2)\xi|\xi}=r^2\lVert\xi\lVert^2-\lVert\pi(g)\xi\lVert^2\leq r^2\lVert\xi\lVert^2.
	\end{align*}
\end{proof}

\subsection*{Appendix: Gelfand-Naimark theorem}

The following exercise outlines a proof of the celebrated Gelfand-Naimark Theorem using Prop. \ref{lb1}. This theorem will not be used in rest of this monograph, and hence can be safely skipped. 

\begin{exe}\label{lb64}
Let $\fk G$ be a (not necessarily finite) set of adjointly commuting self-adjoint bounded operators on $\mc H$. 
\begin{enumerate}
\item Let $\mc P$ be the set of polynomials with commuting  formal variables $\{t_T:T\in\fk G\}$. Namely, a generic element is a $\Cbb$-linear combination of $t_{T_1}^{n_1}\cdots t_{T_k}^{n_k}$ where $n_1,\dots,n_k\in\Nbb$ and $T_1,\dots,T_k\in\fk G$. The involution of $\mc P$ is defined by $(at_{T_1}^{n_1}\cdots t_{T_k}^{n_k})^*=\ovl at_{T_1}^{n_1}\cdots t_{T_k}^{n_k}$ ($a\in\Cbb$). Define a linear map $\pi:\mc P\rightarrow\End(\mc H)$ sending each $at_{T_1}^{n_1}\cdots t_{T_k}^{n_k}$ to $aT_1^{n_1}\cdots T_k^{n_k}$. Show that $\pi$ is a unital $*$-homomorphism. 

\item Let $Y_T=[-\lVert T\lVert,\lVert T\lVert]^2\subset\Cbb$. Let $Y=\prod_{T\in\fk G}Y_T$, which is a compact Hausdorff space by Tychonoff's theorem. Use Prop. \ref{lb1} to show that $\pi(f)\leq \lVert f\lVert_{L^\infty(Y)}:=\sup_{t_T\in Y_T,\forall T\in\fk G}|f((t_T)_{T\in\fk G})|$ for each $f\in\mc P$. Conclude that $\pi$ can be extended uniquely to a unital $*$-homomorphism $\pi:C(Y)\rightarrow\End(\mc H)$. 

\item Let $A$ be the smallest unital (norm-)closed $*$-subalgebra of $\End(\mc H)$ containing $\fk G$, called the \textbf{$C^*$-algebra} generated by $\fk G$. Use Stone-Weierstrass theorem to show that $A=\pi(C(Y))$. Thus we have a surjective unital $*$-homomorphism $\pi:C(Y)\rightarrow A$.


\item Let $\mc I=\Ker(\pi)$. Show that $\mc I$ is a closed $*$-ideal of $C(Y)$, which means that $\mc I$ is a closed subspace of $C(Y)$, and that for each $f\in\mc I,g\in C(Y)$ we have $fh\in\mc I,f^*\in\mc I$.

\item Let $X$ be the (necessarily closed) subset of all $x\in Y$ satisfying $f(x)=0$ for each $f\in\mc I$. Apply Stone-Weierstrass theorem to the family $\mc I$ of functions to show that $\mc I=\{f\in C(Y):f|_X=0\}$. 

\item By Tiezte extension theorem, the restriction map $f\in C(Y)\mapsto f|_X\in C(X)$ is surjective and has kernal $\mc I$. Conclude that we have a well-defined bijective unital $*$-homomorphism $\wtd\pi:C(X)\rightarrow A$ sending each $f|_X$ to $\pi(f)$ (where $f\in C(Y)$). Apply the proof of Prop. \ref{lb46} to $\wtd\pi$ and $\wtd\pi^{-1}$ to show that $\wtd\pi$ is isometric, i.e., $\lVert  \wtd\pi(f)\lVert=\lVert f\lVert_{L^\infty(X)}$ for each $f\in C(X)$.

\item Show that any commutative unital $*$-closed subalgebra $A$ of $\End(\mc H)$ is generated by a set $\fk G$ of self-adjoint commuting operators. Conclude that any $A$ is equivalent (as a normed unital $*$-algebra) to $C(X)$ for some compact Hausdorff space $X$.
\end{enumerate}
\end{exe}














\section{Unbounded operators}




An \textbf{unbounded operator} $T$ from $\mc H_1$ to $\mc H_2$ is, by definition, a linear map from a subspace $\scr D(T)$ of $\mc H_1$ (called the \textbf{domain of $T$}) to $\mc H_2$. Unless otherwise stated, unbounded operators are \textit{densely defined}, which means $\Dom(T)$ is a dense subspace of $\mc H_1$. In the case that $\mc H_1=\mc H_2=\mc H$, we say $T$ is an unbounded operator on $\mc H$. 

Unbounded operators mean non-necessarily bounded operators. Thus, bounded linear operators are also unbounded operators. A \textbf{continuous} unbounded operator is understood in the obvious way, i.e., the map $T:\scr D(T)\rightarrow\mc H_2$ is continuous with respect to the Hilbert-space norms. Thus, bounded operators are precisely continuous unbounded operators whose domains are the full Hilbert space.

\begin{rem}
The study of unbounded operators $T$ from $\mc H_1$ to $\mc H_2$ can be transformed to the study of $\wtd T$ on a single Hilbert space $\mc H$, if we set $\mc H=\mc H_1\oplus\mc H_2$, $\scr D(\wtd T)=\scr D(T)\oplus\mc H_2$, and $\wtd T(\xi\oplus\eta)=T\xi$ if $\xi\in\scr D(T)$ and $\eta\in\mc H_2$.
\end{rem}


For (non-necessarily densely defined) unbounded operators $A,B$ from $\mc H_1$ to $\mc H_2$, and $a,b\in\Cbb$, we define
\begin{gather*}
aA+bB:\scr D(A)\cap\scr D(B)\rightarrow\mc H_2,\qquad \xi\mapsto aA\xi+bB\xi,	
\end{gather*}
which is an unbounded operator with domain
\begin{align*}
\scr D(aA+bB)=\scr D(A)\cap\scr D(B).
\end{align*}


Note that by our definition, we have $A-A\subset 0$, with $\subset$ becomes $=$ if and only if $\Dom(A)=\mc H_1$.

We say
\begin{gather*}
	A\subset B
\end{gather*}
provided that
\begin{gather*}
\scr D(A)\subset\scr D(B),\\
A\xi=B\xi\qquad(\forall\xi\in\scr D(A)).
\end{gather*}
This notation is justified by the definition of the \textbf{graph} of $T:\scr D(T)\rightarrow T$, which is a subset of $\mc H_1\oplus\mc H_2$ defined by \index{GT@$\scr G(T),\fk G(T)$}
\begin{align*}
\scr G(T):=\{(\xi,T\xi):\xi\in\scr D(T)\}.	
\end{align*}
Then $A\subset B$ means precisely $\scr G(A)\subset\scr G(B)$.



If $A,B$ are both (densely/non-densely defined) unbounded operators on $\mc H$, we set
\begin{align*}
AB:\scr D(AB)\rightarrow\mc H,\qquad \xi\mapsto A\cdot B\xi	
\end{align*}
where
\begin{flalign*}
\scr D(AB)=B^{-1}\scr D(A)=\{\xi\in\scr D(B):B\xi\in\scr D(A)\}.	
\end{flalign*}
If this subspace is dense, then $AB$ is an unbounded operator on $\mc H$.


\begin{pp}\label{lb25}
Let $A,B,C$ be (non-necessarily densely-defined) unbounded operators on $\mc H$. Then
\begin{gather*}
(AB)C=A(BC)\\
(A+B)C=AC+BC\\
A(B+C)\supset AB+AC
\end{gather*}
Moreover, the $\supset$ in the last relation becomes $=$ if $A$ is everywhere defined, i.e., $\Dom(A)=\mc H$. (E.g., when $A$ is bounded.)
\end{pp}

As an example that the last $\supset$ is not $=$, take any $A$ whose $\Dom(A)$ is not the full Hilbert space $\mc H$, and take $B=\id,C=-\id$.

\begin{proof}
For each line, if $\xi$ belongs to the domains of both sides, then it is clear that the left and the right send $\xi$ to the same vector. Therefore, it is enough to verify the three relations on the level of domains.

One verifies that both sides on the first relation have domain
\begin{align*}
	\{\xi\in\mc H:\xi\in\Dom(C),C\xi\in\Dom(B),BC\xi\in\Dom(A)\},
\end{align*}
that both sides of the second relation have domain
\begin{align*}
C^{-1}(\Dom(A)\cap\Dom(B))=C^{-1}\Dom(A)\cap C^{-1}\Dom(B),
\end{align*}
and that the left and the right of the third relation have domains
\begin{gather*}
\{\xi\in\Dom(B)\cap\Dom(C):B\xi+C\xi\in\Dom(A)\},\\
\{\xi\in\Dom(B)\cap\Dom(C):B\xi\in\Dom(A),C\xi\in\Dom(A)\}.	
\end{gather*}
The relations are thus verified. When $\scr D(A)=\mc H$, the last two domains are both $\scr D(B)\cap\scr D(C)$.
\end{proof}

\begin{df}
Let $T$ be an unbounded operator from $\mc H_1$ to $\mc H_2$. We define
\begin{align*}
\Dom(T^*):=\{\eta\in\mc H_2:\text{There exists $\psi\in\Dom(\mc H_1)$ such that $\bk{T\xi|\eta}=\bk{\xi|\psi}$ for each $\xi\in\Dom(T)$}\}.	
\end{align*}
Such $\psi$ is unique and is denoted by $T^*\eta$. So for each $\xi\in\Dom(T),\eta\in\Dom(T^*)$, we have
\begin{align*}
	\bk{T\xi|\eta}=\bk{\xi|T^*\eta}.
\end{align*}
$T^*$ is a non-necessarily densely defined unbounded operator from $\mc H_2$ to $\mc H_1$ with domain $\Dom(T^*)$, called the \textbf{adjoint} of $T$.Note that by the Riesz representation theory for Hilbert spaces, we have
\begin{align}
\Dom(T^*):=\{\eta\in\mc H_2:\text{The linear functional $\xi\in\Dom(T)\mapsto \bk{T\xi|\eta}$ is bounded}\}.\label{eq21}
\end{align}
\end{df}



It is obvious that
\begin{align}
A\subset B\quad\Rightarrow \quad B^*\subset A^*.\label{eq22}
\end{align}

\begin{pp}\label{lb9}
Let $A,B$ be unbounded operators on $\mc H$. Then
\begin{gather*}
(A+B)^*\supset A^*+B^*\\
(AB)^*\supset B^*A^*
\end{gather*}
Moreover, if $A$ is bounded, then the $\supset$ in the two relations are both $=$.
\end{pp}


\begin{proof}
The first $\supset$ is easy to verify using \eqref{eq20}. For the second one, suppose $\eta\in\Dom(B^*A^*)$. Then $\eta\in\Dom(A^*)$, and $A^*\eta\in\Dom(B^*)$. The first property says $\bk{A\xi|\eta}=\bk{\xi|A^*\eta}$ for a vector $A^*\eta$ and every $\xi\in\Dom(A)$, in particular, every vector $B\psi$ where $\psi\in\Dom(AB)$. Thus $\bk{AB\psi|\eta}=\bk{B\psi|A^*\eta}$, which because of $A^*\eta\in\Dom(B^*)$ is equal to $\bk{\psi|B^*A^*\eta}$ for a vector $B^*A^*\eta$ and every $\psi\in\Dom(AB)$. This proves the second $\supset$.

Now assume $A$ is bounded. Choose any $\xi\in\Dom((A+B)^*)$. Then the function from $\eta\in\Dom(A+B)=\Dom(B)$ to $\bk{\xi|(A+B)\eta}$ is continuous. Since $\bk{\xi|A\eta}$ is clearly continuous over $\eta$, so is $\bk{\xi|B\eta}$. So $\xi\in\Dom(B^*)=\Dom(A^*+B^*)$. This proves the first equality. Now choose any $\eta\in\Dom((AB)^*)$. Note that $A^*\eta$ is defined. Then for each  $\xi\in\Dom(B)$, we have $\bk{B\xi|A^*\eta}=\bk{AB\xi|\eta}=\bk{\xi|(AB)^*\eta}$, showing that $A^*\eta\in\Dom(B^*)$ and hence $\eta\in\Dom(B^*A^*)$. The second equality is also proved
\end{proof}




\begin{df}
An unbounded operator $T$ from $\mc H_1$ to $\mc H_2$ is called \textbf{adjointable}  if $T^*$ has dense domain in $\mc H_2$. Assume $T$ is adjointable. It is clear that
\begin{align}
T\subset T^{**}.	
\end{align}
Thus $T^*$ is also adjointable (because its adjoint has dense domain, which contains a dense subspace $\Dom(T)$).
\end{df}

Thus, roughly speaking, an adjointable operator is one that we can take adjoint any times we want. But this does not mean that taking adjoints of $T$ will give us infinitely many different operators. Instead, we have only three different ones: $T,T^*,T^{**}$, as indicated by the following obvious property:




\begin{pp}\label{lb10}
Let $T$ be adjointable. Then $T^*=T^{***}$.
\end{pp}

\begin{proof}
Since $T\subset T^{**}$ in general, replace $T$ by $T^*$ and we get $T^*\subset T^{***}$. Take the adjoint of $T\subset T^{**}$ and notice \eqref{eq22}, we have $T^*\supset T^{***}$.
\end{proof}


Adjointability is an analytic condition, since it says roughly that many vectors  $\eta$ in $\mc H_2$ makes the linear functional $\xi\mapsto\bk{T\xi|\eta}$ continuous. 

\begin{exe}
Define an unbounded operator $T:\l^2(\Zbb_+)\rightarrow\Cbb$ whose domain $\Dom(T)$ is the set of all $(a_1,a_2,\dots)$ having finitely many non-zero elements. Define $T(a_1,a_2,\dots)=\sum_n a_n$. Show that $T$ is not adjointable. In general, show that any non-continuous linear map from an infinite dimensional Hilbert space to a finite dimensional one is not adjointable. 
\end{exe}


\begin{rem}
Let $T$ be a (densely defined) unbounded operator from $\mc H_1$ to $\mc H_2$. Let $E$ be the projection of $\mc H_2$ onto the closure of $\Dom(T^*)$. Then the restriction $ET$ from $\mc H_1$ to $\Rng(E)$ is adjointable.
\end{rem}


Continuous operators are certainly adjointable. Recall that if $A$ is bounded, then $\mathrm{Ker}(A)=\Rng(A^*)^\perp$, which shows that $A$ is injective (resp. has dense range) if and only if $A^*$ has dense domain (resp. injective). Using this fact, we can easily produce many unbounded adjointable operators.

\begin{eg}\label{lb13}
Let $A:\mc H_1\rightarrow\mc H_2$ be bounded, injective, and has dense range. By \eqref{eq19}, $A^*:\mc H_2\rightarrow\mc H_1$ is also bounded, injective, and has dense range. Let $\Dom(A^{-1})=\Rng(A)$, define $A^{-1}(A\xi)=\xi$ for each $\xi\in\mc H_1$. Then $A^{-1}$ is an adjointable unbounded operator from $\mc H_2$ to $\mc H_1$ with domain $\Dom(A^{-1})$, and
\begin{align}
(A^{-1})^*=(A^*)^{-1}.	\label{eq8}
\end{align}
\end{eg}

Note that $A^{-1}$ (and similarly $(A^*)^{-1}$) are surjective.

\begin{proof}
If $\xi\in \Dom((A^*)^{-1})$, then  for any $\eta\in\Dom(A^{-1})$, we have $\bk{A^{-1}\eta|\xi}=\bk{A^{-1}\eta|A^*(A^*)^{-1}\xi}=\bk{AA^{-1}\eta|(A^*)^{-1}\xi}=\bk{\eta|(A^*)^{-1}\xi}$, which shows $\xi\in\Dom((A^{-1})^*)$ and $(A^{-1})^*\xi=(A^*)^{-1}\xi$. Thus $(A^{-1})^*\supset(A^*)^{-1}$. In particular, since $(A^*)^{-1}$ has dense domain (which is the range of $A^*$), so does $(A^{-1})^*$. So $A^{-1}$ is adjointable.

Conversely, let $\xi\in \Dom((A^{-1})^*)$. Then $\bk{\xi|\gamma}=\bk{\xi|A^{-1}A\gamma}=\bk{(A^{-1})^*\xi|A\gamma)}=\bk{A^*(A^{-1})^*\xi|\gamma}$ for each $\gamma\in\mc H_1$ shows  $\xi=A^*(A^{-1})^*\xi$. Therefore $\xi$ is in the range of $A^*$, i.e., in the domain of $(A^*)^{-1}$. This implies $\Dom((A^{-1})^*)\subset\Dom((A^*)^{-1})$, which finishes the proof.
\end{proof}


\section{Spectral theorem for unbounded positive operators}

Recall that a \textit{bounded} operator $A$ on $\mc H$ is called \textbf{positive} if $\bk{A\xi|\xi}\geq 0$ for each $\xi\in\mc H$. Equivalently, $A$ is normal, and in the setting of Thm. \ref{lb8},  all $\mu_n$ have supports inside $[0,+\infty)$. It is also clear from Thm. \ref{lb8} that $1+A$ is invertible, and its inverse is also bounded and positive. Motivated by this observation, we consider:

\begin{pp}\label{lb11}
Let $T$ be an unbounded operator on $\mc H$. Assume $\bk{T\xi|\xi}\geq 0$ for each $\xi\in\Dom(T)$. Then  the following two equivalent conditions are satisfied:
\begin{itemize}
	\item The range of $1+T$ is $\mc H$.
	\item $1+T$ is the inverse of a bounded injective positve operator $A$ on $\mc H$. (Cf. Example \ref{lb13}. Note that $A$ has dense range since $A=A^*$.)
\end{itemize}
The second condition also implies $\lVert A\lVert\leq 1$. 
\end{pp}

We say that an unbounded operator $T$ satisfying $\bk{T\xi|\xi}\geq 0$ for each $\xi\in\Dom(T)$ and the above two equivalent conditions is \textbf{positive}.

\begin{proof}
We prove the equivalence of the two conditions. The second one clearly implies the first one by spectral decomposition of $A$. On the other hand, assume $1+T$ has range $\mc H$. Note that $1+T$ is injective, since if $(1+T)\xi=0$ then $0=\bk{(1+T)\xi|\xi}=\lVert\xi|\lVert^2+\bk{T\xi|\xi}\geq \lVert\xi\lVert^2$. Thus, we simply define $A$ to be the (everywhere defined) linear operator on $\mc H$ satisfying that $A(1+T)\xi=\xi$ for every $\xi\in\Dom(1+T)$. $A$ is clearly injective. For each $\xi\in\Dom(T)$, since $\bk{T\xi|\xi}=\bk{\xi|T\xi}\geq 0$, we have $\lVert (1+T)\xi\lVert^2\geq \Vert\xi\lVert^2=\lVert A(1+T)\xi\lVert^2$, which shows that $A$ is bounded and $\lVert A\lVert\leq 1$. Moreover, $\bk{A(1+T)\xi|(1+T)\xi}=\bk{\xi|(1+T)\xi}\geq 0$, showing that $A$ is positive.
\end{proof}



The condition that $1+T$ is the inverse of a bounded injective positive operator has many useful implications. As our first application, we define: 


\begin{df}
An  unbounded operator $T$ on $\mc H$ is called \textbf{symmetric} if $\bk{T\xi|\eta}=\bk{\xi|T\eta}$ for every $\xi,\eta\in\Dom(T)$. Equivalently, $T\subset T^*$. We say that $T$ is \textbf{self-adjoint} if $T=T^*$.
\end{df}

Then we have:
\begin{pp}\label{lb20}
Positive unbounded operators are self-adjoint.
\end{pp}

\begin{proof}
Let $T$ be positive, and let $A$ be the inverse of $1+T$, which is bounded, injective, and positive. So $A^*=A$. By Example \ref{lb13}, we have $1+T^*=(1+T)^*=(A^{-1})^*=(A^*)^{-1}=A^{-1}=1+T$.
\end{proof}



Thus, positive operators are adjointable.




\begin{thm}[Spectral theorem for a positive operator]\label{lb14}
Let $T$ be a positive unbounded operator on $\mc H$. Then there exists a set $(\mu_n)_{n\in\fk N}$ of finite (positive) Borel measures on $[0,+\infty)$ and a unitary
\begin{align*}
	U:\mc H\rightarrow\bigoplus_{n\in\fk N} L^2([0,+\infty),\mu_n)
\end{align*}
satisfying $UTU^*=x$ where $x$ is the identity function on $[0,+\infty)$, understood as the multiplication operator. More precisely, we have $\Dom(UTU^*)=\Dom(x)$, and for  each $(f_n)_{n\in\fk N}\in \bigoplus_{n\in\fk N} L^2([0,+\infty),\mu_n)$ in $\Dom(x)$,
\begin{align*}
	UTU^*\cdot (f_n)_{n\in\fk N}=(xf_n)_{n\in\fk N}.
\end{align*}
\end{thm}

Note that an element $(f_n)_{n\in\fk N}\in \bigoplus_{n\in\fk N} L^2([0,+\infty),\mu_n)$ belongs to $\Dom(x)$ iff $\sum_n\int |xf_n|^2d\mu_n<+\infty$.







Also, any unbounded operator described in such way is a positive operator, since the multiplication operator defined by $(1+x)^{-1}$ is bounded and positive.


\begin{proof}
Let $A$ be the bounded invertible positive operator whose inverse is $1+T$. Since $A$ is bounded and positive, we have $\Sp(A)\subset[0,+\infty)$. Since $\lVert A\lVert\leq 1$, we have $\Sp(A)\subset[0,1]$.  

For each Borel set $X$, we let $B(X)$ be the set of bounded Borel functions of $X$, which is a $*$-algebra. We can thus define a map $\Phi:B([0,+\infty))\rightarrow B([0,1])$ sending each $g$ to $\wtd g$ satisfying $\wtd g(0)=0$ and  $\wtd g(x)=g(x^{-1}-1)$ when $x\neq0$. Then $\Phi$ is a $*$-homomorphism (non-necessarily preserving $1$). For each $g\in B([0,+\infty))$, define
\begin{align*}
g(T)=\wtd g(A).	
\end{align*}
(Notice Remark \ref{lb12}). Since for each $\xi\in\mc H$ the linear functional $g\in C_c(\Cbb)\mapsto \bk{g(T)\xi|\xi}$ is clearly positive (because $\wtd g(A)$ is a positive operator when $g$ is positive and hence $\wtd g$ is positive), we may use Riesz-Markov representation theorem and the cyclic vector method in the proof of Theorem \ref{lb8} to show that there is a set $(\mu_n)_{n\in\fk N}$ of finite Borel measures  and a unitary $U:\mc H\rightarrow\bigoplus_n L^2([0,+\infty),\mu_n)$ such that for each $g\in C_c([0,+\infty))$ and each $(f_n)_{n\in\fk N}\in \bigoplus_n L^2([0,+\infty),\mu_n)$, we have $Ug(T)U^*\cdot (f_n)_{n\in\fk N}=(gf_n)_{n\in\fk N}$.

Let $g(x)=(x+1)^{-1}$. Then $\wtd g(x)=x$ and hence $g(T)=A$. Choose a sequence $g_k\in C_c([0,+\infty))$ uniformly bounded and converges pointwise to $g$.  Then $\wtd g_k$ is uniformly bounded and converges pointwise to $x$ (on $[0,1]$). Let $dE$ be the resolution of the identity for $A$.  So for each $\xi\in\mc H$, $\bk{\wtd g_k(A)\xi|\xi}=\int_0^1\wtd g_k\bk{dE\xi|\xi}$ converges to $\int_0^1\wtd g\bk{dE\xi|\xi}=\bk{\wtd g(A)\xi|\xi}=\bk{\xi|\xi}$ by dominated integral theorem. In other words, $g_k(T)$ converges weakly to $A$. Since $Ug_k(T)U^*$ is the multiplication of $g_n$ to each component of $\bigoplus_n L^2([0,+\infty),\mu_n)$, it also converges weakly to the multiplication of $g$. We conclude that $UAU^*$ equals the multiplication of $g(x)=(x+1)^{-1}$. So $UA^{-1}U^*=(UAU^*)^{-1}$ is the multiplication of $1+x$, and hence $UTU^*$ is the multiplication of $x$.
\end{proof}



The following exercise is important for future application.

\begin{exe}\label{lb15}
Assume $T$ is positive. This exercise shows that we can safely restrict $T$ to a closed subspace containing $\ovl{\Rng(T)}$, which is a positive operator and contains all the information of $T$. This is similar to restricting a function to a subset containing its support.
\begin{enumerate}
\item In Thm. \ref{lb14}, show that the closure of $\Rng(UTU^*)$ is the set of all $(f_n)_{n\in\fk N}$ such that $f_n(0)=0$ for each $n$.
\item Use the above result to show that 
\begin{align*}
\Dom(T)=\big(\Dom(T)\cap\ovl{\Rng(T)}\big)\oplus \Rng(T)^\perp,	
\end{align*}
that $T$ acts trivially on $\Rng(T)^\perp$, and that the restriction of  $T$ to $\ovl{\Rng(T)}$ is a positive operator with domain $\Dom(T)\cap\ovl{\Rng(T)}$ and dense range $\Rng(T)$.
\item Let $\mc H_0$ be a closed subspace of $\mc H$ containing $\Rng(T)$. Use the previous result to show that
\begin{align*}
\Dom(T)=\big(\Dom(T)\cap \mc H_0\big)\oplus\mc H_0^\perp,	
\end{align*}
that $T$ acts trivially on $\mc H_0^\perp$, and that $T|_{\mc H_0}$ is a positive operator with domain $\Dom(T|_{\mc H_0})=\Dom(T)\cap \mc H_0$ and range $\Rng(T|_{\mc H_0})=\Rng(T)$.

\textit{(Hint: Write $\mc H_0=\ovl{\Rng(T)}\oplus\mc H_1$. By the previous step, we have  $\Dom(T)=(\Dom(T)\cap\ovl{\Rng(T)})\oplus\mc H_1\oplus\mc H_0^\perp$ and $\Dom(T|_{\mc H_0})=(\Dom(T)\cap\ovl{\Rng(T)})\oplus\mc H_1$.)}
\item  Let $E$ be the projection of $\mc H$ onto a closed subspace $\mc H_0$ containing $\Rng(T)$. Use the above description to show $T=ET=TE$.
\end{enumerate}

\end{exe}




\begin{rem}\label{lb21}
Let $T$ be a positive operator on $\mc H$. Assume $U$ is a partial isometry from $\mc H$ to $\mc K$ with source space $\sgm(U)\supset\ovl{\Rng(T)}$ and target space $\tau(U)$. Then $S:=UTU^*$ is a positive operator on $\mc K$ whose range is $\Rng(S)=U\Rng(T)$. Its domain is $\Dom(S)=U\Dom(T)\oplus \tau(U)^\perp$. 

Indeed, as shown in the previous exercise, the action $T$ decomposes into two parts: on $\sgm(U)^\perp$ it acts trivially; on $\sgm(U)$ it restricts to a positive operator with range $\Rng(T)$ and domain $\Dom(T)\cap\sgm(U)$. Thus, $S$ decomposes into two parts: on $\tau(U)^\perp$ it acts trivially; on $\tau(U)$ it is unitarily equivalent to the action of $T$ on $\sgm(U)$ via the unitary map $U:\sgm(U)\xrightarrow{\simeq}\tau(U)$. So $S$ has domain $U(\Dom(T)\cap \sgm(U))\oplus \tau(U)^\perp$, which equals $U\Dom(T)\oplus \tau(U)^\perp$ since $U=UU^*U$ and $U^*U$ (which is the projection onto $\sgm(U)$) projects $\Dom(T)=(\Dom(T)\cap\sgm(U))\oplus \sgm(U)^\perp$ onto $\Dom(T)\cap\sgm(U)$.
\end{rem}







The following Lemma will be used later to obtain polar decomposition.

\begin{lm}\label{lb19}
Assume $T$ is a positive unbounded operator on $\mc H$. Then there is a unique positive operator on $\mc H$ satisfying $H^2=T$. We say $H$ is the \textbf{(positive) square root} of $T$ and write $H=T^{\frac 12}=\sqrt T$. 
\end{lm}

\begin{proof}
Existence: Apply the spectral theorem to $T$, we see that $T$ clearly has a positive square root.
	
Uniqueness: Suppose $H,K$ are positive, and $H^2=K^2=T$. Apply the spectral theorem \ref{lb14} to obtain a unitary $U:\mc H\rightarrow\bigoplus_n L^2([0,+\infty),\mu_n)$ such that $UHU^*$ is the multiplication of $x$. Then $U(1+H)^{-1}U^*=(1+x)^{-1}$ and $U(1+T)^{-1}U^*=(1+(UHU)^2)^{-1}=(1+x^2)^{-1}$.

We have $Uf((1+T)^{-1})U^*=f((1+x^2)^{-1})$ whenever $f$ is a polynomial, and hence, by Stone-Weierstrass theorem and Prop. \ref{lb46}, whenever $f\in C([0,1])$.  Note that $f((1+T)^{-1})$ is defined by approximation by polynomials and is hence independent of the spectral decomposition of $H$ or $T$ or $(1+T)^{-1}$. Set $f(x)=\frac{\sqrt x}{\sqrt x+\sqrt{1-x}}=\frac{1}{1+\sqrt{x^{-1}-1}}$. Then
\begin{align*}
Uf((1+T)^{-1})U^*=f((1+x^2)^{-1})=(1+x)^{-1}=U(1+H)^{-1}U^*
\end{align*}
Hence $(1+H)^{-1}=f((1+T)^{-1})$ and similarly $(1+K)^{-1}=f((1+T)^{-1})$. This proves $H=K$.
\end{proof}


\section{Preclosed and closed operators, polar decomposition}

As an application of the spectral theorem for positive operators, we study the problem of closures of unbounded operators. If $T$ is adjointable, then we may regard $T^{**}$ as the algebraic closure of $T$. One may wonder whether $T^{**}$ (the algebraic closure of $T$) can be approximated by $T$ in an appropriate sense. We shall answer this question in this section.

Recall the graph $\scr G(T)=\{(\xi,T\xi):\xi\in\Dom(T)\}$. Equivalently, we can consider $\fk G(T)$ \index{GT@$\scr G(T),\fk G(T)$} to be the same as $\Dom(T)$ as vector spaces, but equipped with a different inner product: for each $\xi,\eta\in\Dom(T)$, we set
\begin{align}
\bk{\xi|\eta}_{\fk G(T)}=\bk{\xi|\eta}+\bk{T\xi|T\eta}\label{eq11}
\end{align}
where $\bk {\cdot|\cdot}$ is the original inner product of $\mc H$. To avoid confusion, we write the vector in $\fk G(T)$ corresponding to $\xi\in\Dom(T)$ as $\Psi\xi$. Namely, we have a bijective linear map
\begin{gather}
\Psi:\Dom(T)\rightarrow\fk G(T),\nonumber\\
\bk{\Psi\xi|\Psi\eta}=\bk{\xi|\eta}+\bk{T\xi|T\eta}	\label{eq7}
\end{gather}
for each $\xi,\eta\in\Dom(T)$. Then $\fk G(T)$ is equivalent to $\scr G(T)$ as inner product spaces.



\begin{df}
Let $T$ be an unbounded operator from $\mc H_1$ to $\mc H_2$. We say $T$ is \textbf{closed} if the following clearly equivalent conditions are satisfied:
\begin{itemize}
\item $\scr G(T)$ is a closed subspace of $\mc H_1\oplus\mc H_2$.
\item $\fk G(T)$ is a complete metric space (i.e., a Hilbert space).
\item Suppose $\xi_n$ is a sequence in $\Dom(T)$ such that both $\xi_n$ and $T\xi_n$ converge. Let $\xi=\lim_{n\rightarrow\infty}\xi_n$ and $\eta=\lim_{n\rightarrow\infty}T\xi_n$. Then $\xi\in\Dom(T)$, and $T\xi=\eta$.
\end{itemize}
\end{df}

Note that the above statement about sequences can be replaced by that of nets.


A closed operator is not necessarily determined by its action on a dense subspace of $\Dom(T)$. The appropriate density notion for unbounded closed operators is that of cores:

\begin{df}
Suppose $T$ is an unbounded operator from $\mc H_1$ to $\mc H_2$. A  subspace $\Dom_0$ of $\Dom(T)$ is called a \textbf{core for $T$} if the following clearly equivalent conditions are satisfied
\begin{itemize}
		\item $\scr G(T|_{\Dom_0})$ is a dense subspace of $\scr G(T)$.
		\item $\Psi\Dom_0$ is a dense subspace of $\Psi\Dom(T)=\fk G(T)$.
		\item For each $\xi\in\Dom(T)$ there exists a sequence $\xi_n\in\Dom_0$, such that $\xi_n\rightarrow\xi$ and $T\xi_n\rightarrow T\xi$.
\end{itemize} 
In the case that $T$ is closed, $\Dom_0$ is a core for $T$ if and only if for each $\xi\in\Dom(T)$, there exists a sequence $\xi_n\in\Dom_0$, such that $\xi_n$ converges to $\xi$ and $T\xi_n$ is a Cauchy sequence.
\end{df}

Note that a core for $T$ is automatically a dense subspace of $\Dom(T)$ and of $\mc H_1$. $T$ is uniquely determined by its restriction to a core.










\begin{eg}\label{lb23}
If $T$ is adjointable, then $T^*$ is closed.
\end{eg}

\begin{proof}
Let $\eta_n\in\Dom(T^*)$ such that $\eta_n\rightarrow\eta\in\mc H_2$ and $T^*\eta_n\rightarrow\psi\in\mc H_1$. Choose any $\xi\in\Dom(T)$. Then
\begin{align*}
\bk{\eta|T\xi}=\lim_{n\rightarrow\infty}\bk{\eta_n|T\xi}=\lim_{n\rightarrow\infty}\bk{T^*\eta_n|\xi}=\bk{\psi|\xi},
\end{align*}
which shows $\eta\in\Dom(T^*)$ and $T^*\eta=\psi$.
\end{proof}

As a consequence, we see that for every adjointable $T$, its algebraic closure $T^{**}$ is closed. Also, any self-adjoint operator (and in particular, positive operator cf. Prop. \ref{lb20}) is closed.



We shall show the converse of the above example, namely, closed operators are adjoints of (adjointable) unbounded operators. If we define an unbounded operator to be algebraically closed provided that it is the adjoint of another one, then we will see that algebraically closedness and (previously defined) analytically closedness are equal. Then it follows easily that for every adjointable unbounded operator, its algebraic closure equals analytic closure. Moreover, the adjointability condition is equivalent to an analytic one.


We first need a crucial result; our treatment here follows \cite[Rem. 2.7.7]{Kad}.

\begin{lm}\label{lb16}
Let $T$ be an unbounded closed operator from $\mc H_1$ to $\mc H_2$. Then $T^*T$ is a (densely defined) unbounded positive operator on $\mc H_1$. Moreover, $\Dom(T^*T)$ is a core for $T$.
\end{lm}


\begin{proof}
Notice the bijective map $\Psi$ in \eqref{eq7}. Since $T$ is closed, $\fk G(T)$ is a Hilbert space. We regard $\Psi$ as an unbounded operator from $\mc H$ to  $\fk G(T)$ with dense domain $\Dom(\Psi)=\Dom(T)$.  We claim that
\begin{align}
\Psi^*\Psi=1+T^*T.	
\end{align}
Choose any $\xi\in\Dom(\Psi^*\Psi)\subset\Dom(\Psi)=\Dom(T)$, and choose any $\eta\in\Dom(\Psi)=\Dom(T)$, we use \eqref{eq7} to calculate
\begin{align*}
\bk{\Psi^*\Psi\xi|\eta}=\bk{\Psi\xi|\Psi\eta}=\bk{\xi|\eta}+\bk{T\xi|T\eta},
\end{align*}
which shows $T\xi\in\Dom(T^*)$ and $T^*T\xi=\Psi^*\Psi\xi-\xi$. So $\Psi^*\Psi\subset 1+T^*T$. Conversely, choose $\xi\in\Dom(T^*T)\subset\Dom(T)$ and $\eta\in\Dom(T)$, then 
\begin{align*}
\bk{\Psi\xi|\Psi\eta}=\bk{\xi|\eta}+\bk{T\xi|T\eta}=\bk{\xi|\eta}+\bk{T^*T\xi|\eta}	
\end{align*}
shows $\Psi\xi\in\Dom(\Psi^*)$ and $\Psi^*\Psi\xi=\xi+T^*T\xi$. So $\Psi^*\Psi\supset 1+T^*T$.

Since $\Psi$ is injective, and since its range is $\fk G(T)$, we can define its inverse $A=\Psi^{-1}$ to be a linear map from $\fk G(T)$ to $\mc H_1$ with dense image $\Dom(T)$. $A$ is clearly injective and bounded. Thus, Example \ref{lb13} applies and $\Psi=A^{-1}$.  Now, noting \eqref{eq8}, we have
\begin{align*}
&(1+T^*T)AA^*=\Psi^*\Psi AA^*=(A^{-1})^*A^{-1}AA^*\\
=&(A^*)^{-1}A^{-1}AA^*=	(A^*)^{-1}\id_{\fk G(T)} A^*=(A^*)^{-1}A^*=\id_{\mc H_1}.
\end{align*}
Since the domain of $(1+T^*T)AA^*$ is the set of all $\xi\in\mc H_1$ satisfying that $AA^*\xi\in\Dom(1+T^*T)=\Dom(T^*T)$, we see that $\Dom(T^*T)$ contains the range of $AA^*$. Since $A$ is bounded and has dense range $\Dom(T)$, and since $A^*$ also has dense range (since $A$ is injective), we see that $AA^*$ has dense range. So $T^*T$ has dense domain. Next, the above long calculation shows that $1+T^*T$ has range $\mc H_1$. It is clear that $\bk{(1+T^*T)\xi|\xi}\geq 0$ whenever $\xi\in\Dom(T^*T)$. Thus $T^*T$ is positive.

Finally, we show $\Dom(T^*T)=\Dom(\Psi^*\Psi)$ is a core for $T$ by showing that $\Psi\Dom(\Psi^*\Psi)$ is a dense subspace of $\fk G(T)=\Psi\Dom(\Psi)$. Since we have proved $\Psi^*\Psi AA^*=\id_{\mc H_1}$, we see that $\Rng(AA^*)\subset\Dom(\Psi^*\Psi)$. Since $\Psi AA^*=A^*$, we see that $\Psi\Rng(AA^*)$ equals $\Rng(A^*)$, which is a dense subspace of $\fk G(T)$ since $A$ is injective. We are done.
\end{proof}



\begin{thm}[Polar decomposition]\label{lb22}
Let $T$ be a closed operator from $\mc H_1$ to $\mc H_2$. 
\begin{enumerate}
\item There exist unique $U,H$ satisfying the following conditions:   $H$ is a positive operator on $\mc H_1$, $U$ is a partial isometry from $\mc H_1$ to $\mc H_2$ whose source space $\sgm(U)$ is the closure of $\Rng(H)$, and
\begin{align*}
	T=UH.
\end{align*} 
\item There exist unique $V,K$ satisfying the following conditions:   $K$ is a positive operator on $\mc H_2$, $V$ is a partial isometry from $\mc H_1$ to $\mc H_2$ whose target space $\tau(V)$ is the closure of $\Rng(K)$, and
\begin{align*}
	T=KV.
\end{align*} 
\end{enumerate}
Moreover, we have $U=V$, $H=(T^*T)^{\frac 12}$,  $K=(TT^*)^{\frac 12}$, $T$ is adjointable, and
\begin{align*}
T^*=HU^*=U^*K	
\end{align*}
are the right and the left  polar decompositions of $T^*$.
\end{thm}

We call $T=UH$ and $T=KV$ respectively the \textbf{left polar decomposition} and the \textbf{right polar decomposition} of $T$. $U=V$ is called the \textbf{phase} of $T$, and $H$ is called the \textbf{absolute value} of $T$. 


\begin{proof}
Existence: By Lemma \ref{lb16}, $T^*T$ is a positive operator,  which by spectral theorem admits a positive square root $H=\sqrt{T^*T}$. So $H^2=T^*T$. Note that $\Dom(H^2)=\Dom(T^*T)$ are inside the domains of $H$ and $T$. We define a linear map
\begin{align*}
U:H\Dom(T^*T)\rightarrow T\Dom(T^*T)	
\end{align*}
sending $H\xi\mapsto T\xi$ for each $\xi\in\Dom(T^*T)$. This map is well defined and preserves inner products since, for another $\eta\in\Dom(T^*T)$, we have
\begin{align*}
\bk{H\xi|H\eta}=\bk{H^2\xi|\eta}=\bk{T^*T\xi|\eta}=\bk{T\xi|T\eta}.	
\end{align*}
Since $\Dom(T^*T)=\Dom(H^2)$ is a core for $H$ and $T$ by Lemma \ref{lb16}, $H\Dom(T^*T)=H\Dom(H^2)$ is dense in $\Rng(H)$, and $T\Dom(T^*T)$ is dense in $T\Dom(T)=\Rng(T)$. Thus $U$ is extended uniquely to a unitary map from $\ovl{\Rng(H)}$ to $\ovl{\Rng(T)}$.	It is further extended to a partial isometry from $\mc H_1$ to $\mc H_2$ by acting trivially on $\Rng(H_1)^\perp$. So the source space $\sgm(U)$ and target space $\tau(U)$ are respectively $\ovl{\Rng(H)}$ and $\ovl{\Rng(T)}$.

From  the construction of $U$, we see that $T|_{\Dom(T^*T)}=UH|_{\Dom(TT^*)}$. Recall that $\Dom(T^*T)$ is a core for both $H$ and $T$. Thus, $\xi\in\Dom(T)$ means that there is a sequence $\xi_n\in\Dom(T^*T)$ converging to $\xi$ such that $T\xi_n$ is a Cauchy sequence, i.e., $\lVert T\xi_n-T\xi_m\lVert\rightarrow 0$ as $\min\{m,n\}\rightarrow \infty$. Similarly, $\xi\in\Dom(H)$ means that there is $\xi_n\in\Dom(T^*T)$ converging to $\xi$ such that $\lVert H\xi_n-H\xi_m\lVert$ (which equals $\lVert UH\xi_n-UH\xi_m\lVert$ since $\sgm(U)$ contains $\Rng(H)$) converges to $0$. So $T$ and $H$ have the same domain. If $\xi_n\in\Dom(T^*T)$ converges to $\xi\in\Dom(H)$ and $H\xi_n\rightarrow H\xi$, then $T\xi_n=UH\xi_n\rightarrow UH\xi$. Since $T$ is closed, we conclude $\xi\in\Dom(T)$ and $T\xi=UH\xi$. This proves $T\supset UH$ and hence $T=UH$.

Since $U^*U$ is the projection of $\mc H_1$ onto $\sgm(U)=\ovl{\Rng(H)}$, by Exercise \ref{lb15}, we have $H=HU^*U$ and hence $T=(UHU^*)U$. Let $K=UHU^*$. By Remark \ref{lb21}, $K$ is a positive operator and $\ovl{\Rng(K)}=U\ovl{\Rng(H)}=\tau(U)$. So $T=KU$ is a right decomposition for $T$.

We now prove the existence of polar decompositions for $T^*$. By Prop. \ref{lb9}, $T=UH$ shows $T^*=HU^*$. Since $\tau(U^*)=\sgm(U)=\ovl{\Rng(H)}$, $T^*=HU^*$ is a right polar decomposition for $T^*$. Since we define $K$ to be $UHU^*$, we have $T^*=U^*UHU^*=U^*K$, which is a left polar decomposition for $T$ since $\sgm(U^*)=\tau(U)=\ovl{\Rng(K)}$. Also, $\Dom(T^*)=\Dom(K)$ is a dense subspace of $\mc H_2$. So $T^*$ is adjointable.

 
	
Uniqueness: Suppose $T=UH=KV$ as described in the theorem. Then $T^*T=H^2$. Thus, by Lemma \ref{lb19}, $H$ is the unique positive square root $(T^*T)^{\frac 12}$ of $T^*T$. Similarly, $K=(TT^*)^{\frac 12}$ is uniquely determined by $T$.

It remains to show that $U$ and $V$ are uniquely determined. Since $\ovl{\Rng(H)}$ is the source space of $U$, $U$ acts trivially on $\Rng(H)^\perp$. The action of $U$ on $\Rng(H)$ is determined by $T$, since it sends $H\xi$ to $T\xi$ ($\xi\in\Dom(T)=\Dom(UH)=\Dom(H)$). So $U$ is unique. Finally, since $\ovl{\Rng(K)}$ is assumed to be $\tau(V)$, we have $T=VV^*KV$. So $T=V\cdot (V^*KV)$ is a left polar decomposition of $T$ since $V^*KV$ is positive and its range is dense in $\sgm(V)$ by Remark \ref{lb21} again. Thus, by the uniqueness of left polar decomposition which we have just proved, $V$ is uniquely determined.
\end{proof}


\begin{rem}\label{lb32}
In the above polar decompositions, it is clear from the proof that we have
\begin{align*}
(TT^*)^{\frac 12}=U(T^*T)^{\frac 12}U^*,\qquad (T^*T)^{\frac 12}=U^*(TT^*)^{\frac 12}U.	
\end{align*}
\end{rem}




\begin{rem}\label{lb45}
Suppse $H$ is a positive operator on $\mc H_1$, and $U:\mc H_1\rightarrow\mc H_2$ is a partial isometry with $\sgm(U)\supset \ovl{\Rng(H)}$. Then $UH$ is closed, since $H$ is closed, and  $U^*UH=H$ shows that $\fk G(H)$ and $\fk G(UH)$ are $\Dom(H)$ with the same inner product. Now assume $\sgm(U)=\ovl{\Rng(H)}$. Then  $U\cdot H$ is the left polar decomposition for $T:=UH$.
\end{rem}



\begin{thm}\label{lb24}
Let $T$ be an unbounded operator from $\mc H_1$ to $\mc H_2$.
\begin{enumerate}
\item The following three are equivalent.

(a) $T$ is closed.

(b) $T$ is adjointable and $T=T^{**}$.

(c) $T=S^*$ for some adjointable unbounded operator $S$ from $\mc H_2$ to $\mc H_1$.

\item $T$ is adjointable if and only if $T$ is \textbf{preclosed} (or \textbf{closable}), the latter means that $T$ is contained in a closed operator from $\mc H_1$ to $\mc H_2$.

\item Suppose $\Dom_0$ is a core for $T$. Then $(T|_{\Dom_0})^*=T^*$.


\end{enumerate}
\end{thm}

Thus, ``adjointable" and ``preclosed" are interchangeable. We will use ``preclosed" more often in the remaining parts of this note.


\begin{proof}
1. (a) $\Rightarrow$ (b): If $T$ is closed, then by Theorem \ref{lb22}, $T$ is adjointable, and we have polar decompisitions $T=KU$ and $T^*=UK$. By Prop. \ref{lb9}, $(T^*)^*=(U^*K)^*$ equals $KU^*=T$.

(b) $\Rightarrow$ (c):  Take $S=T^*$.

(c) $\Rightarrow$ (a): Example \ref{lb23}.

2. If $T$ is adjointable then $T$ is contained in the closed operator $T^{**}$. Conversely, if $T\subset T_1$ for a (densely defined) closed $T_1$, then $T^*\supset T_1^*$ and (by part 1) $T_1^*$ has dense domain. So $T^*$ has dense domain, which means $T$ is adjointable. 

3. $T|_{\Dom_0}\subset T$ implies $(T|_{\Dom_0})^*\supset T^*$. Suppose $\eta\in\Dom((T|_{\Dom_0})^*)$. For each $\xi\in\Dom(T)$, choose $\xi_n\in\Dom(T|_{\Dom_0})$ such that $T\xi_n\rightarrow T\xi$. Then
\begin{align*}
\bk{T\xi|\eta}=\lim_{n\rightarrow\infty}\bk{(T|_{\Dom_0})\xi_n|\eta}=\lim_{n\rightarrow\infty}\bk{\xi_n|(T|_{\Dom_0})^*\eta}=\bk{\xi|(T|_{\Dom_0})^*\eta},	
\end{align*}
which shows $\eta\in\Dom(T^*)$.
\end{proof}


We give a useful method for showing preclosedness:

\begin{pp}
An unbounded operator $T:\mc H_1\rightarrow\mc H_2$  is adjointable/preclosed if and only if the following is true: for any sequence $\xi_n\in\Dom(T)$ converging to $0$ such that $T\xi_n$ converges, we have $T\xi_n\rightarrow 0$.
\end{pp}


\begin{proof}
If $T$ is preclosed, let $\xi_n\in\Dom(T)$ converge to $0$ such that $T\xi_n=T^{**}\xi_n$ converges. Then as $T^{**}$ is closed, we have  $T^{**}\xi_n\rightarrow T^{**}\cdot 0=0$.

Conversely, suppose for each $\xi_n\in\Dom(T)$ converging to $0$ such that $T\xi_n$ converges we have $T\xi_n\rightarrow 0$. Then it is clear that for any  two sequences $\xi_n,\xi_n'\in\Dom(T)$ converging to the same vector $\xi\in\mc H$ such that both $T\xi_n$ and $T\xi_n'$ converge, then they converge to the same vector in $\mc H_2$, which we denote by $T_1\xi$. All such $\xi$ form a subspace $\Dom(T_1)$ of $\mc H_1$ which is dense since it contains $\Dom(T)$. We thus have an unbounded operator from $\mc H_1$ to $\mc H_2$ with domain $\Dom(T_1)$ sending each $\xi$ to $T_1$. It is clear that the graph $\scr G(T_1)$ is the closure of  $\scr G(T)$. So $T_1$ is a closed operator containing $T$. So $T$ is preclosed.  
\end{proof}



If $T$ is contained in  $T_1$, then any subset $\scr G_0$  between $\scr G(T)$ and $\scr G(T_1)$ is the graph of an operator $T_0$ (satisfying $\scr G(T)\subset\scr G(T_0)\subset \scr G(T_1)$). Indeed, we set $\Dom(T_0)$ to be the set of all $\xi$ where $(\xi,\eta)\in\scr G_0$ for some $\eta\in\mc H_2$. Then we necessarily have $\eta=T_1\xi$. Define $T_0$ sending each $\xi\in\Dom(T_0)$ to $T_1\xi$. Then $\scr G_0=\scr G(T_0)$. Thus we can define:

\begin{df}
Assume $T:\mc H_1\rightarrow\mc H_2$ is adjointable/preclosed. Then the (necessarily closed) operator $\ovl T:\mc H_1\rightarrow\mc H_2$ \index{T@$\ovl T=T^{**}$} whose graph $\scr G(\ovl T)$ is the closure of $\scr G(T)$ in $\mc H_1\oplus \mc H_1$ is called the \textbf{closure} of $T$.
\end{df} 

\begin{proof}
Let $\scr G_0$ be the closure of $\scr G(T)$. Since $T$ is preclosed, $T\subset T_1$ for a closed operator $T_1$. (E.g. $T_1=T^{**}$.) Then $\scr G(T_1)$ is closed and contain $\scr G(T)$. Therefore $\scr G_0$ is between $\scr G(T)$ and $\scr G(T_1)$. Thus, according to the previous discussion, $\scr G_0$ is the graph of a necessarily closed operator $\ovl T$.
\end{proof}


\begin{rem}
Note that if $\scr D_0$ is a core for a preclosed $T$, then, as the graph of $T|_{\scr D_0}$ is dense in that of $T$, they have the same closure. So $\ovl{T|_{\scr D_0}}=\ovl T$. 

Also, if $T$ is preclosed, then $\Dom(T)$ is a core for the closure $\ovl T$, since $\scr G(\ovl T)$ is the closure of $\scr G(\ovl T|_{\Dom(T)})=\scr G(T)$.
\end{rem}


\begin{thm}
Let $T:\mc H_1\rightarrow\mc H_2$ be adjointable/preclosed. Then
\begin{align*}
\ovl T=T^{**}.	
\end{align*}
\end{thm}

\begin{proof}
Since $\Dom(T)$ is a core for $\ovl T$, by Theorem \ref{lb24} we see that $T^*=(\ovl T)^*$, and hence $T^{**}=(\ovl T)^{**}$. Since $\ovl T$ is closed, by Theorem \ref{lb24}, $(\ovl T)^{**}=\ovl T$. We are done. 
\end{proof}











\section{Strong commutativity of (pre)closed operators, von Neumann algebras}\label{lb67}



Recall our notation that $\End(\mc H)$ is the $*$-algebra of bounded linear operators of $\mc H$. The $*$-structure is defined by the adjoint. We begin with the following easy observation.

\begin{rem}
Suppose $T$ is an unbounded operator from $\mc H_1$ to $\mc H_2$, and $A_1\in\End(\mc H_1),A_2\in\End(\mc H_2)$. Then the following are equivalent.
\begin{itemize}
\item $A_2T\subset TA_1$.
\item $A_1\Dom(T)\subset \Dom(T)$ and $A_2T\xi=TA_1\xi$ for each $\xi\in\Dom(T)$. 
\end{itemize}
\end{rem}

\begin{pp}\label{lb26}
Let $T:\mc H_1\rightarrow\mc H_2$ be preclosed and $A_1\in\End(\mc H_1),A_2\in\End(\mc H_2)$.
\begin{enumerate}
\item If $A_2T\subset TA_1$, then $A_1^*T^*\subset T^*A_2^*$ and $A_2\ovl T\subset \ovl T A_1$.
\item If $T$ is closed and $\mc H_1=\mc H_2=\mc H$, the set of all $A\in\End(\mc H)$ satisfying $AT\subset TA$ form a strongly closed unital subalgebra of $\End(\mc H)$
\end{enumerate}
\end{pp}



The second statement means that the set of all $A\in\End(\mc H)$ is closed under linear combination, multiplication, and approximation under strong operator topology (i.e., if a net $A_\blt\in \End(\mc H)$ satisfies $A_\blt T\subset TA_\blt$ and converges strongly to $A\in\End(\mc H)$, then $AT\subset TA$). Moreover, the set contains identity.

\begin{proof}
1. We have $(A_2T)^*\supset (TA_1)^*$. By Prop. \ref{lb9}, $A_1^*T^*\subset (TA_1)^*\subset (A_2T)^*=T^*A_2^*$. Take adjoint again. We have $A_2^{**}T^{**}\subset T^{**}A_1^{**}$, i.e., $A_2\ovl{T}\subset\ovl{T}A_1$.

2. By Prop. \ref{lb25}, the set of all $A\in\End(\mc H)$ satisfying $AT\subset TA$ is closed under addition and multiplication. Suppose $A_\blt$ is a net converging strongly to $A\in\End(\mc H)$ such that $A_\blt T\subset TA_\blt$. Choose any $\xi\in\Dom(T)$. Then $A_\blt\xi\in\Dom(T)$ and $A_\blt T\xi=TA_\blt\xi$. Since $A_\blt$ converges strongly, we have $A_\blt\xi\rightarrow A\xi$  and $TA_\blt\xi=A_\blt T\xi\rightarrow AT\xi$. Thus $(A\xi,AT\xi)$ is in the closure of the graph $\scr G(T)$. Thus, as $T$ is closed, we conclude $\xi\in\Dom(A)$ and $TA\xi=AT\xi$.
\end{proof}


\begin{df}\label{lb27}
Let $A\in\End(\mc H)$ and let $T$ be a closed operator on $\mc H$. We say $A$ and $T$ \textbf{commute strongly} if
\begin{align*}
AT\subset TA,\qquad A^*T\subset TA^*.	
\end{align*}
If $A$ and $T$ commute strongly, then so do $A$ and $T^*$, $A^*$ and $T$, $A^*$ and $T^*$ by Prop. \ref{lb26}.
\end{df}

\begin{rem}
Two \textit{bounded operators} $A,B$ commute strongly if and only if they \textbf{commute adjointly}, which means $AB=BA$ and $AB^*=B^*A$.
\end{rem}

In the case that $A$ is unitary, strong and ordinary commutativities are the same:

\begin{pp}\label{lb31}
A unitary operator $U\in\End(\mc H)$ commutes strongly with a closed operator $T$ on $\mc H$ if and only if $UT=TU$ (equivalently, $UTU^*=T$).
\end{pp}

\begin{proof}
If $UT=TU$, then $UTU^*=T$, so $TU^*=U^*T$. This shows $U$ commutes strongly with $T$.

Conversely, suppose $UT\subset TU$ and $U^*T\subset TU^*$. $UT\subset TU$ implies $UTU^*\subset TUU^*=T$. $U^*T\subset TU^*$ implies $U^*TU\subset TU^*U=T$. So  $T=UU^*TUU^*\subset UTU^*$. So $T=UTU^*$.
\end{proof}











From Prop. \ref{lb26}, we see that the set of all $A\in\End(\mc H)$ commuting strongly with a closed $T$ is a strongly closed unital $*$-subalgebra of $\End(\mc H)$, i.e., a strongly closed unital subalgebra which is closed under taking adjoints. In other words, such $A$ form a von Neumann algebra.

\begin{df}
A strongly closed unital $*$-subalgebra of $\End(\mc H)$ is called a \textbf{von Neumann algebra} on $\mc H$. If $\fk S$ is a set of closed operators, then the set $\fk S'$ of all $A\in\End(\mc H)$ commuting strongly with every operator of $\fk S$ is a von Neumann algebra on $\mc H$. We call $\fk S'$ \index{S@$\fk S',\fk S''$ (commutant and double commutant of $\fk S$)}  the \textbf{commutant} of $\fk S$. The double commutant $\fk S''=(\fk S')'$ is also called the \textbf{von Neumann algebra generated by $\fk S$}. If $\mc M$ is a von Neumann algebra on $\mc H$ and $T$ is a closed operator on $\mc H$, we say that $T$ is \textbf{affiliated with} $\mc M$ if $\{T\}''\subset\mc M$.
\end{df}

\begin{proof}
We have seen, from Prop. \ref{lb26}, that each $\{T\}'$ (where $T\in\fk S$)  is a von Neumann algebra. Then $\fk S'=\bigcap_{T\in\fk S}\{T\}'$ is clearly also a von Neumann algebra.
\end{proof}

\begin{rem}
It is obvious that if $\fk S\subset\fk T$ then $\fk S'\supset\fk T'$ and hence $\fk S''\subset\fk T''$. Also, similar to the reasoning in Prop. \ref{lb10}, we have $\fk S'=\fk S'''$.
\end{rem}



Strong commutativity of two unbounded closed operators $T_1,T_2$ cannot be defined in the same way as in \ref{lb27}. Indeed, our definition of strong commutativity will be equivalent to the following form: If we write $T_1=U_1H_1$ and $T_2=U_2H_2$ then we require each of $U_1$ and $H_1$ commutes strongly with each of $U_2,H_2$. However, we must show that this definition agrees with the one in Definition \ref{lb27}. This requires showing that if $A$ commutes strongly with $T$ (or more generally, if $A$ belongs to a von Neumann algebra), then so does its phase $U_A$ and absolute value $H=\sqrt{A^*A}$. We provide a proof below, which suggests the importance of studying von Neumann algebras. A different proof for the general case of unbounded polar decompositions is given in Theorem \ref{lb30}.


\begin{pp}\label{lb29}
Let $\mc M$ be a von Neumann algebra on $\mc H$.
\begin{enumerate}
\item Suppose $A_1,\dots,A_N\in\mc M$ are normal, and $f$ is a bounded Borel function on $\Cbb^N$. Then $f(A_1,\dots,A_N)\in\mc M$. 
\item Let $A\in\End(\mc H)$ with left (resp. right) polar decomposition $A=UH$ (resp. $A=KU$). Then $A$ belongs to $\mc M$ if and only if both $U$ and $H$ (resp. both $U$ and $K$)  belong to $\mc M$.
\item Any element in $\mc M$ is a linear combination of four unitary elements in $\mc M$.
\end{enumerate}
\end{pp}

\begin{proof}
1. Obvious when $f$ is a polynomial of $z_1,\ovl z_1,\dots,z_N,\ovl z_N$, and hence true when $f$ is continuous on  $\Sp(T_1,\dots,T_N)$ by Prop. \ref{lb46} and Stone-Weierstrass theorem. The general case follows from Lemma \ref{lb18} and Thm. \ref{lb28}.

2. It is clear that $U,H\in\mc M$ (resp. $K,U\in\mc M$)  implies $A\in\mc M$. Conversely, we assume $A\in\mc M$. Then $A^*\in\mc M$ since $\mc M$ is a $*$-algebra. So $A^*A,AA^*\in\mc M$ and hence, by part 1, $H=\sqrt{A^*A},K=\sqrt{AA^*}$ belong to $\mc M$.

We show that $U\in \mc M$. For each $r>0$, define $f_r\in L^\infty([0,+\infty))$ to be $f(x)=x^{-1}\chi_{(r,+\infty)}$. Then by part 1, we have $f(T)\in\mc M$ and hence $U\chi_{(r,+\infty)}(T)=UTf_r(T)=Af_r(T)\in\mc M$. As $r\rightarrow 0$, $\chi_{(r,+\infty)}$ converges to $\chi_{(0,+\infty)}$ pointwise. So by Thm. \ref{lb28}, we see that $\chi_{(r,+\infty)}(T)$ converges strongly to $\chi_{(0,+\infty)}(T)$, which shows $U\chi_{(0,+\infty)}(H)\in\mc M$. Recall our assumption in polar decomposition that the source space $\sgm(U)$ equals $\ovl{\Rng(H)}$. By Example \ref{lb15}, the projection onto $\ovl{\Rng(H)}$ is $\chi_{(0,+\infty)}(H)$. So $U=U\chi_{(0,+\infty)}(H)$ belongs to $\mc M$.

3. Any $A\in\mc M$ is a linear combination of two self-adjoint elments in $\mc M$, namely $A+A^*$ and $i(A-A^*)$. Any self-adjoint $A\in\mc M$ satisfying $\lVert A\lVert \leq1$ (and hence $\Sp(A)\subset[-1,1]$) is a sum of two unitary elements in $\mc M$: $A=f_+(A)+f_-(A)$, where $f_\pm(x)=x\pm\im\sqrt{1-x^2}$.
\end{proof}


It follows immediately that a bounded operator $A$ commutes strongly with a closed $T$ if and only if the phase of $A$ and $\sqrt{A^*A}$ (or $\sqrt{AA^*}$) commutes strongly with $T$.


\begin{rem}
The above proof indicates why, for the problem of (strong) commutativity, it is not enough to consider $C^*$-algebras, namely, norm-closed $*$-subalgebras of $\End(\mc H)$: The phase $U$ of a bounded operator $A$ cannot in general be approximated in the norm topology by linear combinations of multiplications and powers of $A,A^*$.
\end{rem}


The close relation between strong/weak operator topology and strong/adjoint commutativity is also indicated by the following celebrated theorem of von Neumann. This result has some similarities with Theorem \ref{lb24} for closed operators.

\begin{thm}[von Neumann bicommutant theorem]\label{lb59}
Let $\mc M$ be a unital $*$-subalgebra of $\End(\mc H)$. Then the following  are equivalent.
\begin{enumerate}[label=(\alph*)]
\item $\mc M$ is closed under strong operator topology (i.e., is a von Neumann algebra).
\item $\mc M$ is closed under weak operator topology. 
\item $\mc M=\mc M''$.
\item $\mc M=\fk S'$ where $\fk S$ is a set of closed operators on $\mc H$.
\end{enumerate}	



\end{thm}	

\begin{proof}
	We have (c) $\Rightarrow$ (b)  and (b) $\Rightarrow$ (a). The first arrow is due to a routine check that the commutant of any set of \textit{bounded} operators is weakly closed, the second one is obvious.
	
	We also have (c) $\Rightarrow$ (d)  and (d) $\Rightarrow$ (a). The first arrow is obvious, and the second one is by Prop. \ref{lb26}. So it remains to show (a) $\Rightarrow$ (c).
	
	
	
	We assume (a), and  show that for each $\xi_1,\dots,\xi_N\in\mc H$, $A\in\mc M''$, and $\epsilon>0$, there exists $B\in\mc M$ such that $\lVert A\xi_j-B\xi_j\lVert<\epsilon$ for each $1\leq j\leq N$.
	
	We first consider the case $N=1$ and $\xi_1=\xi$. Let $e$ be the projection of $\mc H$ onto the closure of $\mc M\xi=\{x\xi:x\in\mc M\}$. For each $x\in\mc M$, since $x$ leaves $\mc M\xi$ (and hence its closure) invariant, we see $xe=exe$. Similarly, we have $x^*e=ex^*e$, whose adjoint gives $ex=exe$. So $xe=ex$. Similarly $x^*e=ex^*$. This proves $e\in\mc M'$. Choose any $A\in\mc M''$. Then $A$ commutes with $e$, which shows $Ae=Ae^2=eAe$, i.e $A$ leaves $\ovl{\mc M\xi}$ invariant. In particular, $A\xi$ belongs to $\ovl{\mc M\xi}$, which thus could be approximated by some $B\xi$ where $B\in\mc M$.
	
	Now, we consider the general case of $N$ vectors. Let $\mc M$ act on $\oplus_N\mc H=\mc H\oplus\cdots\oplus\mc H\simeq\mc H\otimes\Cbb^N$ diagonally by $\pi(x)(\xi_1,\dots,\xi_N)=(x\xi_1,\dots,x\xi_N)$. $\pi(\mc M)$ (the set of all $\pi(x)$) is a unital $*$-subalgebra of $\End(\mc H)$. By easy matrix calculation, one verifies that its commutant $\pi(\mc M)'$ equals $\mc M'\otimes\End(\Cbb^N)$, i.e., the set of $N\times N$ matrices whose elements are in $\mc M'$. Its commutant is then $\pi(\mc M'')$, the set of all $y\in\mc M''$ acting diagonally on $\oplus_N\mc H$. Thus, by the result of the previous paragraph, for each $A\in\mc M''$ and $\epsilon>0$ one can find $B\in\mc M$ such that $\lVert (\pi(A)-\pi(B))\xi\lVert<\epsilon$ where we set $\xi=(\xi_1,\dots,\xi_N)$. This shows $\lVert (A-B)\xi_j\lVert<\epsilon$ for each $j$.
\end{proof}

\begin{rem}
In the above theorem, it can be shown that $\mc M=\mc M''$ if and only if $\mc M$ is closed under \textbf{$*$-strong operator topology}, whose open sets are unions of $\{T\in\End(\mc H):\lVert T\xi_j-T_0\xi_j\lVert<\epsilon, \lVert T^*\xi_j-T_0^*\xi_j\lVert<\epsilon,1\leq j\leq N\}$ (where $T_0\in\End(\mc H),N\in\Nbb,\xi_1,\dots,\xi_N\in\mc H,\epsilon>0$). Thus, $\mc M$ is a von Neumann algebra iff for every net $T_\blt\in\mc M$ such that $T_\blt$ and $T_\blt^*$ converge strongly to $T,T^*$ respectively, we have $T\in\mc M$.
\end{rem}

\begin{proof}
The only if part is obvious. For the if part, assume $\mc M$ is $*$-strongly closed, and choose $A\in\mc M''$. As argued in the proof of Thm. \ref{lb59}, for each $\xi\in\mc H$, $A\xi$ is in the closure of $\mc M\xi$. Then $(A\xi,A^*\xi)\in\mc H\oplus\mc H$ is in the weak closure of $C:=\{(B\xi,B^*\xi):B\in\mc M\}$. Since $C$ is convex, by Hahn-Banach separation theorem, $(A\xi,A^*\xi)$ is in the (strong) closure of $C$, which shows that $A\xi$ can be approximated $*$-strongly by $B\xi$ (where $B\in\mc M$). Then, using the direct sum trick as in the last paragraph of the proof of Thm. \ref{lb59}, we see that $A$ can be approximated $*$-strongly by elements of $\mc M$.
\end{proof}









The following theorem indicates how a set of bounded operators can approximate its double commutant.

\begin{thm}\label{lb60}
Let $\fk G$ be a set of bounded operators on $\mc H$. Let $\mc A$ be the smallest unital $*$-subalgebra of $\End(\mc H)$ containing $\fk G$. Let $\ovl{\mc A}$ be either the $*$-strong or the strong or the weak operator closure of $\mc A$ in $\End(\mc H)$. The following are true.
\begin{enumerate}
\item $\fk G'=\mc A'=\ovl{\mc A}'$.
\item $\ovl{\mc A}=\fk G''$.
\end{enumerate} 
\end{thm}
Note that $\mc A$ is the set of linear combinations of multiplications of elements of $\{1\}\cup\fk G\cup\{x^*:x\in\fk G\}$.

\begin{proof}
That $\fk G\subset\mc A\subset\ovl{\mc A}$ shows $\fk G'\supset\mc A'\supset\ovl{\mc A}'$. It is direct to check that any element commutes strongly/adjointly with $\fk G$ commutes strongly with those in $\mc A$ and moreover those in $\ovl{\mc A}$. (One may also use Prop. \ref{lb26}.)	This proves part 1. Part two follows from the bicommutant theorem.
\end{proof}


As a consequence, we see that any self-adjoint bounded operator $H\in\fk G''$ can be approximated strongly by self-adjoint operators of $\mc A$. (Indeed, we can find a net $A_\blt\in\mc A$ converging $*$-strongly to $H$. So $(A_\blt+A^*_\blt)/2$ converges strongly to $H$.)


Theorem \ref{lb60} provides us with a useful method of showing that a bounded operator $A$ can be approximated strongly or weakly by elements of $\mc A$: instead of explicitly constructing the approximation, one checks that $A$ commutes adjointly with any element in $\fk G'$. Moreover, it suffices to check that $A$ commutes strongly with a small collection $\fk F$ of bounded or closed operators  which generates $\fk G'$, i.e., which satisfies $\fk F''=\fk G'$. Then $A\in \fk F'=\fk F'''=\fk G''=\ovl{\mc A}$. This method has a Hilbert space analog: to show that a subspace $W$ of a Hilbert space $\mc H$ is dense, one shows that any vector in $\mc H$ whose inner product with all elements of $W$ vanishes is $0$. Such algebraic verification is often easier than explicitly constructing approximations.


In the case that we have a set of unbounded closed operators $\fk G$, the study of $\fk G''$ can be reduced to the bounded case via polar decompositions.

\begin{thm}\label{lb30}
Let $\fk G$ be a set of closed operators on $\mc H$. For each $T$, we let $U_T$ be its phase, and let $H_T$ be one of $\sqrt{T^*T}$ and $\sqrt{TT^*}$. Then $\fk G''=\{U_T,(1+H_T)^{-1}:T\in\fk G\}''$.
\end{thm}

\begin{proof}
By Prop. \ref{lb29}, it suffices to show that any unitary operator $V$ commutes strongly with every $T$ iff it commutes strongly with each $U_T$ and $(1+H_T)^{-1}$. Recall Prop. \ref{lb31}. We treat the case  $H_T=\sqrt{T^*T}$ as the other case is similar. $VTV^*=(VU_TV^*)(VH_TV^*)$ is the left polar decomposition of $VTV^*$: indeed, $VH_TV^*$ is clearly positive; we have $\ovl{\Rng(H_T)}=\sgm(U_T)$, so $\ovl{\Rng(VH_TV^*)}=V\ovl{H_T}=V\sgm(U_T)=\sgm(VUV^*)$. By the uniqueness of polar decomposition in Thm. \ref{lb22}, we see that $VTV^*=T$ if and only if $VU_TV^*=U$ and $VH_TV^*=H_T$. Note that $(1+VH_TV^*)^{-1}=V(1+H_T)^{-1}V^*$ since $V$ is unitary. So $VH_TV^*=H_T$ iff $V(1+H_T)^{-1}V^*=(1+H_T)^{-1}$. This finishes the proof.
\end{proof}

It follows immediately that a closed operator $T$ is affiliated with a von Neumann algebra $\mc M$ (i.e., $\{T\}''\subset\mc M$) if and only if its phase and one of $(1+\sqrt{T^*T})^{-1}$ and $(1+\sqrt{TT^*})^{-1}$ are in $\mc M$. 


 


\begin{df}
Let $T,S$ be preclosed operators on $\mc H$. We say $T$ and $S$ \textbf{commute strongly} if $\{\ovl T\}''$ commutes (adjointly) with $\{\ovl S\}''$, i.e. $\{\ovl T\}''\subset\{\ovl S\}'$. In the case that $S,T$ are closed and at least one of them is bounded, this definition agrees with that in Def. \ref{lb27}.
\end{df}

\begin{proof}
We check that the current definition agrees with the previous one when $S$ is bounded and $T$ is closed. $S$ commutes strongly with $T$ iff $\{S\}\subset\{T\}'$ iff $\{S\}''\subset\{T\}'''$ iff $\{S\}''$ commutes (adjointly) with with $\{T\}''$.
\end{proof}


\begin{co}\label{lb34}
Let $S,T$ be closed operators on $\mc H$ with phases $U_S,U_T$ respectively. Let $H_S$ (resp. $H_T$) be one of $\sqrt{S^*S},\sqrt{SS^*}$ (resp. $\sqrt{T^*T},\sqrt{TT^*}$). Then the following are equivalent.
\begin{enumerate}
\item $S$ and $T$ commute strongly.
\item $U_S$ and $(1+H_S)^{-1}$ commute adjointly with $U_T$ and $(1+H_T)^{-1}$.
\item $U_S$ and $(1+H_S)^{-1}$ commute strongly with $T$.
\end{enumerate}
\end{co}

\begin{proof}
The equivalent of 1 $\Leftrightarrow$ 3 and 2 $\Leftrightarrow$ 3 are immediate from the above definition and Theorem \ref{lb30}.
\end{proof}


\begin{co}\label{lb61}
Let $\fk S,\fk T$ be two sets of closed operators on $\mc H$. Then $\fk S''$ commutes (adjointly) with $\fk T''$ (i.e. $\fk S''\subset\fk T'''=\fk T'$) if and only if every $S\in\fk S$ and $T\in\fk T$ commute strongly.
\end{co}

\begin{proof}
Take left polar decompositions $S=U_SH_S,T=U_TH_T$. Then each $S$ and $T$ commute strongly iff each $U_S,(1+H_S)^{-1}$ commute strongly with $U_T,(1+H_T)^{-1}$, iff $\mbf S$ and $\mbf T$ commute strongly where $\mbf S=\{U_S,(1+H_S)^{-1}:S\in\fk S\}$ and $\mbf T=\{U_T,(1+H_T)^{-1}:T\in\fk T\}$, iff $\mbf S\subset\mbf T'$, iff $\mbf S''\subset\mbf T'''=\mbf T'$. By theorem \ref{lb30}, we have $\mbf S''=\fk S''$ and $\mbf T'=\fk T'$.
\end{proof}









\section{Spectral theorem for strongly commuting normal closed operators}\label{lb66}


\begin{df}\label{lb52}
A closed operator $T$ on $\mc H$ with phase $U$ satisfying the following equivalent conditions is called \textbf{normal}:
\begin{enumerate}
\item $T^*T=TT^*$.
\item $U$ is normal and commutes strongly with $\sqrt{T^*T}$. 
\item $U$ is normal and commutes strongly with $\sqrt{TT^*}$. 
\item $T$ commutes strongly with $T$.
\end{enumerate}
\end{df}



\begin{proof}
By Cor. \ref{lb34}, 4 is equivalent to both 2 and 3. Thus, it suffices to prove the equivalence of 1 and 2. 

Let $H:=\sqrt {T^*T}$ and $K:=\sqrt{TT^*}$. Part 1 is equivalent to $H=K$. Note that by Remark \ref{lb32}, we have $K=UHU^*$ and $H=U^*KU$. Suppose $H=K$. Then $\sgm(U)=\ovl{\Rng(H)}$ equals $\tau(U)=\ovl{\Rng(K)}$, so $U^*U=UU^*$, which shows $U$ is normal. We have $H=UHU^*$ and $H=U^*HU$. The first relation shows $U^*H=U^*UHU^*=HU^*$ since $U^*U$ projects onto $\sgm(U)=\ovl{\Rng(H)}$. (Recall Exercise \ref{lb15}.) Likewise, the second equation shows $UH=UU^*HU=HU$ where $\tau(U)=\sgm(U)=\ovl{\Rng H}$ is used. So $U$ commutes strongly with $H$. 

Conversely, suppose $U$ is normal and commutes strongly with $H$. $UH\subset HU$ implies $UHU^*\subset HUU^*=HU^*U=H$ where the last equality is due to $\sgm(U)=\ovl{\Rng(H)}$. Similarly, $U^*H\subset HU^*$ implies $U^*HU\subset HU^*U=H$ and hence $H=UU^*HUU^*\subset UHU^*$. So $H=UHU^*=K$.
\end{proof}


\begin{eg}\label{lb33}
Let $X$ be a measure space  with a set of positive measures $\{\mu_n:n\in\fk N\}$. (One may assume the measures are finite, which is sufficient for application.) Let $f:X\rightarrow \Cbb$ be a (non-necessarily bounded) complex Borel function on $X$, acting on $\mc H=\bigoplus_{n\in\fk N} L^2(X,\mu_n)$ by multiplication on each component, i.e., $f(g_n)_{n\in\fk N}=(fg_n)_{\fk N}$. Then $f$ is closed. Indeed, let $u:X\rightarrow\Cbb$ be equal to $u(x)=f(x)/|f(x)|$ when $f(x)\neq 0$, and $u(x)=0$ when $f(x)=0$. Then $|f|$ is positive (because $1+|f|$ is the inverse of the clearly bounded injective positive operator $(1+|f|)^{-1}$), and it is not hard to check that $\ovl{\Rng(|f|)}$ is the set of all $(g_n)_{n\in\fk N}$ where each $g_n$ is zero at $\{x:f(x)=0\}$. Thus $\ovl{\Rng(|f|)}$ equals both the source space and the target space of $u$.
	
From this, we see that $f=u\centerdot |f|$ (note that $f$ and $|f|$ have the same domain as multiplication operators) is closed and $u\centerdot|f|,|f|\centerdot u$ are the left and right polar decompositions of $f$, understood as compositions of the multiplication operators of $|f|,u$. So $f$ is normal. By Thm. \ref{lb22}, the adjoint of the multiplication of $f$ is the composition $u^*\centerdot|f|$, which clearly equals the multiplication of the function $u^*|f|=f^*$ (since their domains are the same). So the multiplication of the adjoint of $f$ is the adjoint of the multiplication of $f$.
\end{eg}


\begin{cv}
To avoid confusions, when necessary, we write $f\centerdot g$ to denote the product of the multiplication operators of $f$ and $g$, as distinguished from $fg=f\cdot g$.\index{fg@$f\centerdot g$}
\end{cv}

Spectral theorem says that any  finitely many strongly commuting normal operators are simultaneously unitarily equivalent to some complex Borel functions acting as multiplication on the Hilbert space given in Example \ref{lb33}.


The meaning that the closed operators $T_1,\dots,T_N$ on a Hilbert space $\mc H$ are normal and commute strongly (with each other) is clear: $T_i$ commutes strongly with $T_j$ for each $1\leq i,j\leq N$.

\begin{lm}\label{lb36}
Let $T_1,\dots,T_N$ be strongly commuting normal closed operators on $\mc H$ with left polar decompositions $T_j=V_jH_j$. Let $R_j=(1+H_j)^{-1}$. Let $E$ be the resolution of the identity for $V_1,R_1,\dots,V_N,R_N$, considered as an operator valued measure on $(v_1,r_1,\dots,v_N,r_N)\in\Cbb^{2N}$. Let $\mbb S^1=\{z\in\Cbb:|z|=1\}$. \index{S1@$\mathbb S^1$} Then $E$ is zero (more precisely, $\bk{E\xi|\xi}$ is zero for every $\xi\in\mc H$) on any Borel subset of $\Cbb^{2N}$ which is outside
\begin{align*}
X=(\mbb S^1\cup\{0\})\times(0,1]\times\cdots\times(\mbb S^1\cup\{0\})\times(0,1].
\end{align*}
Moreover, for each $1\leq j\leq N$, if we let $\Omega_j$ (resp. $\Gamma_j$) be the subset of $X$ of all $(v_1,r_1,\dots,v_N,r_N)$ satisfying $v_j=0$ (resp. $r_j=1$), then $E$ is zero on $\Omega_j\Delta\Gamma_j:=(\Gamma_j-\Omega_j)\cup(\Omega_j-\Gamma_j)$.
\end{lm}

Namely, the conditions $v_j=0$ and $r_j=1$ are equivalent almost everywhere with respect to $E$. For the spectral decomposition of  $V_1,R_1,\dots,V_N,R_N$, we may restrict the integrals to
\begin{align}
Y:=X-\bigcup_{j=1}^N \Omega_j\Delta\Gamma_j.\label{eq13}
\end{align}

\begin{proof}
$\mc H$ is unitarily equivalent to $\bigoplus_n L^2(\Cbb^{2N},\mu_n)$ such that $V_1,R_1,\dots,V_N,R_N$ are unitarily equivalent to the multiplications of the coordinate functions $v_1,r_1,\dots,v_N,r_N$ respectively. That $V_jV_j^*=V_j^*V_j$ are projections mean $|v_j|^2$ takes values in $\{0,1\}$. So $v_j$ is $0$ a.e. (with respect to any $\mu_n$) when outside $\mbb S^1\cup\{0\}$. Since $r_j$ is positive (as a multiplication operator), it is zero a.e. outside $[0,+\infty)$. Since $r_j$ is injective, it is clear that the set $\{(v_1,r_1,\dots,v_N,r_N):r_j=0\}$ has $\mu_n$-measure $0$. So we can restrict to $r_j>0$. $1+H_j$ is represented by $r_j^{-1}$, so $H_j$ is represented by $r_j^{-1}-1$, which must be positive a.e.. So $r_j$ is zero a.e. outside $(0,1]$.

By the requirement on polar decompositions, we have $\ovl{\Rng(V_j)}=\ovl{\Rng(H_j)}$, or equivalently, $\ovl{\Rng(v_j)}=\ovl{\Rng(r_j^{-1}-1)}$. As in Example \ref{lb33}, it is easy to check that $\ovl{\Rng(v_j)}$ is the set of all $(f_n)_{n\in\fk N}$ vanishing on $\{v_j=0\}=\Omega_j$, and similarly  $\ovl{\Rng(r_j^{-1}-1)}$ is the set of all $(f_n)_{n\in\fk N}$ vanishing on $\{r_j^{-1}-1=0\}=\{r_j=1\}=\Gamma_j$. This finishes the proof.
\end{proof}





\begin{thm}[Spectral theorem]\label{lb35}
Let $T_1,\dots,T_N$ be strongly commuting normal closed operators on $\mc H$. Then there exist a set $(\mu_n)_{n\in\fk N}$ of finite (positive) Borel measures on $\Cbb^N$, and also a unitary map
	\begin{align*}
		U:\mc H\rightarrow\bigoplus_{n\in\fk N} L^2(\Cbb^N,\mu_n)
	\end{align*}
	satisfying that for every $1\leq j\leq N$,  each $(f_n)_{n\in\fk N}\in \bigoplus_{n\in\fk N} L^2(\Cbb^N,\mu_n)$ belongs to $\Dom(UT_jU^*)=U\Dom(T_j)$ iff $\sum_n\int_{\Cbb^N}|z_jf_n|^2d\mu_n<+\infty$, and that
	\begin{align}
		UT_jU^*\cdot (f_n)_{n\in\fk N}=(z_jf_n)_{n\in\fk N}.
	\end{align}
\end{thm}
Here we let $z_j$ be function indicating the $j$-th component of $X$, i.e., the one sending $(z_1,\dots,z_N)$ to $z_j$. We understood $z_j$ also as the multiplication operator. Then the above relation says $UT_jU^*=z_j$, noting that $\Dom(z_j)$ is precisely the set of all $(f_n)_{n\in\fk N}$ satisfying $\sum_n\int_{\Cbb^N}|z_jf_n|^2d\mu_n<+\infty$.


\begin{proof}
The method here is similar to that in Thm. \ref{lb14}. Let $T_j=V_jH_j$ be the left polar decomposition for $T_j$. Let $R_j=(1+H_j)^{-1}$. For each Borel set $X$ we let $B(X)$ be the set of Borel functions on $X$. Define a map $\Phi:B(\Cbb^N)\rightarrow B(Y)$ (where $Y$ is defined in \eqref{eq13}) sending each $g$ to $\wtd g$ defined by $\wtd g(u_1,r_1,\dots,u_N,r_N)=g(u_1(r_1^{-1}-1),\dots,u_N(r_N^{-1}-1))$. Then $\Phi$ is a homomorphism of unital $*$-algebras. So is $g\mapsto \wtd g(V_1,R_1,\dots,V_N,R_N)$ (which is well-defined by Lemma \ref{lb36}). Thus, for each $\xi\in\mc H$, we have a positive linear functional $g\in C_c(\Cbb^N)\rightarrow\Cbb$, $g\mapsto \bk{\wtd g(V_1,R_1,\dots,V_N,R_N)\xi|\xi}$. As argued in the proof of Thm. \ref{lb8} or \ref{lb14}, we find $U:\mc H\rightarrow\bigoplus_{n\in\fk N}L^2(\Cbb^N,\mu_n)$ such that for each $g\in C_c(\Cbb^N)$,  $U\wtd g(V_1,R_1,\dots,V_N,R_N)U^*$ equals the multiplication of $g$. 

We claim that
\begin{align}
U\wtd g(V_1,R_1,\dots,V_N,R_N)U^*=g	\label{eq12}
\end{align}
when $g$ is one of the following two types of functions: (1) $\nu=z_j/|z_j|$ when $z_j\neq 0$ and $\nu=0$ when $z_j=0$; (2) $\kappa=(1+|z_j|)^{-1}$. Indeed, when $g$ is $\nu$ or $\kappa$, we can find a sequence $g_\blt\in C_c(\Cbb^N)$ such that $|g_\blt|\leq |g|$ and $\lim g_\blt=g$ pointwise. The same is true for $\wtd g_\blt$ and $\wtd g$. Choose any $\xi\in\mc H$ and write $U\xi=(h_n)_{n\in\fk A}$. Let $E$ be the resolution of the identity for $V_1,R_1,\dots,V_N,R_N$. Then
\begin{align*}
	\int_{\Cbb^{2N}} \wtd g_\blt\bk{dE\xi|\xi}=\bk{\wtd g_\blt(V_1,R_1,\dots,V_N,R_N)\xi|\xi}=\sum_n\int_{\Cbb^N}g_\blt|h_n|^2d\mu_n	
\end{align*} 
together with the dominated convergence theorem proves \eqref{eq12} for these two types.

%for any bounded Borel $g$. Let $\mc E$ be the set of all Borel $\Omega\subset\Cbb^N$ such that \eqref{eq12} holds with $g=\chi_\Omega$ (the characteristic function of $\Omega$). Suppose first of all that $g=\chi_\Omega$ where $\Omega$ is open. Then by Urysohn's lemma, we can find an increasing sequence of positive functions $g_k\in C_C(\Cbb^N)$ converging pointwise to $g$. So $\wtd g_k$ is also increasing and converges pointwise to $\wtd g$. So $\mc E$ contains any open set (in particular $\Cbb^N$).  Now  If $\Omega\in\mc E$, then $\chi_{\Omega^c}=1-\chi_{\Omega}$ and  $\wtd \chi_{\Omega^c}=\wtd 1-\wtd\chi_{\Omega}$ show $\Omega^c\in\mc E$. If $\Omega_1,\Omega_2\in\mc E$, then $\chi_{\Omega_1\cap\Omega_2}=\chi_{\Omega_1}\chi_{\Omega_2}$ and hence $\wtd\chi_{\Omega_1\cap\Omega_2}=\wtd\chi_{\Omega_1}\wtd\chi_{\Omega_2}$ shows $\Omega\cap\Omega\in\mc E$. Consequently, $\mc E$ is closed under finite unions. Another application of monotone convergence theorem shows that $\mc E$ is closed under infinite union of an increasing sequence of subsets. So $\mc E$ is a $\sigma$-algebra containing all open sets, which must therefore contain all Borel subsets. So \eqref{eq12} holds whene $g$ is any characteristic function, hence whenever $g$ is bounded positive Borel by monotone convergence theorem again, and hence for any bounded Borel $g$.

Let $h=|z_j|$ so that $\kappa=1/(1+h)$. We claim that  $\wtd \nu=v_j$ and $\wtd \kappa=r_j$. Then $\wtd \nu(V_1,R_1,\dots,V_N,R_N)=V_j$ and $\wtd k(V_1,R_1,\dots,V_N,R_N)=(1+H_j)^{-1}$. By \eqref{eq12}, $U(1+H_j)^{-1}U^*=U\wtd k(V_1,R_1,\dots,V_N,R_N)U^*=\kappa=(1+h)^{-1}$ so that $UH_jU^*=h$, and similarly $UV_jU^*=\nu$. So $UT_jU^*=\nu \centerdot h=z_j$, which finishes the proof. (That $\nu\centerdot h=\nu h=z_j$ follows from the obvious fact that $h=|z_j|$ and $z_j$ have domains as multiplication operators.)  

Recall from \eqref{eq13} that $v_j=0$ iff $r_j=1$. By the definition of $\nu$, we see that $\wtd\nu$ equals $0$ when $v_j(r_j^{-1}-1)=0$, i.e., when $v_j=0$. If $v_j\neq 0$, then by \eqref{eq13}, we know that $|v_j|=1,r_j\in(0,1)$, so $\wtd\nu=\frac{v_j(r_j^{-1}-1)}{|v_j(r_j^{-1}-1)|}=v_j$. So $\wtd \nu$ is always $v_j$. Similarly, $\wtd\kappa=1/(1+|v_j(r_j^{-1}-1)|)$. If $v_j=0$, then $r_j=1$, and $\wtd\kappa=1=r_j$; if $v_j\neq 0$, then $|v_j|=1,r_j\in(0,1)$, so $\wtd\kappa=1/(1+r_j^{-1}-1)=r_j$. We always have $\wtd\kappa=r_j$. This finishes the proof.
\end{proof}







\section{Approximating unbounded closed operators by bounded ones}\label{lb68}



We begin with the following observation. Note the easy fact that any (densely defined) continuous closed  operator from $\mc H_1$ to $\mc H_2$ must be bounded, i.e., have domain $\mc H_1$.

\begin{pp}\label{lb42}
Let $T:\mc H_1\rightarrow\mc H_2$ be a closed operator, and let $A$ be a bounded operator on $\mc H_1$. Assume $TA$ has dense domain. 
\begin{enumerate}
\item $TA$ is closed.
\item If the linear map $TA:\Dom(TA)\rightarrow\mc H_2$ is continuous, then $TA$ is an (everywhere defined and) bounded operator from $\mc H_1$ to $\mc H_2$. In particular, $A\mc H_1\subset\Dom(T)$.
\end{enumerate}
\end{pp}



\begin{proof}
If $\xi_n\in\Dom(TA)$ converges to $\xi$ and $TA\xi_n$ converges, then $A\xi_n\in\Dom(T)$ converges to $A\xi$. Since $T$ is closed, we conclude that $A\xi\in\Dom(T)$ and $TA\xi$ is the limit of $TA\xi_n$. This proves that $TA$ is closed.

Now assume $TA$ is continuous. Since any closed continuous operator is (everywhere defined and) bounded, $TA$ is in particular so.
\end{proof}

\begin{df}\label{lb39}
Let $T$ be a preclosed operator from $\mc H_1$ to $\mc H_2$. A net $E_\blt=(E_\alpha)_{\alpha\in\fk A}$ of projections on $\mc H_1$ is called a net of \textbf{right bounding projections} for $T$ if the following hold:
\begin{itemize}
\item $E_\blt$ is increasing. Namely, if $\alpha\leq\beta$, then $\Rng(E_\alpha)\subset \Rng(E_\beta)$ (equivalently, $E_\alpha=E_\alpha E_\beta$). 
\item $E_\blt$ converges strongly to $\id_{\mc H_1}$. Equivalently, $\ovl{\bigcup \Rng(E_\blt)}=\mc H_1$.
\item For each $\alpha\in\fk A$ there exists a bounded operator $F_\alpha$ (not necessarily a projection) on $\mc H_2$ such that
\begin{gather*}
F_\alpha T\subset TE_\alpha,\\
\lim F_\blt \text{ converges strongly to some }F\in\End(\mc H_2).	
\end{gather*}
In particular, $TE_\alpha$ has dense domain (containing $\Dom(T)$).
\item Each $TE_\alpha$ is continuous, equivalently, $T|_{\Rng(E_\alpha)}$ is continuous.
\end{itemize}
If we can choose $F_\blt=E_\blt$,  we say $E_\blt$ is a net of \textbf{(two-sided) bounding projections} for $T$. When $T$ is closed, a net of \textbf{left bounding projections} for $T$ is by definition a net of right bounding projections for $T^*$.
\end{df}

By Prop. \ref{lb26}, if $E_\blt$ is a net of right (resp. two-sided) bounding projections for a preclosed $T$, then it is so for $\ovl T$. Then (by Prop. \ref{lb42}) each $\ovl TE_\alpha$ is (everywhere defined and) bounded.


\begin{eg}\label{lb37}
Let $T=VH=KV$ be the left and right polar decomposition for a closed operator $T:\mc H_1\rightarrow\mc H_2$. Let $\mc H_1$ be unitarily equivalent via a unitary operator $U$ to $\bigoplus_{n\in\fk N}L^2([0,+\infty),\mu_n)$ where each $\mu_n$ is a positive finite  Borel measure, such that $UHU^*$ is the multiplication of the identity function $x$. For each $r\geq 0$, let $\chi_{[0,r]}$ be the multiplication operator of the characteristic function of $[0,r]$. Then $\chi_{[0,r]}$ increases and converges to $1$ as $r\nearrow+\infty$. It is clear that $\chi_{[0,r]}$ is a net of bounding projections for $x$.  

Let $E_r=U^*\chi_{[0,r]}U$. Then $E_r$ is a net of bounding projections for $H$, and hence a net of right bounding projections for $T$. $F_r:=VE_rV^*$ is a net of bounding projections for $K=VHV^*$, and hence (noting $T^*=V^*K$) a net of left bounding projections for $T$. We have $F_rT\subset TE_r$. We say $E_\blt,F_\blt$ are respectively the \textbf{right} and \textbf{left bounding projections for $T$ via polar decompositions}.  

We note that each $E_r$ is in the von Neumann algebra generated by $(1+H)^{-1}$ (and is hence in $\{T\}''$), and similarly $F_r$ is in the von Neumann algebra generated by $(1+K)^{-1}$ (and is hence in $\{T\}''$). (This fact will be generalized later, cf. Thm. \ref{lb50}.)
\end{eg}


\begin{proof}
The statement about bounding projections is easy to check. Note that $F_rT\subset TE_r$ is from $E_rH\subset HE_r$.
	
We explain why each $E_r$ is in $\{(1+H)^{-1}\}''$; equivalently, $\chi_r\in\{(1+x)^{-1}\}''$. Let $h=(1+x)^{-1}$. Then each polynomial of $h$ is in $\{h\}''$. Note that $h$ takes values in $[0,1]$. Then, by Stone-Weierstrass theorem,  $f\circ h$ is in $\{h\}''$  whenever $f\in C([0,1])$. Now choose a sequence $f_n\in C([0,1])$ with $\lVert f_n\lVert_\infty\leq1$ and converging pointwise to $\chi_{[1/(1+r),1]}$. Then $f_n\circ h\in\{h\}''$ is uniformly (with respect to $n$ and $[0,+\infty)$) bounded and converging pointwise  to $\chi_{[1/(1+r),1]}\circ h=\chi_{[0,r]}$. Dominated convergence theorem shows $f_n\circ h$ (as a sequence of multiplication operators) converges strongly to $\chi_{[0,r]}$. So $\chi_{[0,r]}\in\{h\}''$.
\end{proof}








The reason we are interested in left and right bounding projections is due to the following property.

\begin{thm}\label{lb44}
Let $E_\blt=(E_\alpha)_{\alpha\in\fk A}$ be a net of right bounding projections for a preclosed unbounded operator $T:\mc H_1\rightarrow\mc H_2$. Then the dense subspace
\begin{align*}
\Dom_0:=\bigcup_{\alpha\in\fk A}\Rng(E_\alpha)	
\end{align*}
of $\mc H_1$ is inside $\Dom(T)$ and is a core for $T$.
\end{thm}


\begin{proof}
Let $F_\blt$ converge strongly to $F\in\End(\mc H_2)$. That $F_\alpha T\subset TE_\alpha$ shows that $E_\alpha\Dom(T)\subset\Dom(T)$. Suppose $\xi\in\Dom(T)$. Then $E_\blt\xi$ belongs to $\Dom(T)$ and converges to $\xi$ (since $E_\blt\rightarrow 1$). Moreover, $TE_\blt\xi=F_\blt T\xi$ converges  to $FT\xi$.  As $\ovl T$ is closed, and $\ovl TE_\blt\xi\rightarrow FT\xi$, we conclude $\ovl T\xi=FT\xi$ and hence $T\xi=F T\xi$. Now, as $E_\blt\xi\in\Dom(T)$ converges to  $\xi\in\Dom(T)$ and $TE_\blt\xi$ converges to $T\xi$ for each $\xi\in\Dom(T)$, we conclude that $\Dom_0$ is a core for $T$.
\end{proof}

The above proof shows:

\begin{lm}\label{lb40}
Assume $T:\mc H_1\rightarrow\mc H_2$ is preclosed. Let $F\in\End(\mc H_2)$ be the strong operator limit of $F_\blt$ in Def. \ref{lb39}. Then
\begin{align}
FT=T.	
\end{align}
\end{lm}


Thus, to determine a preclosed $T$, it suffices to restrict to each $\Rng(E_\alpha)$ on which $T$ is continuous (and hence (everywhere defined and) bounded when $T$ is closed).


We give another approximation theorem, which is more useful for the strong commutativity problem.

\begin{thm}\label{lb41}
Let $E_\blt=(E_\alpha)_{\alpha\in\fk A}$ be a net of right bounding projections for a closed operator $T$ on $\mc H$. Then $\{T\}''\subset \{TE_\alpha:\alpha\in\fk A\}''$. If moreover each $E_\alpha$ is in $\{T\}''$ (e.g. the case in Example \ref{lb37}), then $\{T\}''=\{TE_\alpha:\alpha\in\fk A\}''$.
\end{thm}

As an application, we get an equivalent condition for the strong commutativity of closed operators $T$ and $S$: that the bounded operator $TE_r$ commutes strongly with $S$, where $E_r$ is as in Example \ref{lb37}.


\begin{proof}
Choose any $A\in\{TE_\blt\}'$. Then $ATE_\alpha=TE_\alpha A$ as (everywhere defined) bounded operators (since $A$ and $TE_\alpha$ are both bounded, notice Prop. \ref{lb42}).  Choose any $\xi\in\Dom(T)$. Then $E_\alpha A$ converges to $A\xi$. Also $TE_\alpha A\xi=ATE_\alpha\xi=AF_\alpha T\xi$ converges to $AFT\xi$ since the sequence $AF_\alpha$ converges strongly to $AF$. By Lemma \ref{lb40}, $FT\xi=T\xi$. So $TE_\alpha A\xi$ converges to $AT\xi$. Thus, as $T$ is closed, we conclude $A\xi\in\Dom(T)$ and $TA\xi=AT\xi$. This proves $AT\subset TA$. Similarly, $A^*T\subset TA^*$.
	
	
Now assume each $E_\alpha\in\{T\}''$. We shall show that each $TE_\alpha$ is in $\{T\}''$,  equivalently, that each $TE_\alpha$ commutes adjointly with $\{T\}'$. Choose any $A\in\{T\}'$ (i.e. $A$ commutes strongly with $T$). Note that $E_\alpha\in\{T\}''$ implies $A$ commutes adjointly with $E_\alpha$. Then $AT\subset TA$, so $ATE_\alpha\subset TAE_\alpha=TE_\alpha A$. Similarly, $A^*T\subset TA^*$ implies $A^*TE_\alpha\subset TE_\alpha A^*$. So $A$ commutes adjointly with $TE_\alpha$.
\end{proof}






\begin{thm}\label{lb38}
Let $S,T$ be strongly commuting closed operators on $\mc H$. Assume $\xi\in\Dom(TS)\cap\Dom(T)$. Then $\xi\in\Dom(ST)$ and $ST\xi=TS\xi$.
\end{thm}


\begin{proof}
Let $E_r,F_r$ be  nets of right resp. left bounding projections of $T$ as in Example \ref{lb37}, which satisfies $E_r,F_r,TE_r\in\{T\}''$ also by Thm. \ref{lb41}. Then these three commute adjointly with $\{S\}''$, equivalently, commute strongly with $S$. So $E_rS\subset SE_r$, and $TE_rS\subset STE_r$ which shows $TE_r\Dom(S)\subset\Dom(S)$. 

Choose $\xi\in\Dom(TS)\cap\Dom(T)$, i.e., $\xi\in\Dom(S)\cap\Dom(T)$ and $S\xi\in\Dom(T)$. Then $TE_r\xi\in\Dom(S)$, and $STE_r\xi=TE_rS\xi=F_rTS\xi$. As $r\rightarrow+\infty$, we have $TE_r\xi=F_rT\xi\rightarrow T\xi$ and $STE_r\xi=F_rTS\xi\rightarrow TS\xi$. We see $(TE_r\xi,STE_r)$ approaches $(T\xi,TS\xi)$. Since $S$ is closed, $(T\xi,TS\xi)$ must be on the graph of $S$. So $T\xi\in\Dom(S)$ and $ST\xi=TS\xi$.
\end{proof}


The above theorem does not imply $ST=TS$ when $S,T$ commute strongly, since we don't know whether $\Dom(ST)$ equals $\Dom(TS)$ or not.







In the case that we have bounding projections for several strongly commuting normal closed operators $T_1,\dots,T_N$ on $\mc H$, we have an approximation for polynomials of these operators and adjoints. 

Let us for now assume $T_1,\dots,T_N$ are closed, but not necessarily normal or strongly commuting. To begin with, a polynomial $p(T_1,T_1^*,\dots,T_N,T_N^*)$ of $T_1,T_1^*,\dots,T_N,T_N^*$ is by definition a finite linear combination of multiplications and powers of  $T_1,T_1^*,\dots,T_N,T_N^*$, e.g.
\begin{align*}
\sqrt 2 T_1^2(T_1^*)^5T_3^6T_1^3-(1+\sqrt 2\im)(T_4^*)^7T_1^3(T_4^*)^2T_4.
\end{align*}
We also define its adjoint polynomial $p^*(T_1,T_1^*,\dots,T_N,T_N^*)$ in an obvious way, by sending each complex number to its conjugate, the order of operators is reversed, and $T_j$ and $T_j^*$ are exchanged. For instance, the adjoint polynomial of the above expression is 
\begin{align*}
\sqrt 2 (T_1^*)^3(T_3^*)^6T_1^5(T_1^*)^2-(1-\sqrt 2\im)T_4^*T_4^2(T_1^*)^3T_4^7.
\end{align*}
By Prop. \ref{lb9}, if $p$ is densely defined, then
\begin{align}
p^*(T_1,T_1^*,\dots,T_N,T_N^*)\subset p(T_1,T_1^*,\dots,T_N,T_N^*)^*.	\label{eq14}
\end{align}
In particular, if both $p(\cdots)$ and $p^*(\cdots)$ are densely defined, then $p(\cdots)$ is preclosed since the domain of its adjoint contains a dense subspace, which is the domain of $p^*(\cdots)$.


Note that even in the case that $T_1,\dots,T_N$ are normal and commute strongly, it is not a priori true that each $T_i$ and $T_j$ (or $T_j^*$) commute, due to the domain issue mentioned above.



\begin{pp}\label{lb43}
Let $T_1,\dots,T_N$ be closed operators on $\mc H$. 
\begin{enumerate}
\item  If $T_1,\dots,T_N$ are normal and strongly commuting, then there is a sequence $E_n$ of (two-sided) bounding projections for $T_1,\dots,T_N$ (and hence for $T_1^*,\dots,T_N^*$).
\item Suppose there is a net $E_\blt$ of (two-sided) bounding projections for $T_1,\dots,T_N$. Let $p(T_1,T_1^*,\dots,T_N,T_N^*)$ be a polynomial of $T_1,T_1^*,\dots T_N,T_N^*$. Then $p(T_1,T_1^*,\dots,T_N,T_N^*)$ is densely defined and preclosed, and $E_\blt$  is also a net of bounding projections for $p(T_1,T_1^*,\dots,T_N,T_N^*)$ (and hence for its closure).
\end{enumerate}
\end{pp}	


\begin{proof}
1. We assume the setting of Thm. \ref{lb35}. Let $\mc D_r$ be the open disc in $\Cbb$ with center $0$ and radius $r$. We let $E_n=U^*\chi_{\mc D_n\times\cdots\times \mc D_n}U$. Then one checks easily that $(E_n)_{n\in\Nbb}$ is a sequence of bounding projections for $T_1,\dots,T_N$.
	
2. Since $E_\alpha T_j\subset T_jE_\alpha$ and $T_jE_\alpha$ is continuous (and hence bounded by Prop. \ref{lb42}), we have $\Rng(E_\alpha)=E_\alpha\mc H\subset \Dom(T)$. Moreover, $T_jE_\alpha$ and hence $T_j|_{\Rng(E_\alpha)}$ are bounded and leave $\Rng(E_\alpha)$ invariant. The same can be said about $T_j^*|_{\Rng(E_\alpha)}$. Thus, the action of $p(T_1,T_1^*,\dots,T_N,T_N^*)$  on each $\Rng(E_\alpha)$  is (everywhere defined and) bounded, which equals $p(T_1E_\alpha,T_1^*E_\alpha,\dots,T_NE_\alpha,T_N^*E_\alpha)$. Thus $p(T_1,T_1^*,\dots,T_N,T_N^*)$ has domain containing $\Dom_0=\bigcup_{\alpha\in\fk A}\Rng(E_\alpha)$. Similarly, $p^*(\cdots)$ has dense domain. Thus, by \eqref{eq14}, $p(T_1,T_1^*,\dots,T_N,T_N^*)$ is preclosed since its adjoint has dense domain. Using Prop. \ref{lb25}, one checks easily that $E_\alpha p(\cdots)\subset p(\cdots)E_\alpha$. 
%Finally, note that $U\pi(f^*)U^*=f^*=(v_f|f|)^*=|f|v_f$ where the functions are understood as multiplication operators. ($f^*$ is the multiplication of $f$, not a priori the adjoint of the multiplication of $f$.)
\end{proof}


\section{Unbounded Borel functional calculus}


In this section, we let $\scr B(X)$ \index{BX@$\scr B(X),B(X)$} be the unital $*$-algebra of complex-valued (non-necessarily bounded) Borel functions on a Borel set $X$. The algebra structure is given in an obvious way, and the $*$-structure is given by $f^*(x)\equiv\ovl f(x)=\ovl{f(x)}$. We let $B(X)$ be the unital $*$-subalgebra of bounded Borel functions.




We fix strongly commuting normal closed operators $T_1,\dots,T_N$ on $\mc H$. In the following statement of the theorem, to avoid confusion of notations, we write the closure of a preclosed operator $A$ as $A^{**}$. Let $z_j$ denote the function sending $(z_1,\dots,z_N)$ to $z_j$.


\begin{thm}[Unbounded Borel functional calculus]\label{lb47}
There is a unique map $\pi$ from $\scr B(\Cbb^N)$ to the set of closed normal operators on $\mc H$ satisfying the following conditions for each $f,g\in\scr B(\Cbb^N),a,b\in\Cbb$.
\begin{enumerate}
\item $\pi(f)$ commutes strongly with $\pi(g)$.
\item $\pi(z_j)=T_j$ for each $1\leq j\leq N$.
\item $\pi(1)=1$, $\pi(af+bg)=\big(a\pi(f)+b\pi(g)\big)^{**}$, $\pi(fg)=\big(\pi(f)\pi(g)\big)^{**}$, $\pi(f^*)=\pi(f)^*$. (Note Prop. \ref{lb43} for the preclosedness.)
\item If $f\in B(\Cbb^N)$, then $\pi(f)$ is a bounded linear operator on $\mc H$.
\item Assume $f\in B(\Cbb^N)$, and $(f_\alpha)_{\alpha\in\fk A}\in B(\Cbb^N)$ is a net such that $\sup_{\alpha\in\fk A}\lVert f_\alpha\lVert_{L^\infty(\Cbb^N)}<+\infty$, and that $\int_{\Cbb^N}|f_\blt-f|d\mu\rightarrow 0$ for each finite (positive) Borel measure $\mu$ on $\Cbb^N$. Then $\pi(f_\blt)\rightarrow\pi(f)$ strongly. 
\end{enumerate}
Moreover, for any $\pi$ satisfying the above conditions, if we choose spectral decomposition as in Thm. \ref{lb35}, then for each $f\in\scr B(\Cbb^N)$, $U\pi(f)U^*$ equals $f$, i.e.  the multiplication operator of $f$ on $\bigoplus_{n\in\fk N}L^2(\Cbb^N,\mu_n)$.
\end{thm}

We write
\begin{align}
f(T_1,\dots,T_N):=\pi(f)	
\end{align}
if we want to stress the dependence on the operators.


\begin{proof}[Proof of existence]
Choose spectral decomposition as in Thm. \ref{lb35}. We define $\pi(f)=U^*fU$ for each $f\in\scr B(\Cbb^N)$.	By Example \ref{lb33}, $f$ is normal with left  and right polar decomposition $f=v_f|f|=v_f\centerdot|f|=|f|\centerdot v_f$, where $v_f$ is defined to be $f/|f|$ when $f\neq 0$, and $0$ otherwise. So $\pi(f)$ is normal with left and right polar decompositions $\pi(f)=\pi(v_f)\pi(|f|)=\pi(|f|)\pi(v_f)$. Clearly $\pi(z_j)=T_j$, and $\pi(f)$ is bounded when $f$ is so. For any other $g\in\scr B(\Cbb^N)$ with left polar decomposition $g=v_g\centerdot|g|$, it is clear that $v_g$ and $(1+|g|)^{-1}$ commute strongly with $v_f$ and $(1+|f|)^{-1}$. So the multiplication operators $f,g$ commute strongly by Cor. \ref{lb34}. Hence $\pi(f),\pi(g)$ commute strongly.

Clearly $\pi(1)=1$. For each $n\in\Zbb_+$, let $\Omega_n\subset\Cbb^N$ be the set of all points at which $|f|,|g|,|af+bg|,|fg|,|f^*|,|g^*|\leq n$. As $\chi_{\Omega_n}$ commute strongly with $f,g,fg$ (as proved above), it is clear that $\chi_{\Omega_n}$ is a sequence of (two-sided) bounding projections for both $f,g,fg$. So $E_n=U^*\chi_{\Omega_n}U$ is a sequence of bounding projections for $\pi(f),\pi(g),\pi(fg)$, and also for $\ovl{\pi(f)\pi(g)}$ by Prop. \ref{lb43}. It is clear that $\pi(fg)$ and $\ovl{\pi(f)\pi(g)}$ are equal on each $\Rng(E_n)$ (noting that $U\Rng(E_n)=\Rng(\chi_{\Omega_n})$ is the set of $(f_\blt)\in \bigoplus L^2(\Cbb^N,\mu_n)$ which are $0$ outside $\Omega_n$). So $\pi(fg)$ and $\ovl{\pi(f)\pi(g)}$ are equal on $\Dom_0:=\bigcup_n\Rng(E_n)$, which by Thm. \ref{lb44} is a core for both operators. So they are the same closed operators. The same method shows also $\pi(af+bg)=\ovl{\pi(af+bg)}$. That $\pi(f)^*=\pi(f^*)$ follows from the same method, or follows from that the adjoint of the multiplication of $f$ equals the multiplication of $f^*$ (cf. Example \ref{lb33}).

Finally, for bounded Borel functions, the strong convergence of $\pi(f_\blt)\rightarrow\pi(f)$ for a uniformly $L^\infty$-bounded $f_\blt$ converging to $f$ in the $L^1(\Cbb^N,\mu)$ norm (for every finite positive Borel $\mu$) can be proved using exactly the same method as in Thm. \ref{lb28}.
\end{proof}	



\begin{proof}[Proof of uniqueness]
Step 1. If $\Omega$ is a Borel subset of $\Cbb^N$, then $\pi(\chi_\Omega)^*=\pi(\chi_\Omega^*)=\pi(\chi_\Omega)$ and $\pi(\chi_\Omega)^2=\pi(\chi_\Omega^2)=\pi(\chi_\Omega)$ shows $\pi(\chi_\Omega)$ is a projection. If $f\in\scr B(\Cbb^N)$ is bounded on $\Omega$, then $\pi(\chi_{\Omega})\pi(f)\subset \ovl{\pi(\chi_{\Omega})\pi(f)}=\pi(f\chi_\Omega)$. Also, since $\pi(f)\pi(\chi_\Omega)$ has closure $\pi(f\chi_\Omega)$, by Prop. \ref{lb42}, we learn that $\pi(f)\pi(\chi_\Omega)$ is bounded and equals its closure. We conclude $\pi(\chi_\Omega)\pi(f)\subset\pi(f)\pi(\chi_\Omega)$. Simiarly, $\pi(\chi_\Omega)\pi(f)^*\subset\pi(f)^*\pi(\chi_\Omega)$ since $\pi(f)^*=\pi(f^*)$. It then follows that if $\Omega_\blt=(\Omega_\alpha)_{\alpha\in\fk A}$ is an increasing net of Borel subsets of $\Cbb^N$ satisfying that $\bigcup_\alpha \Omega_\alpha=\Cbb^N$, and that on each $\Omega_\alpha$ the function $f$ is bounded, then $\pi(\chi_{\Omega_\blt})$ is a net of (two-sided) bounding projections for $\pi(f)$. 

By Thm. \ref{lb44}, the closed operator $\pi(f)$ is determined by its restriction to each $\Rng(\pi(\chi_{\Omega_\alpha}))$, and hence determined by $\pi(f)\pi(\chi_{\Omega_\alpha})=\pi(f\chi_{\Omega_\alpha})$, where $f\chi_{\Omega_\alpha}\in B(\Cbb^N)$. Since we can always find such $\Omega_\blt$ for $f$ (e.g., $\Omega_n$ is the set of all points at which $|f|<n$),  it suffices to prove the uniqueness of $\pi$ on $B(\Cbb^N)$.

Step 2. Choose $f\in\scr B(\Cbb^N)$ with positive values. We claim that $\pi(f)$ is a positive closed operator. We have
\begin{align*}
	\pi(f)=\ovl{\pi(f^{\frac 12})\pi(f^{\frac 12})}\supset \pi(f^{\frac 12})\pi(f^{\frac 12})=\pi(f^{\frac 12})^*\pi(f^{\frac 12})
\end{align*}
since the conjugate of $f^{\frac 12}$ is itself. Let $A=\pi(f^{\frac 12})^*\pi(f^{\frac 12})$, which is positive and hence self-adjoint. $\pi(f)$ is also self-adjoint since $f^*=f$. So $\pi(f)\supset A$ implies $\pi(f)^*\subset A^*$, and hence $\pi(f)\subset A$. So $\pi(f)$ equals $A$, which is therefore positive.

Step 3. Choose $f\in\scr B(\Cbb^N)$, and let $\Omega\subset\Cbb^N$ be the subset of all points at which $f\neq 0$. We claim $\ovl{\Rng(\pi(f))}=\Rng(\pi(\chi_\Omega))$. (Recall that $\pi(\chi_\Omega)$ is a projection.) Note that $f=f\chi_\Omega$. So $\ovl {\pi(\chi_\Omega)\pi(f)}= \pi(\chi_\Omega f)=\pi(f)$, which shows $\Rng(\pi(f))\subset\Rng(\pi(\chi_\Omega))$.

Conversely, for each $n\in\Zbb_+$ we let $\Omega_n\subset\Cbb^N$ be the set of all points at which $1/n<|f|$. Then the $L^\infty$-norms  of $\chi_{\Omega_n}$ are uniformly bounded, and $\chi_{\Omega_n}$ converges to $\chi_\Omega$ in the $L^1(\Cbb^N,\mu)$-norm for any positive finite Borel $\mu$. So $\pi(\chi_n)$ converges strongly to $\pi(\chi)$. Thus, to prove   $\ovl{\Rng(\pi(f))}=\Rng(\pi(\chi_\Omega))$, it suffices to show $\Rng(\pi(\chi_{\Omega_n}))\subset\ovl{\Rng(\pi(f))}$ for each $n$. Define $g_n\in B(\Cbb^N)$ to be $0$ outside $\Omega_n$ and $1/f$ in $\Omega_n$. Then $fg_n=\chi_{\Omega_n}$. So $\pi(\chi_{\Omega})=\ovl{\pi(f)\pi(g_n)}$ proves $\Rng(\pi(\chi_{\Omega_n}))\subset\ovl{\Rng(\pi(f))}$.
	
Step 4. We know each $\pi(z_j)$ is uniquely determined. Let $v_j$ be $z_j/|z_j|$ when $z_j\neq 0$ and be $0$ otherwise. We know that $T_j=\pi(z_j)=\ovl{\pi(v_j)\pi(|z_j|)}$. We claim that $\pi(v_j)\pi(|z_j|)$ is a left polar decomposition for $T_j$. Then, by the uniqueness of left polar decompositions, $\pi(v_j)$ and $\pi(|z_j|)$ are uniquely determined by $T_j$. 

From the previous steps, we know $\pi(|z_j|)$ is positive, and $\ovl{\Rng(\pi(|z_j|))}$ equals $\Rng(\chi_{\Omega_j})$, where $\Omega_j$ is the set of all points at which $z_j\neq 0$. Since $v_j^*v_j=v_jv_j^*=\chi_{\Omega_j}$, we see that $\pi(v_j)$ is a partial isometry whose source space and target space are both $\Rng(\chi_{\Omega_j})$, which equals $\ovl{\Rng(\pi(|z_j|))}$. Thus, by Remark \ref{lb45}, $\pi(v_j)\pi(|z_j|)$ is closed (and hence equals $T_j$), and this product is a left polar decomposition, which must be that of $T_j$.

Step 5. Let $r_j=(1+|z_j|)^{-1}\in B(X)$. Then by Prop. \ref{lb42},  $\pi(1+|z_j|)\pi(r_j)$ is a closed operator, which must be (everywhere defined and) bounded and equals $1$. So $\pi(1+|z_j|)$ sends each $\pi(r_j)\xi$ to $\xi$, and $\pi(r_j)$ is injective (and has dense range by \eqref{eq19}.) We conclude $\pi(1+|z_j|)\supset A$ where $A$ is the inverse of $\pi(r_j)$ (which is self-adjoint by Example \ref{lb13}), and hence $\pi(1+|z_j|)=\pi(1+|z_j|)^*\subset A^*=A$. This proves $\pi(1+|z_j|)$ equals the $A$, the inverse  of $\pi(r_j)$. So $\pi(r_j)$ is unique since (we have proved) $\pi(1+|z_j|)$ is so.



Step 6. Since $\pi$ restricts to a unital $*$-homomorphism $B(\Cbb^N)\rightarrow\End(\mc H)$, it is clear that $\pi(p\circ(v_1,v_1^*,r_1,\dots,v_N,v_N^*,r_N))$ is uniquely determined for any polynomial $p$ of $v_1,v_1^*,r_1,\dots,v_N,v_N^*,r_N$. Let $Y=\mbb S^1\times (0,1)\cup \{0\}\times \{1\}\subset \Cbb^2$.  Then we have a bijective Borel map $(v_1,r_1,\dots,v_N,r_N):\Cbb^N\rightarrow Y^N$. Since $Y^N$ is a bounded subset of $(\Cbb\times\Rbb)^N$, it follows that  each $g\in C(\ovl {Y^N})$ (where the closure $\ovl{Y^N}=\ovl Y^N=(\mbb S^1\times[0,1]\cup \{0\}\times\{1\})^N$ is compact) can be approximated uniformly by a sequence of polynomials $p_n$. By Prop. \ref{lb46}, we conclude that $\pi(p_n\circ(v_1,v_1^*,r_1,\dots,v_N,v_N^*,r_N))$ approaches in the norm topology to $\pi(g\circ(v_1,r_1,\dots,v_N,r_N))$. Thus $\pi(g\circ(v_1,r_1,\dots,v_N,r_N))$ is uniquely determined.

Now, for each $f\in C_c(\Cbb^N)$, set $\wtd f\in B(\ovl Y^N)$ such that for each point $(u_1,\rho_1,\dots,u_N,\rho_N)\in \ovl Y^N$,  $\wtd f(u_1,\rho_1,\dots,u_N,\rho_N)$ equals $f(u_1(\rho_1^{-1}-1),\dots,u_N(\rho_N^{-1}-1))$ when $\rho_1\cdots\rho_N\neq0$, and equals $0$ otherwise. Using the fact that $f$ has compact support, one checks easily that $\wtd f$ is continuous on $\ovl Y^N$. One checks that $\wtd f\circ(v_1,v_1^*,r_1,\dots,v_N,v_N^*,r_N)$ equals $f$. 
So $\pi(f)$ is unique whenever $f\in C_c(\Cbb^N)$. By Lemma \ref{lb18}, $\pi(f)$ is also unique in the general case $f\in B(\Cbb^N)$.
\end{proof}


\begin{thm}\label{lb50}
For each $f\in\scr B(\Cbb^N)$, $f(T_1,\dots,T_N)$ is affiliated with $\{T_1,\dots,T_N\}''$.
\end{thm}

\begin{proof}
If we can prove that $f(T_1,\dots,T_N)$ commuts strongly with any element $U$ in $\mc M=\{T_1,\dots,T_N\}'$, then $\mc M\subset\{f(T_1,\dots,T_N)\}'$, which proves $\{f(T_1,\dots,T_N)\}''\subset \{T_1,\dots,T_N\}''$. By Prop. \ref{lb29}, it suffices to assume $U$ is unitary and prove (cf. Prop.\ref{lb31}) that $Uf(T_1,\dots,T_N)U^*=f(T_1,\dots,T_N)$. Note that this is true when $f=z_1,\dots,z_N$ since $UT_jU^*=T_j$. Thus, the map $\pi:f\in\scr B(\Cbb^N)\mapsto Uf(T_1,\dots,T_N)U^*$ satisfies all the conditions in Thm. \ref{lb47}. So $\pi$ is the unique functional calculus, which equals $f\mapsto f(T_1,\dots,T_N)$. This finishes the proof.

An alternative proof: By Thm. \ref{lb30}, if we take left polar decomposition of each $T_j=V_jH_j$, then $V_j,(1+H_j)^{-1}\in\mc N:=\{T_1,\dots,T_N\}''$. The proof of uniqueness in Thm. \ref{lb47} shows that any $f(T_1,\dots,T_N)$ (where $f\in B(\Cbb^N)$) can be approximated strongly by polynomials of $V_1,(1+H_1)^{-1},\dots,V_N,(1+H_N)^{-1}$. Hence it is in $\mc N$. In the general case that $g\in\scr B(\Cbb^N)$, one constructs bounding projections and uses Thm. \ref{lb41} to conclude $\{f(T_1,\dots,T_N)\}''\subset\mc N$.
\end{proof}


We discuss compositions of functional calculus. In the following, we write $T_1,\dots,T_N$ as $T_\blt$ for short.

\begin{thm}\label{lb48}
Let $g_1,\dots,g_L\in \scr B(\Cbb^N)$ and $f\in\scr B(\Cbb^L)$. Then
\begin{align}
f(g_1(T_\blt),\dots,g_L(T_\blt))=(f\circ(g_1,\dots,g_L))(T_\blt).\end{align}
\end{thm}

Note that on the left hand side, the functional calculus of $g_1,\dots,g_L$ is defined using $T_\blt$, and $f$ is defined using $g_1(T_\blt),\dots,g_L(T_\blt)$. On the right hand side, we have the functional calculus of the function $f\circ(g_1,\dots,g_L)$ defined using $T_\blt$.


\begin{proof}
Define a map $\pi:f\in\scr B(\Cbb^L)\mapsto (f\circ(g_1,\dots,g_L))(T_\blt)$. It suffices to check that this is the unique functional calculus for $g_1(T_\blt),\dots,g_N(T_\blt)$. Namely, we shall verify all the conditions in Thm. \ref{lb47} (with $T_j$ replaced by $g_j(T_\blt)$). The only nontrivial condition is the last one about continuity. Choose $f\in B(\Cbb^L)$ and $(f_\alpha)_{\alpha\in\fk A}$ a net satisfying  $\sup_{\alpha\in\fk A}\lVert f_\alpha\lVert_{L^\infty(\Cbb^L)}<+\infty$ and $\int_{\Cbb^L}|f_\alpha-f|d\nu\rightarrow 0$ for each finite (positive) Borel measure $\nu$ on $\Cbb^L$. Now, for any finite positive Borel measure $\mu$ on $\Cbb^N$, we let $\gamma=(g_1,\dots,g_L)$, and let $\nu=\gamma_*\mu$ which is finite positive Borel measure. Then by \eqref{eq16},
\begin{align*}
\int_{\Cbb^N}|f\circ(g_1,\dots,g_L)-f_\alpha\circ (g_1,\dots,g_N)|d\mu=\int_{\Cbb^L}|f-f_\alpha|d\nu
\end{align*}
converges to $0$. So $f_\alpha\circ(g_1,\dots,g_L)(T_\blt)$ converges strongly to $f\circ(g_1,\dots,g_L)(T_\blt)$.
\end{proof}


\begin{co}\label{lb62}
Let  $N,L\in\Zbb_+$, and let $T_1,\dots,T_{N+L}$ be strongly commuting normal closed operators on $\mc H$.  Assume $f\in\scr B(\Cbb^{N+L})$ depend only on the first $N$ variables $z_1,\dots,z_N$, and let $\wtd f\in\scr B(\Cbb^N)$ be the restriction of $f$ to $\Cbb^N\simeq \Cbb^N\times\{0_{\Cbb^L}\}$. Then
\begin{align*}
f(T_1,\dots,T_{N+L})=\wtd f(T_1,\dots,T_N).	
\end{align*}
\end{co}

\begin{proof}
Let $p:\Cbb^{N+L}\rightarrow\Cbb^N$ be the projection on the first $N$ variables. Then $f=\wtd f\circ p$. So by Thm. \ref{lb48}, $f(T_1,\dots,T_{N+L})=\wtd f(p(T_1,\dots,T_{N+L}))=\wtd f(T_1,\dots,T_N)$.
\end{proof}


\begin{rem}
Define $\Sp(T_1,\dots,T_N)$ \index{Sp@$\Sp(T_1,\dots,T_N)$} to be the set of all points in $\Cbb^N$ such that every open set $W$ containing this point satisfies $\chi_W(T_1,\dots,T_N)\neq 0$. This is a closed subset of $\Cbb^N$, called the \textbf{joint spectrum} of $T_1,\dots,T_N$. 

If we choose spectral decomposition as in Thm. \ref{lb35}, then it is clear that $\Sp(T_1,\dots,T_N)$ is the closure of the union of the supports of all $\mu_n$. From this description, we see that $f(T_1,\dots,T_N)$ depends only on the values of $f$ on the joint spectrum. Also, for $f_1,\dots,f_L\in\scr B(\Cbb^N)$,  using Thm. \ref{lb48} (by composing characteristic functions with $(f_1,\dots,f_L)$) one sees that
\begin{align}
\Sp(f_1(T_1,\dots,T_N),\dots,f_L(T_1,\dots,T_L))\subset\ovl{(f_1,\dots,f_L)(\Sp(T_1,\dots,T_N))}.
\end{align}
\end{rem}



\begin{rem}
For strongly commuting normal $T_1,\dots,T_N$, we have
\begin{align}
\Sp(T_1,\dots,T_N)\subset\Sp(T_1)\times\cdots\times\Sp(T_N).	
\end{align}
Indeed, suppose $(\lambda_1,\dots,\lambda_N)\neq \Sp(T_1)\times\cdots\times\Sp(T_N)$. Then one of $\lambda_1,\dots,\lambda_N$ is not in $\Sp(T_j)$, say $\lambda_1\notin\Sp(T_1)$. Choose a neighborhood $W\subset\Cbb$ of $\lambda_1$ such that $\chi_W(T_1)=0$. Then Cor. \ref{lb62} shows that $\chi_{W\times\Cbb^{N-1}}(T_1,\dots,T_N)=\chi_W(T_1)=0$. So $(\lambda_1,\dots,\lambda_N)$, which has $W\times\Cbb^{N-1}$ as a neighborhood, is outside $\Sp(T_1,\dots,T_N)$.
\end{rem}


\begin{exe}
Let $T$ be normal. Use spectral theorem to show that $\Sp(T)\subset[0,+\infty)$ iff $T$ is positive,  that $\Sp(T)\subset\Rbb$ iff $T$ is self-adjoint, and that $\Sp(T)$  is compact iff $T$ is bounded.
\end{exe}



\begin{eg}
To see the power of Theorem \ref{lb48}, we do an example about von Neumann algebras and commutants, which can easily be generalized to other more complicated examples. Let $A,B$ be strongly commuting closed operators on $\mc H$, $A$ is self-adjoint and $B$ is positive. We show that $\{A,B\}''=\{\ovl{A+B^2},A^3\}''$. (Note that by the spectral theorem \ref{lb35}, it is clear that $A^3$ is closed.)
\end{eg}
\begin{proof}[Solution]
Let $C=\ovl{A+B^2}$ and $D=A^3$. That $\{A,B\}''\supset\{C,D\}''$ follows from Thm. \ref{lb50}. Note that $\Sp(A,B)\subset \Sp(A)\times\Sp(B)\subset X:=\mbb R\times[0,+\infty)$. Let $(f_1,f_2)=(z_1+z_2^2,z_1^3)$. Then $(C,D)=(f_1,f_2)(A,B)$. The range of $(f_1,f_2)$ is in $Y:=\{(w_1,w_2):w_2\in\Rbb,w_1-\sqrt[3]{w_2}\geq 0\}$. We can define an inverse function $(g_1,g_2):Y\rightarrow X$ by $g(w_1,w_2)=(\sqrt[3]{w_2},\sqrt{w_1-\sqrt[3]{w_2}})$. Then $g\circ f=(z_1,z_2)$. So, by Thm. \ref{lb48}, $(g_1,g_2)(C,D)=(g_1\circ f_1,g_2\circ f_2)(A,B)=(A,B)$. By Thm. \ref{lb50} again, we conclude $\{A,B\}''\subset\{C,D\}''$.
\end{proof}



For each Borel $\Omega\subset\Sp(T_1,\dots,T_N)$, set
\begin{align}
	E(\Omega)=\chi_\Omega(T_1,\dots,T_n)
\end{align}
as in the bounded case. Then following theorem is similar to Thm. \ref{lb49}.

\begin{thm}
Let $X=\Sp(T_1,\dots,T_N)$. For each $f\in\scr B(X)$, a vector $\xi\in\mc H$ belongs to $\Dom(f(T_1,\dots,T_N))$ if and only if
\begin{align*}
\int_X |f|^2\bk{dE\xi|\xi}<+\infty.	
\end{align*}
Moreover, for such $\xi$, we have
\begin{align}
\bk{f(T_1,\dots,T_N)\xi|\xi}=\int_X f\bk{dE\xi|\xi}.
\end{align}
\end{thm}

Due to the above relation, we also write
\begin{align}
\int_Xf dE:=f(T_1,\dots,T_N).	
\end{align}


\begin{proof}
Choose a spectral decomposition as in Thm. \ref{lb35}, and note that all $\mu_n$ have supports in $X=\Sp(T_1,\dots,T_N)$. So we may replace $\Cbb^N$ by $X$. For each $\xi\in\mc H$, if we write $U\xi=(f_n)_{n\in\fk N}\in\bigoplus_n L^2(X,\mu_n)$, it follows easily that
\begin{align}
\bk{dE\xi|\xi}=\sum_n |f_n|^2d\mu_n.
\end{align}
It is now straightforward to verify the claimed properties.
\end{proof}






\section{Self-adjoint operators, Stone's theorem}


We begin this section with a useful criterion on self-adjoint operators.

\begin{thm}
Let $T$ be a closed and symmetric (i.e. $T\subset T^*$) operator on $\mc H$. The following are equivalent.
\begin{enumerate}[label=(\alph*)]
\item $T$ is self-adjoint.
\item The ranges of $T+\im$ and $T-\im$ are both $\mc H$.
\item  The ranges of $T+\im$ and $T-\im$ are both dense in $\mc H$.
\end{enumerate}
\end{thm}


\begin{proof}
(a) $\Rightarrow$ (b): If $T$ is self-adjoint, then $\Sp(T)\subset \Rbb$. By spectral theorem, one can identify $T$ with the multiplication of $x$ on $\bigoplus L^2(\Rbb,\mu_n)$. It is clear that $x\pm\im$ are surjective since $(x\pm\im)^{-1}$ are bounded so that $\xi=(x\pm\im)(x\pm\im)^{-1}\xi$ for each $\xi\in \bigoplus L^2(\Rbb,\mu_n)$.

(b) $\Rightarrow$ (c): Obvious.

(c) $\Rightarrow$ (b): That $\bk{(T+\im)\xi|(T+\im)\eta}=\bk{T\xi|T\eta}+\bk{\xi|\eta}=\bk{\Psi\xi|\Psi\eta}$ (cf. \eqref{eq7}) for all $\xi,\eta\in\Dom(T)$ shows that $\Rng(T+\im)$ is unitarily equivalent to $\Rng(\Psi)=\fk G(T)$ under the unitary map $(T+\im)\xi\mapsto \Psi\xi$. So $\Rng(T+\im)$ is complete since $\fk G(T)$ is so. So $\Rng(T+\im)$ must be $\mc H$. So does $\Rng(T-\im)$ for a similar reason.

(b) $\Rightarrow$ (a): It suffices to show $\Dom(T)=\Dom(T^*)$. By Prop. \ref{lb9}, $(T+\im)^*=T^*-\im\supset T-\im$. Note that $T-\im$ is surjective. Thus, if we can show that $(T+\im)^*$ is injective, then we have $(T+\im)^*=T-\im$ and therefore $\Dom(T^*)=\Dom(T^*-\im)=\Dom((T+\im)^*)=\Dom(T)$.

Choose any $\xi\in\Dom((T+\im)^*)$ such that $(T+\im)^*\xi=0$. Then $\bk{\xi|(T+\im)\eta}=0$ for each $\eta\in\Dom(T)$, namely, $\xi$ is orthogonal to $\Rng(T+\im)=\mc H$. So $\xi$ must be $0$.
\end{proof}


A \textbf{one parameter unitary group} on $\mc H$ is by definition a strongly continuous map $\Rbb\rightarrow\End(\mc H),t\mapsto U_t$ (i.e., continuous with respect to the strong operator topology) satisfying that each $U_t$ is unitary, $U_0=1$, and $U_{s+t}=U_sU_t$ for each $s,t$. It follows that $U_t^*=U_{-t}$ and $U_s$ commutes adjointly with $U_t$.


\begin{thm}
Let $H$ be a self-adjoint closed operator on $\mc H$. Then $t\mapsto e^{\im tH}$ is a one parameter unitary group on $\mc H$. Moreover, any $\xi\in\mc H$ belongs to $\Dom(H)$ if and only if the limit
\begin{align}
\lim_{t\rightarrow 0}\frac{e^{\im tH}-1}{t}\cdot \xi\label{eq17}	
\end{align}
exists in the strong operator topology. In that case, the limit is $\im H\xi$.
\end{thm}


\begin{proof}
We have $\Sp(H)\subset\Rbb$ since $H$ is self-adjoint. Let $E$ be the resolution of the identity for $H$. By properties of functional calculus, it is clear that $U_t:=e^{\im tH}$ are unitary and satisfies $U_0=1$ and $U_{s+t}=U_sU_t$. Since $U_s\xi-U_t\xi=(U_{s-t}-1)U_t\xi$, to check the strong continuity, it suffices to check $U_t\xi-\xi\rightarrow 0$ for each $\xi$ as $t\rightarrow 0$. But
\begin{align*}
\lVert (e^{\im tH}-1)\xi\lVert^2=\bk{(e^{\im tH}-1)^*(e^{\im tH}-1)\xi|\xi}=\int_\Rbb |e^{\im tx}-1|^2\bk{dE\xi|\xi},
\end{align*}
which converges to $0$ by dominated convergence theorem.

We now choose $\xi\in\Dom(H)$ and show that the derivative of $U_t\xi$ at $t=0$ exists strongly and equals $\im H\xi$. We have
\begin{align}
&\Big\lVert \frac{e^{\im tH}-1}{it}\cdot \xi-H\xi  \Big\lVert^2=\int_\Rbb \Big|\frac{e^{\im tx}-1}{it}-x  \Big|^2\bk{dE\xi|\xi}\nonumber\\
=&	\int_\Rbb \Big|\frac{e^{\im tx}-1}{itx}-1  \Big|^2\cdot |x|^2\bk{dE\xi|\xi},\nonumber
\end{align}
Since $\xi\in\Dom(H)$, we have $\int_\Rbb (1+|x|^2)\bk{dE\xi|\xi}<+\infty$. So by Lemma \ref{lb51} and dominated convergence theorem, the above expression converges to $0$ as $t\rightarrow 0$. This proves that \eqref{eq17} converges strongly to $\im H\xi$ whenever $\xi\in\Dom(H)$.

Conversely, assume $\xi\in\mc H$ satisfies that \eqref{eq17} converges strongly to $\psi\in\mc H$. Then using the result from the last paragraph, for any $\eta\in\Dom(H)$, we have
\begin{align}
\bk{\xi|H\eta}=\lim_{t\rightarrow 0}\bk{\xi|(\im t)^{-1}(e^{\im tH}-1)\eta}=\lim_{t\rightarrow 0}\bk{(-\im t)^{-1}(e^{-\im tH}-1)\xi|\eta}=\bk{\psi|\eta}.	\label{eq18}
\end{align}
This shows $\xi\in\Dom(H^*)=\Dom(H)$ and $H\xi=H^*\xi=\psi$. This proves $\xi\in\Dom(H)$.
\end{proof}





\begin{lm}\label{lb51}
For each $h\in\Rbb-\{0\}$ we have
\begin{align*}
\Big|\frac{e^{\im h}-1}{\im h}-1\Big|\leq 3.	
\end{align*}
\end{lm}

\begin{proof}
If $|h|\geq 1$ then the left hand side is $\leq 3$. If $|h|<1$ then
\begin{align*}
\frac{e^{\im h}-1}{\im h}-1=\im h\sum_{n\in\Nbb}\frac{(\im h)^n}{(n+2)!} 	
\end{align*}
whose absolute value is $\leq |h| e^{|h|}<3|h|$.
\end{proof}

To prove the converse of the above Theorem, we first need:

\begin{lm}\label{lb53}
Let $T$ be a closed operator on $\mc H$ affiliated with an abelian von Neumann algebra $\mc M$ (i.e., any two elements of $\mc M$ commute (strongly)). Then $T$ is normal, and in particular $\Dom(T)=\Dom(T^*)$.
\end{lm}

\begin{proof}
We have $\{T\}''\subset\mc M$. Since $\mc M$ commutes strongly with $\mc M$ (as $\mc M$ is abelian),  $\{T\}''$ commutes strongly with $\{T\}''$. So $T$ commutes strongly with $T$. Hence $T$ is normal by Def. \ref{lb52}. We know $\Dom(T)=\Dom(\sqrt{T^*T})=\Dom(\sqrt{TT^*})=\Dom(T^*)$.
\end{proof}




\begin{thm}\label{lb54}
Let $U_t$ be a one parameter unitary group on $\mc H$. Then $U_t=e^{\im tH}$ for some self-adjoint closed operator $H$ on $\mc H$.
\end{thm}

\begin{proof}
We let $\Dom_0$ be the subspace of all $\xi$ such that $\lim_{t\rightarrow 0}(U_t\xi-\xi)/(\im t)$ exists. We first show that $\Dom_0$ is a dense subspace of $\mc H$. Indeed, any $\xi\in\mc H$ can be approximated by
\begin{align*}
\xi(f)=\int_\Rbb f(s)U_s\xi ds	
\end{align*}
where $f\in C_c(\Rbb)$,  $\int_\Rbb fdt=1$. (Consider a sequence $f_n$ where each $f_n$ has support inside $[-1/n,1/n]$. Then $\xi(f_n)\rightarrow\xi$.) Thus, all such $\xi(f)$ span a dense subspace of $\mc H$. We claim $\xi(f)\in\Dom_0$. Indeed,
\begin{align*}
U_t\xi(f)=\int_\Rbb f(s)U_{s+t}\xi ds=\int_\Rbb f(s-t)U_s\xi ds,	
\end{align*}
whose derivative at $t=0$ converges in norm to $-\xi(f')$. This proves the claim, and we see that $\Dom_0$ is dense in $\mc H$.

We define $\Dom(H)=\Dom_0$, and let $H\xi$ be the derivative of $U_t\xi$ at $t=0$ for each $\xi\in\Dom(H)$. We first show that $H$ is self-adjoint. Using the calculation in \eqref{eq18}, it is easy to check that $\bk{\xi|H\eta}=\bk{H\xi|\eta}$ for each $\xi,\eta\in\Dom(H)$. Thus $H$ is symmetric, i.e. $H\subset H^*$. In particular, $H$ is preclosed. Let $\mc M=\{U_s:s\in\Rbb\}''$, which is abelian since any $U_s,U_t$ commute adjointly. We claim that $H$ is affiliated with $\mc M$, equivalently, that $\mc M'\subset\{H\}'$. By Prop. \ref{lb29}, it suffices to show that $H$ commutes strongly with any unitary $V\in\mc M'$. Indeed, we note that $U_t\xi$ has derivative at $t=0$ iff $(VU_tV^*)\cdot V\xi$ does. So the derivative of $VU_tV^*$ at $t=0$ exists precisely when acting on $V\Dom_0=V\Dom(H)$. When the derivative exists, it must be the action of $VHV^*$. But $VU_tV^*=U_t$ since $V\in\mc M'$. So $H=VHV^*$. This proves $\{H\}''\subset\mc M$. By Lemma \ref{lb53}, we know $\Dom(H)=\Dom(H^*)$. So $H=H^*$.

We now show $U_t=e^{\im tH}$ for each $t$. Note that for each $\xi\in\Dom(H)$, $e^{\im tH}\xi=\xi+\im tH\xi+o(t)$ where $o(t)\in\mc H$ denotes an expression satisfying $\lim_{t\rightarrow 0}o(t)/t\rightarrow 0$. Similarly, $U_t\eta=\eta+i\im tH\eta+o(t)$ for each $\eta\in\Dom(H)$. So
\begin{align*}
&\bk{U_{-t}e^{\im tH}\xi|\eta}=\bk{e^{\im tH}\xi|U_t\eta}=\bk{\xi+\im tH\xi+o(t)|\eta+\im tH\eta+o(t)}\\
=&\bk{\xi|\eta}+\im t(\bk{H\xi|\eta}-\bk{\xi|H\eta})+o(t)=\bk{\xi|\eta}+o(t).	
\end{align*}
Since $H$ commutes strongly with $\mc M$ (as $\mc M$ is abelian and contains $\{H\}''$) and with each $e^{\im sH}$, we have $U_sHU_s^*=H$ and $e^{\im sH}H e^{-\im sH}=H$. So $U_s\Dom(H)=\Dom(H)$ and  $e^{\im sH}\Dom(H)=\Dom(H)$. Using the fact that each $U_s,e^{\im sH}$ commute strongly with $U_t,e^{\im t H}$ since they belong to the abelian von Neumann algebra $\mc M$, we have
\begin{align*}
\bk{U_{-t-s}e^{\im (t+s)H}\xi|\eta}=\bk{U_{-t}e^{\im tH}e^{\im sH}\xi|U_s\eta}=\bk{e^{\im sH}\xi|U_s\eta}+o(t)=\bk{U_{-s}e^{\im sH}\xi|\eta}+o(t).
\end{align*}
Thus, the derivative of $s\mapsto \bk{U_{-s}e^{\im sH}\xi|\eta}$ is zero everywhere, which shows that $U_{-s}e^{\im sH}$ must be constant, which is $1$.
\end{proof}


The following proposition provides a criterion for self-adjoint operators $H,K$ on $\mc H$ to commute strongly: it is equivalent to that $e^{\im tH}$ commutes (adjointly) with $e^{\im sK}$ for each $t,s\in\Rbb$.


\begin{pp}
Let $H$ be a self-adjoint closed operator on $\mc H$. Then $\{H\}''=\{e^{\im tH}:t\in\Rbb\}''$.
\end{pp}

\begin{proof}
Let $U_t=e^{\im tH}$. We have shown in the proof of Thm. \ref{lb54} that $\{H\}''\subset \mc M:=\{U_t:t\in\Rbb\}''$. The relation $\supset$ follows from Thm. \ref{lb50}.
\end{proof}



Let $f$ be a Lebesgue $L^1$ function on $\Rbb$. Its Fourier transform is $\wht f(s)=\int_\Rbb f(t)e^{-\im ts}dt$, which is a bounded continuous function on $\Rbb$. On the other hand, we can define $\int_\Rbb f(t)e^{-\im tH}dt$ to be the bounded operator sending any $\xi\in\mc H$ to the vector whose evaluation with every $\eta\in\mc H$ is
\begin{align*}
\Big\langle \big(\int_\Rbb f(t)e^{-\im tH}dt\big)\xi|\eta  \Big\rangle	=\int_\Rbb\big\langle f(t)e^{-\im tH}\xi|\eta\big\rangle dt.
\end{align*} 
The following proposition relates functional calculus and the one parameter group $e^{\im tH}$.


\begin{pp}
We have
\begin{align*}
\wht f(H)=	\int_\Rbb f(t)e^{-\im tH}dt.
\end{align*}
\end{pp}



\begin{proof}
By spectral theorem, it suffices to assume $\mc H=\bigoplus_{n\in\fk N}L^2(\Rbb,\mu_n)$ (where each $\mu_n$ is Borel) and $H$ is the multiplication of the function $x\in\Rbb\mapsto x$ on each direct sum component. Then for any $\xi=(g_n)_{n\in\fk N},\eta=(h_n)_{n\in\fk N}$ in $\mc H$, 
\begin{align*}
\int_\Rbb\bk{f(t)e^{-\im tH}\xi|\eta}dt=\int_\Rbb\sum_n \int_\Rbb f(t)e^{-\im ts}g_n(s)\ovl{h_n(s)}d\mu_n(s)dt.	
\end{align*}
The above sum and integrals are interchangeable, since the stuff to be integrated and summed are $L^1$. So the above expression becomes
\begin{align*}
\sum_n \int_\Rbb \int_\Rbb f(t)e^{-\im ts}g_n(s)\ovl{h_n(s)}dt d\mu_n(s)=	\sum_n \int_\Rbb  \wht f(s)g_n(s)\ovl{h_n(s)}dt d\mu_n(s),
\end{align*}
which equals $\bk{\wht f(H)\xi|\eta}$.
\end{proof}




\appendix
\section{Vector/operator valued holomorphic functions}\label{lb55}

Let $\mc B$ be a Banach space, and let $O$ be an open subset of $\Cbb$. A function $f:O\rightarrow\mc B$ is called \textbf{holomorphic} if the limit
\begin{align*}
	\lim_{w\rightarrow z}\frac{f(w)-f(z)}{w-z}
\end{align*}
exists for each $z\in O$. The limit is denoted by $f'(z)$ or $\partial_z f(z)$.

If $f:O\rightarrow\mc B$ is holomorphic, and $C$ is an oriented piecewise smooth curve in $O$, we define
\begin{align}
\int_C f(z)dz=\int_a^b f(\gamma(t))\gamma'(t)dt	
\end{align}
for any parametrization $\gamma:[a,b]\rightarrow O$ of $C$, and the right hand side can be understood as e.g. approximation in the norm topology of $\mc B$ using Riemann sums. Thus 
\begin{align*}
\varphi\Big(\int_Cf(z)dz\Big)=\int_C \varphi\circ f(z)dz
\end{align*}
for each bounded linear functional $\varphi\in\mc B^*$, which shows that our definition of $\int_Cfdz$ is independent of the choice of $\gamma$.


\begin{pp}\label{lb57}
	Assume $\mc B=\End(\mc H)$, $f:O\rightarrow\End(\mc H)$ is holomorphic, and $C$ is a piecewise smooth curve in $O$. Let $\ovl O=\{\ovl z:z\in O\}$. Let $\ovl C=\{\ovl z:z\in C\}$ whose orientation is the reflection of that of $C$ along the $x$-axis. Define $f^*:\ovl O\rightarrow\End(\mc H)$ by $f^*(z)=f(\ovl z)^*$ (i.e., the adjoint of $f(\ovl z)$). Then $f^*$ is holomorphic, and
	\begin{align}
		\Big(\int_C f(z)dz \Big)^*=\int_{\ovl C} f^*(z)dz.	
	\end{align}
\end{pp}


\begin{proof}
It is a straightforward check using definition that $f^*$ is holomorphic. If $\gamma:[a,b]\rightarrow\Cbb$ is a parametrization of $C$, then $\ovl \gamma:t\in[a,b]\mapsto \ovl{\gamma(t)}$ is one of $\ovl C$. Then
\begin{align*}
&\Big(\int_C f(z)dz \Big)^*=\Big(\int_a^b f(\gamma(t))\gamma'(t) \Big)^*=\int_a^b f(\gamma(t))^*\ovl\gamma'(t)dt\\
=&\int_a^b f^*(\ovl\gamma(t))\ovl\gamma'(t)dt=\int_{\ovl C} f^*(z)dz.
\end{align*}
\end{proof}



\begin{thm}\label{lb56}
Assume $f$ is continuous in the norm topology of $\mc B$. Let $\Phi$ be a set of (bounded) linear functionals of $\mc B$, separating in the sense that for any $x,y\in\mc B$, if $\phi(x)=\phi(y)$ for every $\phi\in\Phi$, then $x=y$. Assume that for each $\phi\in\Phi$, the function $\phi\circ f:O\rightarrow\Cbb$ is holomorphic, then $f$ is holomorphic.
\end{thm}

\begin{proof}
Choose any circle $C$ such that both $C$ and its inside is contained in $O$. Then for any $z\in O$, as $\varphi\circ f$ is holomorphic, 
\begin{align*}
\varphi(f(z))=\int_C \frac{\varphi(f(\zeta))}{\zeta-z}d\zeta.
\end{align*}
It follows that
\begin{align*}
f(z)=\int_C \frac {f(\zeta)}{\zeta-z}d\zeta,
\end{align*}
and hence
\begin{align*}
\frac{f(w)-f(z)}{w-z}=	\int_C \frac {f(\zeta)}{(\zeta-z)(\zeta-w)}d\zeta,
\end{align*}
which converges as $w\rightarrow z$, because the integrand converges uniformly with respect to $\zeta$.
\end{proof}


\begin{rem}
The above Proposition reduces the study of holomorphic operator/vector valued functions to that of ordinary ones. For instance, suppose $f_n:O\rightarrow\mc B$ is a sequence of holomorphic functions converging uniformly to a function $f:O\rightarrow\mc B$ on compact subsets of $O$. $f$ is clearly continuous. Since the evaluation of $f$ with any bounded linear functional of $\mc B$ is clearly holomorphic, we conclude that $f$ is holomorphic. As a special case, $\sum_{n\in\Nbb} a_nz^n$ (where each $a_n\in\mc B$) is holomorphic on any open set on which the series converges absolutely. 
\end{rem}






If $O$ is a subset of $\Cbb^n$, we say $f:O\rightarrow\mc B$ is \textbf{holomorphic} if $f=f(z_1,\dots,z_n)$ is continuous (with respect to the norm topology) and holomophic on each variable $z_j$ (when the other variables are fixed).
























\printindex	
	\begin{thebibliography}{999999}
		\footnotesize	
		
		
		
		
		%\bibitem[ABD04]{ABD04}
		%Abe, T., Buhl, G. and Dong, C., 2004. Rationality, regularity, and $C_2$-cofiniteness. Transactions of the American Mathematical Society, 356(8), pp.3391-3402.
		
\bibitem[Fol]{Fol}
Folland, G.B., 1999. Real analysis: modern techniques and their applications (Vol. 40). John Wiley \& Sons.	
		
\bibitem[Kad]{Kad}
Kadison, R.V. and Ringrose, J.R., 1983. Fundamentals of the theory of operator algebras. Volume I: Elementary Theory

\bibitem[Mun]{Mun}
Munkres, J., Topology, 2nd ed., Pearson Education.

\bibitem[RS]{RS}
Reed, M., Simon, B., Simon, B. and Simon, B., 1972. Methods of modern mathematical physics I: Functional Analysis. New York: Academic press.

\bibitem[Rud-R]{Rud-R}
Rudin, Walter. Real and Complex Analysis. New York: McGraw-Hill, 1987.
		
\bibitem[Rud-F]{Rud-F}
Rudin, W., 1991. Functional analysis. McGraw-Hill Science, Engineering \& Mathematics.
	
\bibitem[Sch]{Sch}
Schm\"udgen, K., 2012. Unbounded self-adjoint operators on Hilbert space (Graduate Text in Mathematics, Vol. 265). Springer Science \& Business Media.
		
		
		
		
		
	\end{thebibliography}
	%\noindent {\small \sc Department of Mathematics, Rutgers University, New Brunswick, USA.}
	
	\noindent {\textit{E-mail}}: binguimath@gmail.com
\end{document}